%That in the nature of the exercise that we have to contend with the vagueness of morality. And that this is not a weakness, but a critical component of earnestly engaging with a complex social problem.

%What we find is that different moral intuitions suggest different judgements and solutions that are incompatible with each other; and it isnt particularly clear how these incompatabilities should be resolved.

%Admittedly we could just directly state the elements of our judgement, and proceed without wider a consideration or perspective, but such an approach would miss a lot of humility and leave open the question why thoes elements should be selected.

%For instance, we could (as some have) simply designed or outlined a mechanism for the allocation of power which attempts to maximise the sum of people's utility; but supposing that such an implementation succeeds - would that outcome even be desirable?
%alternatively other systems sacrifice the maximisation of utility for more egalitarian purposes, and others do this to ensure that nobody could gain by market manipulation - would these constitute an improvement?

%Although there has been a lot of work on moral philosophy throughout the years (of which we can only begin to survey in this work) it is not clear that we have arrived at a singular ethical system that perfectly matches people's moral intuitions in a general circumstance. It is conceivable that a future science might bring such an invention.
%But unfortunately constructing such a pinnacle work is beyond our focus here.

%But as an adjoinder to that apology, 

%we feel that it is in the nature of the exercise that we have to contend with the vagueness of morality. And that this is not a weakness, but a critical component of earnestly engaging with a complex social problem.

%In writing about the mathematical equality embedded in the total internal self-consistency of moral Utilitarianism, Will Kymlicka writes in his well-cited textbook ``Contemporary Political Philosophy -- An Introduction'':
%\begin{displayquote}
%Political philosophy is not like logic, where the conclusion is meant to be already fully present in the premises. The idea of moral equality is too abstract for us to be able to deduce anything very specific from it. There are many different and conflicting kinds of equal treatment ... The question is which form of equal treatment best captures that deeper ideal of treating people as equals. This is not a question of logic. It is a moral question, whose answer depends on complex issues about the nature of human beings and their interests. In deciding which particular form of equal treatment best captures the idea of treating people as equals, we do no want a logician, who is versed in the art of logical deductions, We want someone who has an understanding of what it is about humans that deserves respect and concern, and of what kinds of activities best manifest that respect and concern.\\
%The idea of moral equality, while fundamental, is too abstract to serve as a premise from which we deduce a theory of justice. What we have in political argument is not a single premise and then competing deductions, but rather a single concept and then competing conceptions or interpretations of it. Each theory of justice is not \textit{deduced from} the ideal of equality, but rather \textit{aspires to} it, and each theory can be judged by how well it succeeds in that aspiration.
%... \\
%... utilitarianism seems implausible as an account of moral equality, at odds with our intuitions about that basic concept. But its implausibility is not a matter of logical error, and the strength of an [alternative] theory of fair shares is not a matter of logical proof. This may be unsatisfying to those accustomed to more rigorous forms of argument. But if the egalitarian suggestion is correct --- if each of these theories is aspiring to live up to the ideal of treating people as equals --- then this is the form that political argument must take. To demand that it achieve logical proof simply misunderstands the nature of the exercise. Any attempt to spell out and defend our beliefs about the principles which should govern the political community will take this form of comparing different conceptions of the concept of equality.
%\cite{kymlicka2002contemporary}
%\end{displayquote}

%It is from this perspective that we consider the space of moral concerns and subsequent technical considerations that bear upon the question of what kind of market mechanism \textit{should} be implemented in an electricity context
%And this is not to say that any particular system is \textit{as good} as any other but that we will proceed modestly, and critically reflect on the system we develop.






%What is notable is that differing specific interpretations of equality can and do conflict between themselves, in their implications, even as bearing on everyday life. There is no shortage of arguments between those positions (and alongside other moral and physical factors), as to which equalities are desirable, or even sensible - see section \ref{sec:equality_conflict}.

%A quick sampling of differing answers to the ``Equality of what?'' question is of the following list:
%\begin{itemize}
%    \item blah1
%    \item blah2
%    \item blah3
%    \item equality of income and/or wealth between peoples
%    \item equal opportunity to gain an income and/or wealth
%    \item equal opportunity to gain income and/or wealth across peoples lifespan
%    \item equal opportunity to enjoy income and/or wealth across peoples lifespan against racial/gender/background differences
%    \item equal opportunity to pursue a fulfilling life
%    \item equal access to resources
%    \item equal pay for equal work
%    \item equal opportunity to gain or equal possession of political/personal power
%    \item equal respect and consideration as persons
%    \item equal ability and unrestricted freedom to vote; except for prisoners.
%\end{itemize}




%The question of what specific considerations should warrant the imposition of differential treatment is quite contested,
%particular historic conceptions include atleast contract-utilitarianism, and Rawlsian Difference principle justifications.\cite{,}
%But there are no easy answers as to when/where/how different considerations should warrant (among others) the imposition of differential treatments.


%The idea of equality of opportunity is a widely adopted notion, and has several aspects of appeal.
%The central idea is that there should be an `even playing field' of fair competition for positions of responsibility/authority/reward/advantage, where nobody inherits unmerited advantage - or atleast that nobody should have arbitrary disadvantage.
%The philosophy has been the object of widespread political discussion and many book-length treatments.\cite{roemer_equalityofopportunity,mason2006levelling}
%Unfortunately we cannot give a thorough treatment here, except to relate some high-level observations of the arguments and general take-aways from the philosophy. The interested reader is encouraged to read further about the debates and details.

%Equality of Opportunity has several kinds appeal (see p56 \cite{kymlicka2002contemporary}):
%\begin{itemize}
%    \item It appeals to the idea that people's fate should be determined by their choices, rather than their circumstances. This contrasts strongly against dynamics of nepotism and entrenched social stratification.
%    \item It might appeal to our notion of a more efficient society, as the idea of a even playing field would competitively select better performing individuals for positions of social responsibility.
%    \item It can appeal to our ideas of deserving, in that wherever an individual competes fairly against other individual for a position or reward it is felt to be more deserved than if the process was unfairly biased.
%\end{itemize}

%However several problems and avenues of criticism arise in refining what is ment by an equality of opportunity: primarily concerning what precisely is meant by `opportunity', and what kind and scope that equalization should cover. And secondarily how differences in respective opportunities should be measured and also rectified.

%Consider -- what is an opportunity? and which opportunites should we equalise?
%For instance, John Roemer develops a algebra of opportunities in his book \cite{roemer_equalityofopportunity}, but generally states:

%\begin{displayquote}
%Indeed, an opportunity is a vague thing. It is not a school or a plate
%of nourishing food or a warm abode, but is, rather, a capacity which is
%brought into being by properly using that school, food, and hearth. It is not
%immediately obvious how to equalize opportunities, because they are not
%material things with self-evident sizes. 
%Building identical schools and staffing them with identical teachers in several communities in which children live in very different circumstances therefore will not generally equalize their opportunities for success: the commonly held view to the contrary is based on a fetishist error of identifying an opportunity with a material object that can at best help bring it about. 
%An opportunity, to use Cohen’s (1989) phrase, is an access to advantage.
%\cite{roemer_equalityofopportunity}
%\end{displayquote}

%Some components of peoples conception of what an opportunity may include:
%In a \textit{context}, an opportunity is something that a person can \textit{choose} to avail themselves of, which may/will result in a \textit{better}\footnote{usually equality of opportunities are framed positively, though nothing prohibits discussion about negative opportunities, or viewing a positive opportunity as the opportunity not to engage with a negative opportunity.} outcome for the person (in some sense), with some reasonable \textit{probability}.

%Equality of opportunity can be defined and argued for-and-against along these points ---

%\textbf{Choice \& Probability}: while it may be obtuse to be arguing between conceptions of free-will, the question of what is and should count as a choice is actually very relevant. (see entire chapter 7 of \cite{mason2006levelling})
%One consensus is that what counts as a choice is at-least as big as that which a person can be held to be morally responsible for\footnote{perhaps summarized by the saying `ought implies can'.} - which can be difficult to determine and possibly subject to interpretation of degrees.
%So for instance, the question of how much people's choices in life are determined/restricted/conditioned by environment and upbringing has a strong influence on the degree to which equalizing opportunities (say, educational ones) might translate into equalizing and/or compensating for circumstances.

%One conception is that choice may be related to (or perhaps measured by) one's ability `to do otherwise' or a degree of choice among alternatives - specifically, that if there is no alternative actions then there is no choice to be had at all.
%This raises questions about when and how relevant alternatives exist, and there is even conceptual space to consider choices and moral responsibility even in the total absence of alternatives (see frankfurt experiments).

%This contrasts other conceptions of choice (such as are compatible with determinism) such as defining a choice as a person's action without external coercion, or as amenable to a person's reasoning process (in some way).

%So, for instance, is discriminatory practices about religious peoples in anyway ameliorated by considering a person's particular religious practices as a matter of their own choice? even if they are raised in an environment where they would not have likely come to other beliefs?

%The choices that people can make can perhaps be divided into immediate and extended categories, where immediate choices are largely unmediated by external factors, whereas extended choices are primarily mediated by external factors - eg. the difference between `trying to get a job' and `getting a job' as both being choices.
%The extended choices that a person has can be thought to be mediated by probability; as an opportunity for promotion with a probability of zero is thought to be no opportunity at all. These extended choices that are primary focus for political equality of opportunity.
%This raises the question of when (and on what information) are the probabilities are to be calculated?
%An argument that draws out the question of the relationship between probability and opportunity is the baby swapping argument (Authoer X,Y,Z). We could also imagine a society where every job position was filled by pure lottery among the applicants. Would such situations qualify as `equal opportunity'?
% perhaps the probability for promotion is zero only because a more qualified candidate was already going to apply.

%The delineation of what reasonably counts as being an `equal' or `unequal' opportunity and a choice is important. As equality of opportunity seems primarily concerned with equalizing peoples ability to choose the conditions of their life, over their unchosen circumstances.

%\textbf{Context \& Betterment}: An example instance of a context, is the the workplace, and betterment is in terms of career advancement.  known also as `careers open to talents':
%\begin{displayquote}
%Originally formulated in the period of the French revolution, this position called for high public offices to be open to anyone able to fill them, not just nobility. The general extension of the principle is that jobs and educational positions should be filled on merit alone. Though widely proclaimed, it's still considered in practice to apply more strongly to public sector jobs than private business. For instance, laws against discrimination by sex and colour apply to all employers, but only public employers are supposed to ignore family connections and friendships.\cite{baker1992arguing}
%\end{displayquote}
%Even within this context, what should count as `merit' in attaining a position is sometimes rather clear, but sometimes not so much.
%Mason \cite{mason2006levelling} gives some examples: about jobs that require specific clothing can isolate the religious\footnote{construction firms requiring a Sikh to take off his turban to wear a hard-hat.}, with regards to careers that deal with with sex matters and/or sex appeal\footnote{specific customer service jobs may require a women who is a Muslim to wear a skirt}, or about hiring practices in a broader racist/sexist society\footnote{hiring a white man over an equally qualified black man may lead to greater workplace cohesion and sales.}; Indeed even the non-technical requirements of being a `good' employee are defined in the context of particular cultural practices\footnote{and perhaps even necessarily so}.
%`careers open to talents' is potentially just one narrow focus for a broader equality of opportunity position, and wider stances bringing on more and/or different considerations in their specification.\\

%However different conceptions of equality of opportunity (depending on their focus and breadth of context) have the potential to yield differing and conflicting answers on practical matters.
%Consider an example from Janet Richards\cite{} presentation of Christopher Jencks' character Ms Higgins:
%\begin{displayquote}
%Ms Higgins is a school teacher committed to equality of opportunity. She is anxious to spend her time and effort among her pupils accordingly, but quickly finds herself baffled. Should she make herself equally available for all, or give equal time to all, or give more time to children from deprived backgrounds, or try to make all the children equally proficient in everything by the end of the year, or think beyond the classroom to the effects her children will eventually have on the community or the world at large? Dozens of incompatible policies seem plausible as candidates of equality of opportunity.
%\end{displayquote}
%Therefore it makes some sense to characterize the space of possible opportunities that are subject to potential equalization.

%There are specific divisions between conceptions of particular opportunities, for instance there is sometimes expressed a difference between of opportunities that people are conceived as \textit{having}, against opportunities which people are \textit{given}.\cite{mason2006levelling}
%For instance, we can speak of general financial opportunity in life as something that a person can have, without easily identifying the people who would give it; or a particular job position as something more appropriately describable as being given by a distinct employer.

%Another specific way we can speak of opportunity as perhaps being between `means-regarding' and/or `prospects-regarding' forms. %Perhaps, in the sense that we can speak of a child's being in school as `an opportunity'.Janet Richards\cite{}
%Which emphasizing the opportunities tied to the specific means (such as specific school attendance); as opposed to the broader or multi-avenue opportunities (such as gaining entry into a favored career).

%We can also categorise equalities into the normatively inert descriptive equalities (of opportunity) and presecriptive ones.




%In a similar fashion to how equality before the law is regarded as being perhaps nessisary but not sufficient to ensure wider social equality (necessary insofar as there are envied or finite positions or scarse resources which can only be allocated to finite individuals).
%Even if we consider the most thoroughgoing equality of opportunity position - specifically Rawles `fair equality of opportunity' doctrine.\cite{sep-egalitarianism} - we might still have large social inequalities.
%John Baker puts the point bluntly:

%\begin{displayquote}
%The biggest problem is that principles of equal opportunity help to make systems of inequality seem reasonable and acceptable. They shift the whole issue away from whether inequalities of wealth, power, status, and education are themselves justifiable, to the question of how to distribute these inequalities, The implication is that as long as the competition for advantage is fair, advantage itself is beyond criticism. The winners feel entitled to their winnings and the losers blame themselves.
%\end{displayquote}

%Or another example, is that strict equality of opportunity would morally restrict parents giving preference or advantage to their own children (as argued by Mason).

%Perhaps a weak equality of opportunity is necessary but that `equality' is a concept that is wider or different than equality of opportunity.

%Notwithstanding people find specific equality of opportunity doctrines is specific fields particularly uncontroversial - An example of which is democratic equality.
%particularly that every member of a society is afforded equal right and capability to participate (and/or vote) and be informed about (some) governmental decisions (such as referenda and elections), and that this freedom should be irrespective of most exogenous conditions (gender, sex, height, race, etc).





\section{Some alternatives to equality}

In these particular contexts (and even across them), particular arguments can be made for and against the ideal of equal allocations specifically.
Well known and general modes of objection include contrasting equality with efficiency (per the `leveling-down objection'), and contrasting equality with particular freedoms (such as the freedom to buy/sell/give property, or prioritize the treatment of one's own family).



And even in these contexts, a person might value these allocations as morally worthwhile \textit{in themself} and/or perhaps \textit{instrumentally} as a means to other ends (such as promoting community, avoiding envy/favoritism etc).

While it may be usefull to sample the innumerable fields to which equality principles could be applied it makes a better approach to attempt to characterise the fields where people have a strong egalitarian predisposition against other specific feelings, and then to analyse why.



One potentially helpful way to consider the field of possible applications and to analyse the contexts of when/where/how equality principles can/should be applied is to re-frame claims to about equalities (against other principles) as if they were personal statements of moral rights for an arbitrary individual.

For instance, notions of equality particularly regarding equality of opportunity are often strongly conceptually associated with notions of \textit{impartiality} in a selection process.
Framing an impartiality principle as a statement of rights might go along the lines of: "I have a right not to be arbitrarily excluded from X".
Which may be felt to have significant weight depending on what the X is; for instance, if the X is voting in government referenda then we would tend to agree, but if X is 'being married to Mary Sue' we would tend to disagree - as it would run afoul of her rights claim of "I have a right to choose who I wont spend my life with".
A more complicated example is a quasi-egalitarian principle offered by Mason: "the effects of peoples difference in circumstances should never be such that some can acquire the resources that are necessary to lead a decent life whilst choosing not to work to earn an income when others are not able to do so.". Which could potentially be re-framed: "I have a right to be as undeservedly fortunate to be able to live decently without working as anybody else".
Note that while I think it can be usefull to make comparisons like this, it can be difficult to frame and compare some other moral ideas in this way, eg. utilitarianism is more difficult to frame as a first-person rights claim.\footnote{perhaps: "I have a right to count as much as others in decisions which divide utility."}

In anycase, this re-framing can make a specific equality claim a bit more personal and allows some direct comparisons between and against other rights claims. It also frames these moral positions in a manner that is potentially more likely to correspond with what people might feel and/or morally think.

It also encodes equality doctrines (and others) generally as social principles or processes for universal application - formal equality.



%All these kinds of rights claims have an equality of a very basic sense, as they are all with reference to any arbitrary indiviual `I' (ie. the indexical), however this usually not sufficient to embody anything notably `egalitarian'.

%From this perspective, the fields which people have strong egalitarian predispositions are fields where people may have strong affinity towards rights claims to quantities which are conceived of as having a relative character; such as might be equalized.
%It therefore follows that things which are naturally conceived of as important relative quantities are potential candidates for strong egalitarian predispositions.

%For instance, "I have a right to be paid as much as anyone else" and "I have a right to be paid the same hourly rate as everyone else" are both notably egalitarian in character (even if they are potentially conflicting) as they deal with money in relative terms among potential claimants, whereas "I have a right to be paid at-or-above \$5 per hour" is distinctly less egalitarian as it deals with money in absolute terms.
%For instance, `having a job' is not a relational property or itself a quantity, the numerical probability of `having a job' is a quantity but not necessarily a relational one, and the numerical probability of `having a job' above another's probability, is a relational quantity.



\subsubsection{Equality of bigger things}

If people have some degree of equality before the law, and are subjected to the same (or otherwise similar) rules, and if they have some degree of equality among opportunities in life, what more could eqaulity demand?
And here we have a greater degree of bifurcation of incompatible ideas about which bigger targets of equality should be pursued.

The most obvious target of wider equality doctrine is with regards to economic purchasing power and money.
In many ways, the quality of a person's life is atleast associated with the income that they are capable of generating. and inequalities of other kinds can be caused by differences in economic positions.
Particularly that economic advantage can be self self-sustaining, and economic disadvantage can be sticky.
However, for many equalising economic positions directly seems like a bad target.
A classic argument (see Hume, section 3, part 2 \cite{hume2006enquiry}) is that an equality of people's freedoms to buy/own/sell/give would lead to economic inequality, and that perfect economic equality would lead to severe restrictions on economic freedoms and privacy.

One of the more concrete equality ideas is that wealth and/or income should distributed equally between peoples.
Money and purchasing power is one example of a freedom which is afforded to us in society, and per doctrines of equality of income/wealth, that positions are favored which increase the inequality of this freedom between peoples.
in the strictest sense total equality of income/money is seen to be undesirable for several reasons.
particularly that whatever policy enforces equality of income/money is seen to interfere on the individual level between peoples and their trading.
Indeed ownership of property and the freedom to work are even in the UN declaration of Human Rights. additionally, it is seen that total and completel equality of money and income would remove any drive to wealth bruilding and creative enterprises.
Additionally it is not even certain whether total economic freedom is something which is desirable anyway, as having equal money does not additionally compensate thoes with disabilitites or bad luck.
or nessisarily disuade or punish thoes that squander their income/money that would be afforded to them by socieity.
One of the aspsects of consumerism is that having equal money does not nessisarily mean equal ability (or nessisarily any ability at all) to engage in activities which are viewed to be 'truly important'.
Indeed the equal ability to purchase knick-knacks is hardly what people really view as a morally desirable system.
The more abstract refinement of equality of money/income shifts the focus onto the effect that money bestows, namely the freedom to purchase, and actualize a persons desired, and/or essential processes and freedom capabilities in life.


simple equality in choices or money is not what is cheifly meant. That actual attainment of sufficient resources/position/money nessisary to live a good life is the important thing, not just arbitrary choices of alternatives amoung knick-knacks.
This measure of `living a good life' or summarily `welfare' is to be equalised.
This bumps up against the question of how to measure it, and also if it is subjective.
Should welfare be identified with the satisfying of desires in life (good or bad or immoral or antisocial ones?)
Or is welfare to be associated with happiness or other emotions - such as subject to an experience machine.\cite{nozick2013anarchy}
(https://medium.com/moral-robots/utilitarianism-in-robot-ethics-507fef1a3d59)

or against welfare equality, to deny that peoples differing attainment of their own good-life is a thing that can be reasonably compared; and hence cannot then be made equal.
These considerations aren't just simply arm-chair distinctions either, for instance one objection that has been made to welfare equality, is to deny that peoples differing attainment of their own good-life is a thing that can be reasonably compared, as being lesser or greater; and hence cannot then reasonably be made equal.

A most commonly discussed (and quite general) contention is over the `levelling down' objection, which prompts the consideration of when an hypothetical equality society is better/worse than a hypothetical unequal society which is better off (see parfit). Another general contention is where a particular equality contrasts with a persons specific rights, such as people claiming they should be compensated more money for greater work (contrasting economic equality) (see Baker), or that they should be able to choose arbitrarily who they work-with/live-with or marry (contrasting equal opportunity) (see Mason).


The technique again, is to move away from (various and contestable) numerical interpretations of equality into the more abstract conception, and thence see what implications follow.


%In a political context the word ``Equality'' has occasionally been used as a political slogan, with people attaching various feelings toward the word.
%For some it is a slippery term and an political umbrella device\footnote{see Janet Richard's description of progressively embracing disparate `equality of opportunity' doctrines, as sliding down a snake in a snakes-and-ladders game. Or Kymlicka's \cite{kymlicka2002contemporary} p152 consideration that the same `slippery-slope' argument surrounding where to draw the line between choice and circumstance tends to lead people to Libertarianism}.

%So there are principally two approaches to be taken, the first is to attempt to to take on a more descriptive process by characterizing the things which people tend to desire by the term `equality' and also despise by `inequality'.
%The second approach is to try and further refine a more central notion - what is it specifically about equality that we all find motivating?
%We will both approaches and relate them to electricity systems, but the second is more direct and we will take it first.



\subsection{that the `rich should not get richer' because they are rich}

That the rich should get richer because they are rich, violates peoples feelings regarding deservingness.
Best encoded by Marx' objection to CMC and MCM cycles of capitalistic wealth generation and exploitation.
One primary potential way to issue incentives is to make them non-monetary, or non-monetarisable. as Baker considers social incentives such as trophies and gratitude shows of reputation. while this is unrealistic perhaps there is other compensations that can be given.

Primarily can be seen by hindering the rewarding participants in terms of money.
perhaps in terms of future promise contracts for reciprocal electricity supply.
the schema should be made to facilitate these contracts.
Notwithstanding it is within some people's understanding of equality that dominance in one `sphere' should not sill over into other's.
It might not be easy to see how to prevent this from being the case - even in the event of micky mouse currencies.
but it is hardly foreseeable that any realistic electricity system should not feature monetary trade.

Alternatively and/or optionally, all that would need to be done is for regulatory body to imbue trading electricity with a lower expected rate of return (ROE) than other potential asset classes (see XYZ's economic theory) to disuade powerfull investors from the electricity system.
but would this be what people actually want? as it certainly wouldnt solve this feature of capitalism for society, only shifting it elsewhere; and it would inhibit the actualizing of economies of scale for consumers.





%One way in which a market mechanism can be judged is by the outcome that it produces and another way is by the process used to produce to that outcome.
%Both of these can be judged on whether it coincides with people's conceptions of what is \textit{fair} and/or \textit{just}; aswell as other example considerations such as \textit{efficiency} and \textit{practicality}, etc.

%In any system there is a structured choice of interractions and the coutcome they would produce, and the choices that people make can be
%revealed preferences.
%according to chosen actions that may or may not reveal preferences - allocating such things as money and energy between parties.

%And the choice of resultant distribution would also inform how the parties might interact with they system in the future.


%These questions of what constitutes a fair or unfair distribution of resources in society is fundamentally the problem of the branch of ethics and philosophy of distributive justice\cite{sep-justice-distributive}.
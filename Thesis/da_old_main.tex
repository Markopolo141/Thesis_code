\documentclass{article}
\usepackage[utf8]{inputenc}

\title{Thesis}
\author{Mark Burgess}
\date{June 2019}

\usepackage{natbib}
\usepackage{graphicx}
\usepackage{amsmath}
\usepackage{amssymb}
\usepackage{csquotes}

\DeclareMathOperator*{\argmin}{\arg\!\min}
\DeclareMathOperator*{\argmax}{\arg\!\max}

\begin{document}

\maketitle

\section{Introduction}



The way that electricity is being produced and consumed is changing in a direction not fully understood, and it has this before.

The means by which electricity has been produced and consumed between individuals and industry players has suffered structural changes over time and these changes are expected to continue into the future.
When Electricity was first introduced in Britain in the 1800s there were different companies offering the service of supplying electricity to the homes of individuals, these companies covered different regions of service in and between cities, and each operated according to their own guidelines on safety and service stability.
These electricity operators owned the poles and lines and supplied energy to houses per subscription at their own voltage and current specifications.
When electricity was first being introduced, its use was in direct competition to other technologies such as gas and coal/keroscene, and even though the electricity companies often had regional monopoly on electricity supply, ther competition between industries kept the pricing in-line.
Through the course of buisiness these companies eventually either went out of buisiness or consolidated into larger coporations, and proceeded to supply larger amounts of electricy at scale to the residents of Britain.
As the people's daily reliance on electricity increased in plurality and popularity of electrical appliances and lighting, greater public pressure came to the fore, for the regulation of the electricity grid.
Primarily for the regulation of Voltages and frequency and for the safe interopration of devices between networks but also to prevent price hikes and extortionate pricing of the electricity companies.



The way that electricity is being produced and consumed is changing in a direction not fully understood, and it has this before.

The means by which electricity has been produced and consumed between individuals and industry players has suffered structural changes over time and these changes are expected to continue into the future.
The challenges today are reminiscient of the challenges historical.
The introduction of new technologies and sources of energy into the strata of society is a process that is prone to problems and features that are historical, political and social.
Before discussing the process of new technologies and their integration into the electricity grid, we will divolve the discussion into some similar features of the historical introduction of electrical grids themselves.
One example of a tension with the introduction of new technology is its regulation.
Today there are institutions such as AEMO and AER which provide industry regulations to incentivise competition and ease monopolising of indistry participants over the concerns of individuals.
It is worth noting that the institution of overly-tight regulations can and potentially has actually hampered the deployment of new technologies.
A particular example from British history is the Electric Lighting act of 1882 which occured in the context of an economic resession; and which was strongly blamed for reducing Brtiain's uptake of electrical energy for a period of 4 years.
The government at the time, was regional-socialist, and



\section{Distributive Justice and Fairness evaluation}

%\begin{displayquote}
%My claim is merely that there is no single fundamental principle that determines or provides guidance on what justice requires in relation to the distribution of access to overall advantage.\cite{mason2006levelling} 
%\end{displayquote}

The electricity grid is seen as an evolving system of increasingly interconnected and complex devices, providing a structured platform for the exchange of power and services between people.
However at the heart of this evolving construction there is a morally ambiguous question - how should it be determined how much people \textit{should} pay or be-paid for the power they consume or generate?

Or, to ask the question with more granularity: when the possible power-flows on an electricity network are valued and influenced by participants differently, 
what is a reasonable cooperative outcome for the network and what monetary transactions should occur between the participants in that case? -- and this is to ask: how should electricity be traded?\\

In considering the question of what distribution energy market structure \textit{should} be implemented, it is essential to consider a range of moral and practical factors that bear on the question.
If one particular market mechanism \textit{should} be implemented over another,
or to say that one market mechanism is \textit{better} than another, begs the question of how the evaluation was determined.

What factors make one market mechanism \textit{better than} another?

One way in which a market mechanism can be judged is by the outcome that it produces and another way is by the process used to produce to that outcome.
Both of these can be judged on whether it coincides with people's conceptions of what is \textit{fair} and/or \textit{just}; aswell as other example considerations such as \textit{efficiency} and \textit{practicality}, etc.

In this Section we consider the moral, social and philosophical considerations that frame the question about what mechanism should be implemented.
These considerations range much wider than the specific context of electricity markets. And we elucidate them here to help articulate the motivations and perspective that a reader may-or-may not have when it comes to a particular mechanism for the allocation of electrical power; and to frame our process towards our particular solution.

The ethical side of our question is broadly associated with a branch of moral philosophy called `Distributive Justice', which seeks to ask and make headway on the question of how resources (such as money/power/goods/etc) should be distributed in society; and particularly, what does `fair' allocation mean? --- this frames our central ethical component.

Our development also relates to mechanisms which are intentioned to be practical, and a brief elucidation of some economic and game-theory concepts will ultimately be involved. However we begin with the ethical considerations.

What is evident is that different people have different conceptions of how the world should be, and not all of these conceptions are compatible with each other.
In constructing any particular answer to the question of what counts as a good/better/fair mechanism we will necessarily isolate those who would disagree.

But rather than simply constructing an answer to the question \textit{de novo}, the situation leads us to consider the space of peoples conflicting moral intutions before settling down to develop apon a judgement.
What we find is that different moral intuitions suggest different judgements and solutions that are incompatible with each other; and it isnt particularly clear how these incompatabilities should be resolved.

There is a long history of philosophical skepticism about the nature of moral knowledge and judgements.
For instance, Hume's Guillotine \cite{HumeGutenberg}\footnote{``For as this ought, or ought not, expresses some new relation or affirmation, it is necessary that it should be observed and explained; and at the same time that a reason should be given, for what seems altogether inconceivable, how this new relation can be a deduction from others, which are entirely different from it. But as authors do not commonly use this precaution, I shall presume to recommend it to the readers; and am persuaded, that this small attention would subvert all the vulgar systems of morality, and let us see, that the distinction of vice and virtue is not founded merely on the relations of objects, nor is perceived by reason.'' T3.1.1} is often read as stating that no material facts about the physical world as-it-is, could ever by-itself seem to justify any claim about how the world \textit{should} be.
Or as another instance, in G.E. Moore's open-question argument \cite{MooreGutenberg}\footnote{
``Moreover any one can easily convince himself by inspection that the predicate of this proposition—`good'—is positively different from the notion of ‘desiring to desire' which enters into its subject: `That we should desire to desire A is good’ is not merely equivalent to `That A should be good is good.' ... clearly that we have two different notions before our minds.''Ch1:13\\
``If I am asked ‘What is good?’ my answer is that good is good, and that is the end of the matter. Or if I am asked ‘How is good to be defined?’ my answer is that it cannot be defined, and that is all I have to say about it.'' Ch 1:6}, that for anything which is equal to what is morally good, then a question-about or statement-of that equivalence would only be a meaningless tautology - and hence moral goodness cannot meaningfully be defined.
Such arguments are probably best used as discussion-starters today, however, talking about the nature and basis of moral knowledge is not our focus.
Whether or not moral truth ultimately amounts to collective sentiment, or reduces to  statements of prudence, or is metaphysically identical/grounded in some deeper objective fact is beyond the scope of this work.
Instead, our source of moral considerations extends from the moral intuitions that people are likely to have upon reflection and consideration; and for that we survey the attitudes expressed in literature between thinkers.

Admittedly we could just directly state the elements of our judgement, and proceed without wider a consideration or perspective, but such an approach would miss a lot of humility and leave open the question why thoes elements should be selected.
For instance, we could (as some have) simply designed or outlined a mechanism for the allocation of power which attempts to maximise the sum of people's utility; but supposing that such an implementation succeeds - would that outcome even be desirable?
alternatively other systems sacrifice the maximisation of utility for more egalitarian purposes, and others do this to ensure that nobody could gain by market manipulation - would these constitute an improvement?

Although there has been a lot of work on moral philosophy throughout the years (of which we can only begin to survey in this work) it is not clear that we have arrived at a singular ethical system that perfectly matches people's moral intuitions in a general circumstance. It is conceivable that a future science might bring such an invention.
But unfortunately constructing such a pinnacle work is beyond our focus here.
Instead we give a brief survey of what we believe are some of the elements in people's moral thinking, and do our best to draw out a particular synthesis which bears direct relevance to electricity systems.

Unfortunately it must be noted that any particular ethical system is likely to be rooted in particular maxims and axioms and give outcomes that may be disagreeable to some people and agreeable to others.
And for this we must make a modest apology.

But as an adjoinder to that apology, we feel that it is in the nature of the exercise that we have to contend with the vagueness of morality. And that this is not a weakness, but a critical component of earnestly engaging with a complex social problem.

In writing about the mathematical equality embedded in the total internal self-consistency of moral Utilitarianism, Will Kymlicka writes in his well-cited textbook ``Contemporary Political Philosophy -- An Introduction'':
\begin{displayquote}
Political philosophy is not like logic, where the conclusion is meant to be already fully present in the premises. The idea of moral equality is too abstract for us to be able to deduce anything very specific from it. There are many different and conflicting kinds of equal treatment ... The question is which form of equal treatment best captures that deeper ideal of treating people as equals. This is not a question of logic. It is a moral question, whose answer depends on complex issues about the nature of human beings and their interests. In deciding which particular form of equal treatment best captures the idea of treating people as equals, we do no want a logician, who is versed in the art of logical deductions, We want someone who has an understanding of what it is about humans that deserves respect and concern, and of what kinds of activities best manifest that respect and concern.\\
The idea of moral equality, while fundamental, is too abstract to serve as a premise from which we deduce a theory of justice. What we have in political argument is not a single premise and then competing deductions, but rather a single concept and then competing conceptions or interpretations of it. Each theory of justice is not \textit{deduced from} the ideal of equality, but rather \textit{aspires to} it, and each theory can be judged by how well it succeeds in that aspiration.
... \\
... utilitarianism seems implausible as an account of moral equality, at odds with our intuitions about that basic concept. But its implausibility is not a matter of logical error, and the strength of an [alternative] theory of fair shares is not a matter of logical proof. This may be unsatisfying to those accustomed to more rigorous forms of argument. But if the egalitarian suggestion is correct --- if each of these theories is aspiring to live up to the ideal of treating people as equals --- then this is the form that political argument must take. To demand that it achieve logical proof simply misunderstands the nature of the exercise. Any attempt to spell out and defend our beliefs about the principles which should govern the political community will take this form of comparing different conceptions of the concept of equality.
\cite{kymlicka2002contemporary}
\end{displayquote}


It is from this perspective that we consider the space of moral concerns and subsequent technical considerations that bear upon the question of what kind of market mechanism \textit{should} be implemented in an electricity context
And this is not to say that any particular system is \textit{as good} as any other but that we will proceed modestly, and critically reflect on the system we develop.

Let us begin.

\subsection{Judgment by the fairness of the outcome}

The choice of centralized Market structures and process can be seen as a choice between methods of allocating resources between multiple parties in a system based on their interactions with it.
An example market structure might be a type of auction, and the interractions of the parties might be their choices of bidding or bidding strategy.
%In any system there is a structured choice of interractions and the coutcome they would produce, and the choices that people make can be
%revealed preferences.
%according to chosen actions that may or may not reveal preferences - allocating such things as money and energy between parties.
In this context the choice of the resultant distribution of resources can be viewed as being morally/socially desirable or undesirable. And the choice of resultant distribution would also inform how the parties might interact with they system in the future.

%These questions of what constitutes a fair or unfair distribution of resources in society is fundamentally the problem of the branch of ethics and philosophy of distributive justice\cite{sep-justice-distributive}.

In the process of discussing a choice of system it is important to come to a relatively clear understanding of what concepts are in play. And in these sections we will attempt to survey and break-down some of the surrounding moral concepts.

Throughout time there have been an array of philosophers who have discussed the problem of distribution, and the various ideas surrounding the allocation and distribution of resources and capital. And one of the major ideas surrounding the notion of fairness is \textit{Equality}.

\subsubsection{Equality}

\begin{displayquote}
``A common characteristic of virtually all the approaches to the ethics of social arrangements that have stood the test of time is to want equality of \textit{something}... They are all `egalitarians' in some essential way ... To see the battle as one between thoes `in favor of' and thoes `against' equality (as the problem is often posed in the literature) is to miss something central to the subject."\cite{18084} 
\end{displayquote}

\begin{displayquote}
``for all men have some natural inclination to justice ... what is equal appears just, and is so; but not to all; only among those who are equals: and what is unequal appears just, and is so; but not to all, only amongst those who are unequals;\\
which circumstance some people neglect, and therefore judge ill; the reason for which is, they judge for themselves, and every one almost is the worst judge in his own cause." Aristottle, Politics, chapter III.9\cite{AristotleGutenberg}
\end{displayquote}

People tend to believe that they are, should be, or be treated, `equal' in some sense.
And this broad conception has changed throughout time and place in history \cite{themeaningofequalitycapaldi}.
From at least as far back as Aristotle \cite{AristotleGutenberg}\footnote{see section quote}, notions and concepts about equality have come from across culture and peoples, and between Spiritual \footnote{across multiple religions, eg. in Islam ``No Arab is superior to a non-Arab, no colored person to a  white person, or a white person to a colored person except by Taqwa (piety)." [Ahmad and At-Tirmithi], and in Christianity, St Paul's Galatians 3:28 ``There is neither Jew nor Gentile, neither slave nor free, nor is there male and female, for you are all one in Christ Jesus'' (NIV) } and the Materialist \footnote{Such as in Engel's Anti-D\"{u}hring Part 1 Chapter 10 ``The idea that all men, as men, have something in common, and that to that extent they are equal, is of course primeval. But the modern demand for equality is something entirely different from that; this consists rather in deducing from that common quality of being human, from that equality of men as men, a claim to equal political social status for all human beings''} thought.
Throughout the ages the way in which equality in socieity has been constructed and implemented has varied dramatically - even the abolition of slavery is recent in human history.

There is something appealing about the idea of Equality between people, even if it is difficult to nail down exactly how and why.
From the asthetic perspective equality is an ideal and simple structure. From a humanitarian perspective equality is associated with relief from envy and want. From the social perspective it is associated with community and solidarity. From the philosophical perspective equality is directly associated with fundamental and core ideas.

What is morality and justice? and what Rights and Freedoms does it require? and are people actually equal in any morally relevant way?

On a practical level, the divergences between people's ideas of equality can be seen as regarding what things should-be equal (when, where and for whom); and also what should be done about respective inequalities as may exist.
And also on a theoretical level, how/if these ideas are given justification.

The question: ``Equality of what?'' can have many answers, some of which are commonly held and seldom controversial today, such as might be gleamed from the United Nation's declaration of Human Rights: Equality before the Law, Democratic Equality to vote, equal freedoms to marry and to live, etc.\cite{udhr}
However more controversial answers tend to have broader social and political scope, such as: Equality of Opportunity, and Equality of Welfare and/or Economic Equality. And the controversial nature of these ideas notably come to the fore in discussions such as surrounding affirmative action initiatives and also political socialism.

For some, equality is a contestable notion, or an ideal for the direction of efforts in narrow and specific contexts, but for others equality (conceived precisely or more broadly) is an attainable and far-reaching goal with multifaceted implications apon processes across and between social `spheres' \cite{walzer2008spheres,millerandwalzer,baker1992arguing}.

What is notable is that differing specific interpretations of equality can and do conflict between themselves, in their implications, even as bearing on everyday life. There is no shortage of arguments between those positions (and alongside other moral and physical factors), as to which equalities are desirable, or even sensible - see section \ref{sec:equality_conflict}.

A quick sampling of differing answers to the ``Equality of what?'' question is of the following list:
\begin{itemize}
    \item blah1
    \item blah2
    \item blah3
%    \item equality of income and/or wealth between peoples
%    \item equal opportunity to gain an income and/or wealth
%    \item equal opportunity to gain income and/or wealth across peoples lifespan
%    \item equal opportunity to enjoy income and/or wealth across peoples lifespan against racial/gender/background differences
%    \item equal opportunity to pursue a fulfilling life
%    \item equal access to resources
%    \item equal pay for equal work
%    \item equal opportunity to gain or equal possession of political/personal power
%    \item equal respect and consideration as persons
%    \item equal ability and unrestricted freedom to vote; except for prisoners.
\end{itemize}

And further specific ideas about equality diverge even still, however we briefly discuss some these theses to demonstrate the disagreement between them.

\subsection{Some Answers and how they measure}\label{sec:equality_conflict}

\subsubsection{Formal Equality, and of Equal Status}

One primary notion surrounding the concept of equality is that people should be subject to systems that treat them in a manner that is \textit{impartial}. Although it is potentially difficult to define, the minimal idea is that an impartial system should not afford special treatment toward any particular individual, or (perhaps) group of individuals. That the system should be designed to be blind to particular identity, and sensitive to morally relevant characteristics.

In literature this idea of moral impartiality has been elucidated by various thought experiments and also stated with moral maxims.
Particularly famous examples such as imaginative devices in Rawl's `veil of ignorance' and Hare's "ideal sympathiser", Kant's Universality categorical imperative, etc.

The particular idea of impartiality is rather mathematically expressible, in that people who are (in all relevant ways) equal should be treated identically. (see Aristotle quote) This maxim that `equals should be treated equally' is sometimes known as formal equality.\cite{whatisbasicequalitynathan}
Although formal equality is (at some level) considered to be an important component of any equal system, it is generally seen to be insufficient to be itself (or to `capture') broader notions of moral equality and social justice.

For instance, an uncontroversial example of such a formal equality doctrine is `equality before the Law'; as embodied in Article 7 of the Universal Declaration of Human Rights: 
\begin{displayquote}
``All are equal before the law and are entitled without any discrimination to equal protection of the law"\cite{udhr}
\end{displayquote}
Many people might consider such a provision to be an necessary constituent of a just and fair society.
What is noteworthy is that systems of Law generally encode conditions from which people will be treated differently and unequally, and even occasionally have different applicable laws for different groups of people - such as depending on their relevant status or societal role.
Examples include lawful specific freedoms and duties for doctors and laywers, rights and responsibilities for parents, provisions for children and directions for public officials, etc.
And these laws, even though they are in a very straightforward sense discriminating, do not impede a doctine of equality before the law, since these identities (doctor,lawyer,child,official) are regarded as being morally relevant categories for differential treatment, within which people ought be treated identically.

Although it is thought to be necessary, equality before the law (by itself) is often thought to be wildly insufficient to capture and ensure the broader concepts of moral equality and social justice:
\begin{displayquote}
In its majestic equality, the law forbids rich and poor alike to sleep under bridges, beg in the streets and steal loaves of bread.\\
--Anatole France, Le Lys Rouge [The Red Lily] (1894), ch. 7
\end{displayquote}

However there is perhaps some ground to interpret `equality' between people as being (or being `captured by') a kind of formal equality.
One potential way of interpreting the broader concept of moral `equality', is that everybody should have equal status, where status is interpreted to be the conditions among which (at least some) certain moral/social principles depend.
\footnote{This contrasts against other kinds of `status' as we will get to in chapter \ref{blah}}
Under this interpretation, and bluntly put, when people are treated `equally' it means that they are subject to the same rules:
\begin{displayquote}
``The idea that two persons are prescriptively equal presupposes a standard for equal treatment that both satisfy. Prescriptive equality supposes a principle. What, then, do we make of ``equal status'' or ``equal value''? ``Status'' involves no more than the fact that the thing falls under a principle. We need only identify the rules, and not the prior value of individuals, in order to have a full account of what we ought to do and why.
...
it seems that ``equal value'' means nothing more than ``falling under the same rule.'' ...''\cite{whatisbasicequalitynathan}
\end{displayquote}
In this sense, that people's having `equality' might be interpreted to be identical to formal equality, perhaps generally, or in the context of an idealised or sufficiently good rule-set.

However if this route is taken, there is some ambiguity as to what should be considered as a set of rules or principles by which people can have this kind of equality.
As we might imagine that a society with lawful slavery might have a singularly coherent and universally applied rule-set; in that context it might be said that people would effectively be subject to different rule-sets, but that is also true (in-degrees) of societies generally.

Indeed by imagination almost any treatment or process could be rationalised as being by a principle which is universally applied. And in this way it might be said that any system technically satisfies a formal equality.
Although formal equality such as  `equality before the law' might be a widely held ideal, and may not in-practice be perfectly implemented in a real context; and this may not technically be problematic. If we conversely consider a society in which everyone were technically subject to different rules and treatment (which is inevitably in-practice actually true), we could still potentially thence argue if it was `equal' or just in a social sense. and therefore what value is formal equality anyway?

Yet perhaps the focus should turn away from considering technicalities and analytic considerations to the much more vague notion of degrees of inequalities between different kinds of rule-sets and treatments that people can be said to be effectively subject to in-practice.
perhaps in answer to Anatole France's quote, that a lawful system that provided similar impediments to the rich obtaining shelter, money and sustenance would respectively be more equal.

%In considering equality before the law, a central component to be gleamed is that is something that is valued, it is probably better that is is amenable to be valued by people, and hence publicly disclosed; which is essential component.
%rather than simply that all should be subject to the same rules, but that those rules should be publicly disclosed. , and that these exceptions are not encoded into public Law.

A particular of formal equality is that it should not be directly partial to specific and particular individuals / groups, is a very low bar. But above that, it is perhaps better to characterise a system not by whether or not it is impartial, but rather by what characteristics it is (and is not) partial to.
And hence the question then turns to being primarily about what characteristics should be considered relevant for differential treatment, and henceforth what differential treatment should result.

The particular take-away from considering formal equality and equalty before the law, is that whatever principles of allocation or consequence that are endorsed, that they should be:
\begin{itemize}
    \item not \textit{partial} to specific persons and groups - atleast not directly.
    \item impose differential treatment only in-light-of morally defensible considerations.
    \item considering and/or minimising the difference in treatment and processes that is imposed upon people/s \textit{in practice}.
 %   \item publicly seen to be impartial by these factors.
\end{itemize}

The question of what specific considerations should warrant the imposition of differential treatment is quite contested,
particular historic conceptions include atleast contract-utilitarianism, and Rawlsian Difference principle justifications.\cite{,}
But there are no easy answers as to when/where/how different considerations should warrant (among others) the imposition of differential treatments.

\subsubsection{Equality of Opportunity, and of Equal Capabilities}

The idea of equality of opportunity is a widely adopted notion, and has several aspects of appeal.
The central idea is that there should be an `even playing field' of fair competition for positions of responsibility/authority/reward/advantage, where nobody inherits unmerited advantage - or atleast that nobody should have arbitrary disadvantage.
The philosophy has been the object of widespread political discussion and many book-length treatments.\cite{roemer_equalityofopportunity,mason2006levelling}
Unfortunately we cannot give a thorough treatment here, except to relate some high-level observations of the arguments and general take-aways from the philosophy. The interested reader is encouraged to read further about the debates and details.

Equality of Opportunity has several kinds appeal (see p56 \cite{kymlicka2002contemporary}):
\begin{itemize}
    \item It appeals to the idea that people's fate should be determined by their choices, rather than their circumstances. This contrasts strongly against dynamics of nepotism and entrenched social stratification.
    \item It might appeal to our notion of a more efficient society, as the idea of a even playing field would competitively select better performing individuals for positions of social responsibility.
    \item It can appeal to our ideas of deserving, in that wherever an individual competes fairly against other individual for a position or reward it is felt to be more deserved than if the process was unfairly biased.
\end{itemize}

However several problems and avenues of criticism arise in refining what is ment by an equality of opportunity: primarily concerning what precisely is meant by `opportunity', and what kind and scope that equalization should cover. And secondarily how differences in respective opportunities should be measured and also rectified.

Consider -- what is an opportunity? and which opportunites should we equalise?
For instance, John Roemer develops a algebra of opportunities in his book \cite{roemer_equalityofopportunity}, but generally states:

\begin{displayquote}
Indeed, an opportunity is a vague thing. It is not a school or a plate
of nourishing food or a warm abode, but is, rather, a capacity which is
brought into being by properly using that school, food, and hearth. It is not
immediately obvious how to equalize opportunities, because they are not
material things with self-evident sizes. 
Building identical schools and staffing them with identical teachers in several communities in which children live in very different circumstances therefore will not generally equalize their opportunities for success: the commonly held view to the contrary is based on a fetishist error of identifying an opportunity with a material object that can at best help bring it about. 
An opportunity, to use Cohen’s (1989) phrase, is an access to advantage.
\cite{roemer_equalityofopportunity}
\end{displayquote}

Some components of peoples conception of what an opportunity may include:
In a \textit{context}, an opportunity is something that a person can \textit{choose} to avail themselves of, which may/will result in a \textit{better}\footnote{usually equality of opportunities are framed positively, though nothing prohibits discussion about negative opportunities, or viewing a positive opportunity as the opportunity not to engage with a negative opportunity.} outcome for the person (in some sense), with some reasonable \textit{probability}.

Equality of opportunity can be defined and argued for-and-against along these points ---

\textbf{Choice \& Probability}: while it may be obtuse to be arguing between conceptions of free-will, the question of what is and should count as a choice is actually very relevant. (see entire chapter 7 of \cite{mason2006levelling})
One consensus is that what counts as a choice is at-least as big as that which a person can be held to be morally responsible for\footnote{perhaps summarized by the saying `ought implies can'.} - which can be difficult to determine and possibly subject to interpretation of degrees.
So for instance, the question of how much people's choices in life are determined/restricted/conditioned by environment and upbringing has a strong influence on the degree to which equalizing opportunities (say, educational ones) might translate into equalizing and/or compensating for circumstances.

One conception is that choice may be related to (or perhaps measured by) one's ability `to do otherwise' or a degree of choice among alternatives - specifically, that if there is no alternative actions then there is no choice to be had at all.
This raises questions about when and how relevant alternatives exist, and there is even conceptual space to consider choices and moral responsibility even in the total absence of alternatives (see frankfurt experiments).

This contrasts other conceptions of choice (such as are compatible with determinism) such as defining a choice as a person's action without external coercion, or as amenable to a person's reasoning process (in some way).

So, for instance, is discriminatory practices about religious peoples in anyway ameliorated by considering a person's particular religious practices as a matter of their own choice? even if they are raised in an environment where they would not have likely come to other beliefs?

The choices that people can make can perhaps be divided into immediate and extended categories, where immediate choices are largely unmediated by external factors, whereas extended choices are primarily mediated by external factors - eg. the difference between `trying to get a job' and `getting a job' as both being choices.
The extended choices that a person has can be thought to be mediated by probability; as an opportunity for promotion with a probability of zero is thought to be no opportunity at all. These extended choices that are primary focus for political equality of opportunity.
This raises the question of when (and on what information) are the probabilities are to be calculated?
An argument that draws out the question of the relationship between probability and opportunity is the baby swapping argument (Authoer X,Y,Z). We could also imagine a society where every job position was filled by pure lottery among the applicants. Would such situations qualify as `equal opportunity'?
% perhaps the probability for promotion is zero only because a more qualified candidate was already going to apply.

The delineation of what reasonably counts as being an `equal' or `unequal' opportunity and a choice is important. As equality of opportunity seems primarily concerned with equalizing peoples ability to choose the conditions of their life, over their unchosen circumstances.

\textbf{Context \& Betterment}: An example instance of a context, is the the workplace, and betterment is in terms of career advancement.  known also as `careers open to talents':
\begin{displayquote}
Originally formulated in the period of the French revolution, this position called for high public offices to be open to anyone able to fill them, not just nobility. The general extension of the principle is that jobs and educational positions should be filled on merit alone. Though widely proclaimed, it's still considered in practice to apply more strongly to public sector jobs than private business. For instance, laws against discrimination by sex and colour apply to all employers, but only public employers are supposed to ignore family connections and friendships.\cite{baker1992arguing}
\end{displayquote}
Even within this context, what should count as `merit' in attaining a position is sometimes rather clear, but sometimes not so much.
Mason \cite{mason2006levelling} gives some examples: about jobs that require specific clothing can isolate the religious\footnote{construction firms requiring a Sikh to take off his turban to wear a hard-hat.}, with regards to careers that deal with with sex matters and/or sex appeal\footnote{specific customer service jobs may require a women who is a Muslim to wear a skirt}, or about hiring practices in a broader racist/sexist society\footnote{hiring a white man over an equally qualified black man may lead to greater workplace cohesion and sales.}; Indeed even the non-technical requirements of being a `good' employee are defined in the context of particular cultural practices\footnote{and perhaps even necessarily so}.
`careers open to talents' is potentially just one narrow focus for a broader equality of opportunity position, and wider stances bringing on more and/or different considerations in their specification.\\

However different conceptions of equality of opportunity (depending on their focus and breadth of context) have the potential to yield differing and conflicting answers on practical matters.
Consider an example from Janet Richards\cite{} presentation of Christopher Jencks' character Ms Higgins:
\begin{displayquote}
Ms Higgins is a school teacher committed to equality of opportunity. She is anxious to spend her time and effort among her pupils accordingly, but quickly finds herself baffled. Should she make herself equally available for all, or give equal time to all, or give more time to children from deprived backgrounds, or try to make all the children equally proficient in everything by the end of the year, or think beyond the classroom to the effects her children will eventually have on the community or the world at large? Dozens of incompatible policies seem plausible as candidates of equality of opportunity.
\end{displayquote}
Therefore it makes some sense to characterize the space of possible opportunities that are subject to potential equalization.

There are specific divisions between conceptions of particular opportunities, for instance there is sometimes expressed a difference between of opportunities that people are conceived as \textit{having}, against opportunities which people are \textit{given}.\cite{mason2006levelling}
For instance, we can speak of general financial opportunity in life as something that a person can have, without easily identifying the people who would give it; or a particular job position as something more appropriately describable as being given by a distinct employer.

Another specific way we can speak of opportunity as perhaps being between `means-regarding' and/or `prospects-regarding' forms. %Perhaps, in the sense that we can speak of a child's being in school as `an opportunity'.Janet Richards\cite{}
Which emphasizing the opportunities tied to the specific means (such as specific school attendance); as opposed to the broader or multi-avenue opportunities (such as gaining entry into a favored career).

We can also categorise equalities into the normatively inert descriptive equalities (of opportunity) and presecriptive ones.

Although these divisions seem a bit vague and fussy, a better division between peoples conceptions of opportunity can be seen, by viewing particular opportunities as being positive or negative freedoms.\footnote{this connection is not made by me, see works by authors A and B.}
Although admittedly vague, the difference between a positive and negative freedom has distinct philosophical treatment (atleast as far back as Kant \cite{}), and has known moral importance in several contexts.
A negative freedom is the absence of specifically identifiable obstacles, barriers, or coercion about executing a particular action; it is more directly about external material circumstances and the absence of particular things.
By contrast, notions of positive freedom tend to be more expansive, and can relate to possible choices and control that might be realized by a person to bring about their purposes in life; it can touch on what is internal and mental, and seems to be more about the presence of agency.

In the context of equality of opportunity, the negative freedom of being externally unhindered about conducting an action might be considered as a prerequisite to the positive freedom of actually having a choice about its execution.
And hence a particular equality of opportunity which is conceived negatively has a  potentially wider base of appeal than its positive counterpart.

So for instance, in the context of opportunity for educational attainment, the difference between opportunity concieved more negatively (equal freedom from specific obstructions to attainment - eg. class structure, racial discrimination, etc) is different from the equality of opportunity conceived more positively (having equal propensity in choice to attain).

The difference between positive and negative freedoms is vague but can be seen to have significant moral relevance, particularly in the consideration of cases about doing vs allowing harm to another person.
The distinction of moral relevance between doing vs allowing harm is one of the cornerstone cases against consequentialist morality; and is encoded in most legal systems.
For instance, various thought experiments exist asking what the moral and legal difference should be between drowning another person, and neglecting to save another person from drowning; or between active and passive euthenasia.
If a person has a right to negative freedom about a particular domain, then that counts as a right to be free of others doing harm to them in that context (as that would count as an introduction of a specifically identifiable obstacle).
%If a person has a right to be free of others' doing harm to them, then that counts as a negative freedom right from such external harm.
Whereas If a person has a right to positive freedom about a particular domain, then that would seem to count as a right to be free of others even allowing harm to them in that context (an example of a positive freedom might be to live free of absolute poverty).

Some of the vagueness between positive and negative freedoms can be resolved by considering them as being about an action/s, above a set of factors.
If the set of factors is very specific and narrow - such as above racial discrimination or physical disability - then it characteristically more negative.
If the set of factors is very expansive - such as above anything that might impinge apon a person's agency at all - then it is characteristically more positive.
This can be seen in Gerald MacCallum'm presentation (1967) (stanford encyclopedia's) of freedoms as triadic relations. Specifcally that a liberty or freedom can be considered as a triadic relation between an agent, an action and preventing conditions.

Particularly the context about people's having negative freedom from other's causing harm brings to the fore the distinction between positive and negative actions.
Where causing harm is concieved as a positive action whereas allowing harm is concieved as negative action.
The vague distinction between positive and negative actions, is that positive actions are considered to be intrinsically `effortfull' whereas negative actions are not nessisarily so.\cite{Mossel2009} And to some extent it also links up with notions of positive and negative rights, where negative rights are to be free of positive adverse actions of others, and positive rights mandate positive beneficial actions of others.
%positive and negative actions have also connection in terms of the commision and/or ommission of actions - although the idea is vague.

In anycase, more people support equalities of opportunity that are more negative in flavour - that people should be able to persue choice of life agasint arbitrary injunctions (especially where they are conceived as positive actions of others), 
much more than are for equality of opportunity concieved more positively - that each should be positively supported to equally be able to persue choice of life -period.
In part simply because the latter tend to be stronger than the former.


The more agreed-apon position of, specific negative freedom of opportunity, is the philosophical take-away.
In sections \ref{} and \ref{} we consider particular negative freedom opportunities, particularly the goals of an electricity system that it should aim to not exclude participants arbitrarily (or without specific warranted justification), and also that it should specifically exclude negative external utilities.
This distinction between postive and negative rights will also be given greater treatment in chapter \ref{} as there it is taken to be a core defining difference between financial transactions of compensation and extortion.

In a similar fashion to how equality before the law is regarded as being perhaps nessisary but not sufficient to ensure wider social equality (necessary insofar as there are envied or finite positions or scarse resources which can only be allocated to finite individuals).
Even if we consider the most thoroughgoing equality of opportunity position - specifically Rawles `fair equality of opportunity' doctrine.\cite{sep-egalitarianism} - we might still have large social inequalities.
John Baker puts the point bluntly:

\begin{displayquote}
The biggest problem is that principles of equal opportunity help to make systems of inequality seem reasonable and acceptable. They shift the whole issue away from whether inequalities of wealth, power, status, and education are themselves justifiable, to the question of how to distribute these inequalities, The implication is that as long as the competition for advantage is fair, advantage itself is beyond criticism. The winners feel entitled to their winnings and the losers blame themselves.
\end{displayquote}

Or another example, is that strict equality of opportunity would morally restrict parents giving preference or advantage to their own children (as argued by Mason).

Perhaps a weak equality of opportunity is necessary but that `equality' is a concept that is wider or different than equality of opportunity.

Notwithstanding people find specific equality of opportunity doctrines is specific fields particularly uncontroversial - An example of which is democratic equality.
particularly that every member of a society is afforded equal right and capability to participate (and/or vote) and be informed about (some) governmental decisions (such as referenda and elections), and that this freedom should be irrespective of most exogenous conditions (gender, sex, height, race, etc).


\subsubsection{Equality of bigger things}

If people have some degree of equality before the law, and are subjected to the same (or otherwise similar) rules, and if they have some degree of equality among opportunities in life, what more could eqaulity demand?
And here we have a greater degree of bifurcation of incompatible ideas about which bigger targets of equality should be pursued.

The most obvious target of wider equality doctrine is with regards to economic purchasing power and money.
In many ways, the quality of a person's life is atleast associated with the income that they are capable of generating. and inequalities of other kinds can be caused by differences in economic positions.
Particularly that economic advantage can be self self-sustaining, and economic disadvantage can be sticky.
However, for many equalising economic positions directly seems like a bad target.
A classic argument (see Hume, section 3, part 2 \cite{hume2006enquiry}) is that an equality of people's freedoms to buy/own/sell/give would lead to economic inequality, and that perfect economic equality would lead to severe restrictions on economic freedoms and privacy.

One of the more concrete equality ideas is that wealth and/or income should distributed equally between peoples.
Money and purchasing power is one example of a freedom which is afforded to us in society, and per doctrines of equality of income/wealth, that positions are favored which increase the inequality of this freedom between peoples.
in the strictest sense total equality of income/money is seen to be undesirable for several reasons.
particularly that whatever policy enforces equality of income/money is seen to interfere on the individual level between peoples and their trading.
Indeed ownership of property and the freedom to work are even in the UN declaration of Human Rights. additionally, it is seen that total and completel equality of money and income would remove any drive to wealth bruilding and creative enterprises.
Additionally it is not even certain whether total economic freedom is something which is desirable anyway, as having equal money does not additionally compensate thoes with disabilitites or bad luck.
or nessisarily disuade or punish thoes that squander their income/money that would be afforded to them by socieity.
One of the aspsects of consumerism is that having equal money does not nessisarily mean equal ability (or nessisarily any ability at all) to engage in activities which are viewed to be 'truly important'.
Indeed the equal ability to purchase knick-knacks is hardly what people really view as a morally desirable system.
The more abstract refinement of equality of money/income shifts the focus onto the effect that money bestows, namely the freedom to purchase, and actualize a persons desired, and/or essential processes and freedom capabilities in life.


simple equality in choices or money is not what is cheifly meant. That actual attainment of sufficient resources/position/money nessisary to live a good life is the important thing, not just arbitrary choices of alternatives amoung knick-knacks.
This measure of `living a good life' or summarily `welfare' is to be equalised.
This bumps up against the question of how to measure it, and also if it is subjective.
Should welfare be identified with the satisfying of desires in life (good or bad or immoral or antisocial ones?)
Or is welfare to be associated with happiness or other emotions - such as subject to an experience machine.\cite{nozick2013anarchy}
(https://medium.com/moral-robots/utilitarianism-in-robot-ethics-507fef1a3d59)

or against welfare equality, to deny that peoples differing attainment of their own good-life is a thing that can be reasonably compared; and hence cannot then be made equal.
These considerations aren't just simply arm-chair distinctions either, for instance one objection that has been made to welfare equality, is to deny that peoples differing attainment of their own good-life is a thing that can be reasonably compared, as being lesser or greater; and hence cannot then reasonably be made equal.

A most commonly discussed (and quite general) contention is over the `levelling down' objection, which prompts the consideration of when an hypothetical equality society is better/worse than a hypothetical unequal society which is better off (see parfit). Another general contention is where a particular equality contrasts with a persons specific rights, such as people claiming they should be compensated more money for greater work (contrasting economic equality) (see Baker), or that they should be able to choose arbitrarily who they work-with/live-with or marry (contrasting equal opportunity) (see Mason).


The technique again, is to move away from (various and contestable) numerical interpretations of equality into the more abstract conception, and thence see what implications follow.

\subsubsection{Moving away from specific Equalities}

%A most commonly discussed (and quite general) contention is over the `levelling down' objection, which prompts the consideration of when an hypothetical equality society is better/worse than a hypothetical unequal society which is better off (see parfit).

Expecting a person to give a precisely defined answer to the question ``Equality of what?'' may be asking too much; as even the phrases which people use in everyday life are seldom given exact specifications\footnote{Degrees of vagueness are well witnessed in everyday sentences, eg. ``There is a pumpkin by that tree", such as argued in the classic Sorties paradox\cite{frances_2018}}, let alone concepts pertaining to the spectra of possible societies.
So, for instance, the space of various contemporary political philosophies which faithfully attempt to construct and interpret some reasonable form of equality between persons has been described as belonging to an `egalitarian plateau'.\footnote{The phrase is originally attributed to Dworkin and subsequently adopted by others.}\cite{Brown2007}
Or conversly, while a specific ethical equality may not be agreed apon, perhaps there may be a more broadly accepted notion of what an `inequality' looks like,
particularly as it is sometimes blurred with the concept of a `social injustice'.\footnote{There is a debate as to when/where/how an inequality \textit{simpliciter} also becomes an injustice. It is possible to believe that an inequality constitutes an injustice directly, or perhaps that an inequality is proof (or perhaps only potential evidence) of a injustice in procedure or treatment. see parfit's concept of Telic vs. Deontic Egalitarianism.}
Though some vagueness may be inevitable or even necessary, unnecessary vagueness is probably unhelpful.

In a political context the word ``Equality'' has occasionally been used as a political slogan, with people attaching various feelings toward the word.
For some it is a slippery term and an political umbrella device\footnote{see Janet Richard's description of progressively embracing disparate `equality of opportunity' doctrines, as sliding down a snake in a snakes-and-ladders game. Or Kymlicka's \cite{kymlicka2002contemporary} p152 consideration that the same `slippery-slope' argument surrounding where to draw the line between choice and circumstance tends to lead people to Libertarianism}.
While for others the term encodes the positive hope of a society, free of abusive power relations that perpetuate social injustices.\footnote{Perhaps as most directly evident in various feminist literature which give analyses of (eg. see \cite{Cudd2006-CUDAO}) the relationship of men\&women in relation to oppressive dominance/submission: eg. MacKinnon writes ``difference is the velvet glove on the iron fist of domination. The problem is not that differences are not valued; the problem is that they are defined by power''\cite{mackinnon1989toward}. Though not all feminists view power so negatively, as some view it as uneven distribution of positive (or potentially neutral) empowerment \cite{doi:10.1111/j.1527-2001.1998.tb01350.x}. Furthermore not all egalitarians about power view it specifically in relation to gender, see Richard Norman's egalitarian position - that disparities of power drive coercion\&exploitation and prohibit a society from being `genuinely cooperative'}

Some alleged motivations for the adoption of various political positions on equality are that they extend from humanitarian concerns about the suffering of the poor (in some sense defined), and also that motivation can allegedly extend from anger about the gratuitously rich\footnote{for instance, George Orwell's, The road to wigan pier, chapter 11: ``Though seldom giving much evidence of affection for the exploited, he [a socialist] is perfectly capable of displaying hatred - a sort of queer, theoretical, in vacua hatred - against the exploiters. ... It is strange how easily almost any Socialist writer can lash himself into frenzies of rage against the class to which, by birth or by adoption, he himself invariably belongs.'' }.
And while these allegations may be true or false, neither the provision of resources for the poor (such as per meeting a level of sufficiency, or as a matter of priority) or curtailment of the wealth of the rich (sometimes called `quasi-egalitarianism') entail any direct equality of any measure \textit{per sei}.

So there are principally two approaches to be taken, the first is to attempt to to take on a more descriptive process by characterizing the things which people tend to desire by the term `equality' and also despise by `inequality'.
The second approach is to try and further refine a more central notion - what is it specifically about equality that we all find motivating?
We will both approaches and relate them to electricity systems, but the second is more direct and we will take it first.

\subsubsection{A more central and specific Equality}

\begin{displayquote}
\begin{tabular}{ll}
\multicolumn{2}{l}{[reaches for the coffee pot with the regular coffee,} \\
\multicolumn{2}{l}{\-\hspace{5mm}and starts pouring it into the pot with decaffeinated coffee]} \\
Derek Philby:  & Excuse me, what are you doing?\\
Monk:  & Oh - um... just making them even.\\
Derek Philby:  & But you're mixing the regular with the decaf!\\
Monk:  & But they're even.\\
Derek Philby:  & But they're mixed together!\\
Monk:  & But they're - they're even.\\
Derek Philby:  & But they're mixed together.\\
Monk:  & But they're even...\\
\end{tabular}\\
\vspace{-0.5mm}\\
\null\hfill\textit{Monk, Season 2, Mr. Monk Goes Back to School}
\end{displayquote}

There are several ways in which direct equality of measure has been argued-for specifically.
For instance, as a practical measure in some settings, and also as an extension of more fundamental moral principles.

In some cases, what amounts to an equal allocation can also coincidentally satisfy other values, and thus a person can advocate for equality in a given context \textit{for itself} and/or \textit{instrumentally}.
For instance, David Miller \cite{equalityandjustice:1998} argues for the practicality of giving equal remuneration between hypothetical employees in the context total uncertainty about how much each of them deserved (and/or lack of means distinguishing them). He also gives broader examples of reasons for which we can argue for equality in various contexts: for asthetic and pro-social reasons, because it can be a sufficiently practically simple social contract, or because it might be politically inevitable. etc.
And these may count (among many other things) as instrumental or prudent reasons to value the implementation of an equality in a specific context. However in other contexts, these same reasons could potentially point to other arrangements.

However, reasons to value an equality \textit{in-and-of-itself} can and have been argued from more abstract moral considerations.
Particularly there is a distinction to be made between equal treatment, and a more general `moral equality'.
Consider the quote from Christopher Nathan:

\begin{displayquote}
The distinction between ``equal treatment'' and ``treatment as equals'' expresses this difference between offering people the same treatment, and acting in accordance with the fact that they are moral equals. Equal status does not constrain us to a set of identical actions regardless of our differences.\cite{whatisbasicequalitynathan}
\end{displayquote}

The question then becomes how interpret moral equality.

Perhaps the better thing is to ask what moral inequlity would look like between peoples.
Are people morally unequal, are some \textit{worth more} than others?

\begin{displayquote}
The essential thing, however, in a good and healthy aristocracy is that it should not regard itself as a function either of the kingship or the commonwealth, but as the SIGNIFICANCE and highest justification thereof--that it should therefore accept with a good conscience the sacrifice of a legion of individuals, who, FOR ITS SAKE, must be suppressed and reduced to imperfect men, to slaves and instruments. Its fundamental belief must be precisely that society is NOT allowed to exist for its own sake, but only as a foundation and scaffolding, by means of which a select class of beings may be able to elevate themselves to their higher duties, and in general to a higher EXISTENCE.
\cite{NietzscheGutenberg} Neitzsche, Beyond Good and Evil, Ch 9
\end{displayquote}


\begin{displayquote}
I want to emphasise what is, on my view, the most important object of egalitarian distribution, and that is \textit{power}. Of course power is not something which can be parcelled up and shared out like a commodity but we can properly talk of `the distribution of power' and this is, more than anything, the determinant of whether a community is authentically cooperative.\cite{TheSocialBasisofEquality:1998}
\end{displayquote}




\section{Some alternatives to equality}

In these particular contexts (and even across them), particular arguments can be made for and against the ideal of equal allocations specifically.
Well known and general modes of objection include contrasting equality with efficiency (per the `leveling-down objection'), and contrasting equality with particular freedoms (such as the freedom to buy/sell/give property, or prioritize the treatment of one's own family).
Conversely there are general arguments that can be made for equality itself as opposed to other modes of distribution, such as arguing for equality in light of `equal moral worth'.
Some other modes of distribution can be contrast-with or overlapping with utilitarianism, prioritarianism, sufficientarianism.

For instance, utilitarianism is commonly articulated as a moral philosophy for the maximising of the sum of social welfare (or utility) in a given circumstance (or across them), and it might coincide (or not) that equal allocation of a particular thing is also maximising of its sum.
Prioritarianism is the idea that the less-well-off should have priority in allocation (perhaps even independent of their utility gains), such as might be encoded in Rawl's difference principle: the idea that any particular inequality is permitted insofar as the worse-off member of society is better-off than he/she would be otherwise.
sufficientarianism is the position such that each person should have some `sufficient' level of the allocated quality - however decided.

In society it has been recognised that some degree of inequality is inevitable, and perhaps evern preferable, one such articulation is Rawles `Difference Principle' - the idea that particular inequality is permitted insofar as the wors-off member of society is better-off than he/she would be otherwise.
perfect equality is not desirable if everyone is therefore equally miserable.

Indeed these approaches are hardly exhaustive, and the result of the considerations of which one is appropriate may be highly context dependant. (people should have food of sufficient nutrition, public housing should be allocated as priority to thoes in most need of it. government subsidies should maximise the total welfare of the nation etc.) 
And even in these contexts, a person might value these allocations as morally worthwhile \textit{in themself} and/or perhaps \textit{instrumentally} as a means to other ends (such as promoting community, avoiding envy/favoritism etc).

While it may be usefull to sample the innumerable fields to which equality principles could be applied it makes a better approach to attempt to characterise the fields where people have a strong egalitarian predisposition against other specific feelings, and then to analyse why.



One potentially helpful way to consider the field of possible applications and to analyse the contexts of when/where/how equality principles can/should be applied is to re-frame claims to about equalities (against other principles) as if they were personal statements of moral rights for an arbitrary individual.

For instance, notions of equality particularly regarding equality of opportunity are often strongly conceptually associated with notions of \textit{impartiality} in a selection process.
Framing an impartiality principle as a statement of rights might go along the lines of: "I have a right not to be arbitrarily excluded from X".
Which may be felt to have significant weight depending on what the X is; for instance, if the X is voting in government referenda then we would tend to agree, but if X is 'being married to Mary Sue' we would tend to disagree - as it would run afoul of her rights claim of "I have a right to choose who I wont spend my life with".
A more complicated example is a quasi-egalitarian principle offered by Mason: "the effects of peoples difference in circumstances should never be such that some can acquire the resources that are necessary to lead a decent life whilst choosing not to work to earn an income when others are not able to do so.". Which could potentially be re-framed: "I have a right to be as undeservedly fortunate to be able to live decently without working as anybody else".
Note that while I think it can be usefull to make comparisons like this, it can be difficult to frame and compare some other moral ideas in this way, eg. utilitarianism is more difficult to frame as a first-person rights claim.\footnote{perhaps: "I have a right to count as much as others in decisions which divide utility."}

In anycase, this re-framing can make a specific equality claim a bit more personal and allows some direct comparisons between and against other rights claims. It also frames these moral positions in a manner that is potentially more likely to correspond with what people might feel and/or morally think.

It also encodes equality doctrines (and others) generally as social principles or processes for universal application - formal equality.



%All these kinds of rights claims have an equality of a very basic sense, as they are all with reference to any arbitrary indiviual `I' (ie. the indexical), however this usually not sufficient to embody anything notably `egalitarian'.

%From this perspective, the fields which people have strong egalitarian predispositions are fields where people may have strong affinity towards rights claims to quantities which are conceived of as having a relative character; such as might be equalized.
%It therefore follows that things which are naturally conceived of as important relative quantities are potential candidates for strong egalitarian predispositions.

%For instance, "I have a right to be paid as much as anyone else" and "I have a right to be paid the same hourly rate as everyone else" are both notably egalitarian in character (even if they are potentially conflicting) as they deal with money in relative terms among potential claimants, whereas "I have a right to be paid at-or-above \$5 per hour" is distinctly less egalitarian as it deals with money in absolute terms.
%For instance, `having a job' is not a relational property or itself a quantity, the numerical probability of `having a job' is a quantity but not necessarily a relational one, and the numerical probability of `having a job' above another's probability, is a relational quantity.





\subsection{Some broader notions on what inequality means}



In this section I attempt to outline some particular notions about what people have tended to associate with the word ``inequality''.

The particular notions that I explore is that:
\begin{itemize}
    \item not \textit{partial} to specific persons and groups - atleast not directly.
    \item impose differential treatment only in-light-of morally defensible considerations.
    \item considering and/or minimising the difference in treatment and processes that is imposed upon people/s \textit{in practice}.
    \item That people should be generally be compensated for their contributions to society and be free from extortion.
    \item That people should be free from the negative influences of others particularly when thoes influences are intentional, positive and/or deliberate.
    %\item That interactions should be crafted to leave a person worse-off while others are better-off - that interactions should not be regretfull.
    \item That undeserved reward should be limited - the `rich get richer while the poor get poorer' is bad.
\end{itemize}

\subsubsection{that nobody should be excluded and/or suffer negative externalities}

particularly some electricity systems dont scale, and exclude individual or small suppliers/generators/consumers for equal consideration.
Alternatively that people should not be able to force an individual out, but potentially that they should be able to help each other out.
that the system should be responsive to positive external preferences, but not negative external preferences.
This has been argued in the context of an ethical utilitarianism (see \cite{kymlicka2002contemporary})

\subsubsection{Compensation and Extortion}

One primary way which people concieve of inequality is in the context of extorition. Indeed many marxists viewed the inequality of power between the classes that was so objectionable interms of being maintatined by extortive measures.
but what is an extortion?

Consider the following game: Two people are brought to a table, 1 \& 2, player 1 has a binary choice to make A or B, and depending on which choice is made, an amount of money is dolled out to both players, however before player 1 makes a choice, player 2 has the option to enter into a binding promise to transfer some of the money he will be given to player 1 depending on his choice. In this game we consider that the players are describably ethical.

consider:

\begin{table}[h!]
\begin{tabular}{lll}
               & A   & B   \\
Rewards Given: & 0,10 & 1,0
\end{tabular}
\end{table}

How much money should an ethical player 2 transfer to player 1 for choosing action A?
Obviously he would need to transfer \$1 atleast to make it worthwhile for player 1 to bother choosing action A, but what above that?

If players 1 and 2 ended up with a financial setting of \$5+0.5 \$5-0.5 that would be concieved of being pretty fair. as a splitting of the difference above a compensation for player 1.

However the thing is that it really depends on what actions A and B mean.
So for instance, if A is `save player 2 from a burning building' and B is `dont save player 2', then we would no doubt say that player 1 is deserved some compensation for the good deed done to player 2.

But if A is `do nothing' and B is `steal money from player 2 and buy lollies', then we would think that A should be credited nothing for refraining from action B.

In the first case a `split-the difference' principle is viewed as providing a fair compensation, but in the second the principle is viewed as being extortion.
I submit the thesis that this difference between the two is in-part captured by viewing the actions as positive or negative. where in the second case player 2 is being compensated for not engaging in a positive action, but in the first case is being compensated for not engaging in a negative action.

what defines a positive action over and above a negative action?
primarily it is somewhat vague in practice, but positive actions are generally characterised as being: effortfull, deliberate and/or against the normal and default state of affairs. whereas negative actions are not so.
However from a strictly mathematical perspective - where this kind of lucid distinction is absent - both compensation and extortion are actually similar.

In an electricity system, the question of what should be considered (or would likely be viewed) as default/normal state of affairs may be a bit difficult to characterise.
But more generally, the supplying of electricity (as opposed to the non-supply of electricity) may be viewed as an `effortfull' postive action - even if it is done automatically by a machine that is totally indifferent.

A practical example of this is that compensating a company for not positively oversupplying electricity to the grid is an example of extortion, whereas companies which can drop their power generation quickly to limit oversupply on the network are credited with compensation. (and this can be seen as a double standard, and if not- then what precisely should distinguish these cases? is one action above the other to be considered to be positive?)

This general non-exploitation clause, can be viewed as being in-some ways similar to `individual rationality' in mechanism design. but is extremely context sensitive.
It can be directly mathematically specified (as we will get to in bargaining mathematizations) or perhaps more implicit.

\subsubsection{That people should be rewarded for their contribution}

Particularly as best and directly encoded by VCG imputations.

\subsubsection{That they should not be imputed less than what they could otherwise get for themselves}

Ie that imputations should belong to the Core of the cooperative game.
The shapley value is compromise between this and marginal contribution payments.
best embodied in some of the work of Marxist philosophers such as John Roemer. (citations)

\begin{displayquote}
This is the main aim of John Roemer's work on exploitation. He defines Marxist exploitation, not in terms of surplus transfer, but in terms of unequal access to the means of production. Whether one is exploited or not, on his view, depends on whether one would be better off in a hypothetical situation of distributive equality -- namely, where one withdraw with one's labour and per capita share of external resources. If we view the different groups in the economy as players in a game whose rules are defined by existing property-relations then a group is exploited if its members would do better if they stopped playing the game, and withdrew their per capita share of external resources and started playing their own game.\cite{kymlicka2002contemporary}
\end{displayquote}



\subsubsection{that the `rich should not get richer' because they are rich}

That the rich should get richer because they are rich, violates peoples feelings regarding deservingness.
Best encoded by Marx' objection to CMC and MCM cycles of capitalistic wealth generation and exploitation.
One primary potential way to issue incentives is to make them non-monetary, or non-monetarisable. as Baker considers social incentives such as trophies and gratitude shows of reputation. while this is unrealistic perhaps there is other compensations that can be given.

Primarily can be seen by hindering the rewarding participants in terms of money.
perhaps in terms of future promise contracts for reciprocal electricity supply.
the schema should be made to facilitate these contracts.
Notwithstanding it is within some people's understanding of equality that dominance in one `sphere' should not sill over into other's.
It might not be easy to see how to prevent this from being the case - even in the event of micky mouse currencies.
but it is hardly foreseeable that any realistic electricity system should not feature monetary trade.

Alternatively and/or optionally, all that would need to be done is for regulatory body to imbue trading electricity with a lower expected rate of return (ROE) than other potential asset classes (see XYZ's economic theory) to disuade powerfull investors from the electricity system.
but would this be what people actually want? as it certainly wouldnt solve this feature of capitalism for society, only shifting it elsewhere; and it would inhibit the actualizing of economies of scale for consumers.


\subsubsection{Free from envy}

There is much discussion and topicality about envy free-distributions.
particularly that nobody would prefer what other people are allocated.
While envy freeness, is a difficult criterion to satisfy in general, there is some attempts to systematize envy-freeness into distribution systems.
Particularly the work of Dworkin. (cite,cite,cite)




\section{VCG}

One of the elements of an electricity system is the bidding, and there is no loss for different strategies under these systems.
and in the context of stragegic bidding the system will fall into a less than optimal situation.
and it may beg the question as to what state the system under manipulation will fall into, and if potentially we could design the best possible lowest-common denominator system from the start.

one avenue of investigation that takes this approach to analysing the worst possible case under manipulation is called `Mechanism Design'.
which famously won some people the nobel prize.

particularly mechanim design broadly analyses the situation between multiple parties where each party has different preferences over the possible outcomes, and reported preferences over the outcomes.
mechanism design then poses the question what systems can be put inplace to ensure the best (considered broadly and/or variously) outcome in the context of stragegic reporting.

one of the primary contexts into which Mechanism Design has been historically considered is in the context of voting. -- is it possible to strategically vote, and by misrepresenting your preferences gain a better outcome for yourself?

And is it possible to design an appealing system where the answer to this question is invariably `no'.

There are many famous theorems (many of them strongly Negative - arrow and satherswaite) that apply broadly to these kinds of questions.

however more particularly than the voting context, Mechanism design also oftend covers situations where an outcome is implemented in the context of transferrable utility (ie. money), and where outcomes are reported by their utility - than by transitive preference.
In this situation the potential outcomes discussed involve not only the utility of the outcome to each player but usually in addition to the utility transferred/taken from him/her.

$$insert equation$$

this condition is often referred to as Linearity (CHECK).

Among all the possible mechanisms that allocate outcomes and transferrs between parties, a historic mechanism called VCG is most famous.


\begin{itemize}
    \item let $I$ be the set of agents participating
    \item let $n=|I|$ be the number of agents
    \item let $O$ be the set of possible outcomes
    \item let $v_i(o)$ denote the \textit{true} value to $i$ of outcome $o\in O$
    \item let $\hat{v}_i(o)$ denote the \textit{reported} value to $i$ of outcome $o\in O$
    \item let $o^*$ denote the socially optimal outcome, according to reported evaluations, ie $$o^*=\argmax_{o\in O}\sum_{i\in I}\hat{v}_i(o)$$
    \item let $o^*_{-i}$ denote the outcome that maximises payoff amoung agents other than $i$ according to reports, ie, $$o^*_{-i}=\argmax_{o\in O}\sum_{i\in I\setminus\{i\}}\hat{v}_i(o)$$
\end{itemize}

A groves mechanism chooses an outcome $o^*$ and defines a transfer $T_i$ to each agent $$T_i=\sum_{j\in I\setminus\{i\}}\hat{v}_j(o^*) - C_{-i}$$
where $C_{-i}$ is any value that is independent of $i$'s reported preferences.

VCG is a groves mechanism where $C_{-i}=\sum_{j\in I\setminus\{i\}}\hat{v}_j(o^*_{-i})$.

Fundamentally the VCG mechanism allocates the socially optimal outcome (ie. \textit{pareto-optimal}), and payments between inviduals in a way that is \textit{incentive compatable}, \textit{indivually rational}, but is not \textit{budget-balanced}.
And is particularly proven to be optimal in the context of unrestricted player preferences.

And this system might be considered very fine method for allocating a physical outcome on an electricity network and allocation of payments between participants; and this topic has had some exploration \cite{SESSA2017189, 8264596} (MORE CITES POSSIBLE).

One of the fundamental problems of applying VCG to electricity networks is the budget-surplus that often results and where this money should go, particularly that if the money should go back into the participants in the network then it would destroy the incentive compatability that was essential to the scheme in the first place.

Unfortunatly it is also proven that incentive compatability, unrestricted player preferences, individual rationality, and pareto-optimality are an impossible combination.

There are several methods which have been developed to handle this issue
\begin{enumerate}
    \item adding extra elements into consideration, such as baysean priors on preferences, and reducing incentive compatability to baysean incentive compatability (see section on bargaining)
    \item By throwing away consideration on the unrestrictedness of the player preferences, and by moving to other groves mechanisms which allow some redistribution of the surplus
    \item by sacrificing pareto-optimality in the choice of the social outcome while keeping budget balance and incentive compatability
\end{enumerate}

particularly the first is probably most interesting for our considerations, and is the source of the d'AVGA mechanism which allocates redistribution of the surplus bassed on baysean priors on player preferences.
The addition of baysean priors also has links to bargaining, particularly with the work of Meyerson, and his bargaining games.
Which in turn have been extened (by himself) to coalitional games, and also through some of his work to Harsanyi games. (CITE)
Particularly through the introduction of baysean considerations other schemes can redily be made to be incentive compatable.

However there are certain issues with supposing accurately known baysean priors (particularly if they are learnt from past behavior).

The second avenue involves the use of knowledge (or the imposition of restraints) on the possible preferences of players.
This avenue often goes under the description of VCG redistribution, and there has been significant discussion about this particular avenue.
Originally the first to propose this scheme was Ruggiero Cavallo \cite{Cavallo:2006:ODM:1160633.1160790}, particularly in the context of allocating a single physical object exclusively to one party.
this game `All-Or-Nothing' games (AON) --- where any party's not recieving the object has known utility of zero --- has a better-than-VCG allocation mechanism.
Cavallo, discussed this allocation mechanism and then consideres the associations and implications for the potential design of other allocation mechanisms where information (or constraints) are known about the players true preferences.

The design of redistribution systems has since seen some more development, from redistribution systems in the context of other situations and games.
But, it is generally known that designing optimal (non-linear) redistribution systems is a hard task, and even neural networks have been employed to construct such systems \cite{DBLP:conf/atal/ManishaJG18}.

However it is known that these systems will (by construction) almost always produce some budget surplus.

A tertiary avenue to dealing with the problem is to throw away the pareto-optimality of outcome selection.
Probably one of the first people to think of this process way Boi Faltings \cite{10.1007/978-3-642-25510-6_14}
Who proceeds about the process of designing a non-pareto optimal VCG mechanism by splitting the population into two groups, where the VCG outcome from the first group is selected irrespective of the preferencs in the second group.
and where the second group recieves the budget surplus from the VCG mechanism applied to the first group.
This rather genius mechanism is the subject of a patent in the US (CITE), and constitutes a budget balanced VCG-type mechanism.
which can be made a little bit more baysean regular by randomy selecting the `sink', and splitting the budget surplus between the parties evenly without knowledge of who will be selected.\cite{10.1007/978-3-642-25510-6_14}

particularly this method selecting a non-pareto optimal outcome has been the subject of investigation, and the process is proven to nessisarily need a sink.
and that there are different ways to select an appropriate sink randomly \cite{NATH2019673}.

What is particularly interesting is that various VCG type mechanisms can potentially be scaled up, leading to `effectively' the impossible combination \cite{NATH2019673}.

There are several objections (and unanswered questions) to VCG as applied to electricity networks and more generally, and reasons why it has been seldom implemented in practice.\cite{journals/ior/Rothkopf07}\cite{Ausubel2006}
Though it has been proposed (and presumably implemented) in the context of cooperative multi-agent reinforcing (CITE-NASA)

Particularly it is not very obvious that VCG type payments are nessisarily economically reasonable payments between competitive parties, but much more payments between parties with incentives that are made to cooperate.

And although incentive compatability is a noble objective in mechanism design, it is still subject to strategic manipulation, particularly in the context of the forming of coalitions.

Coalition strategy proofness, has been proven to be quite generally a difficult concept to design, and in some cases has categorically proven to be impossible \cite{10.2307/2297048} without further elements, such as private information transfer and uncertainty.
Although there has been some effort in designing and analysing the occurance of coalition proof mechanisms, these remain difficult.

One of the considerations of using VCG is how it is to be constructed as a mechanism in the context of an electricity system.
and different people have published different ways to instigate such a system.
fundamentally, it involves the question of what the network should be considered to be in the absense of a particular player.
The solving of optimal utility is a classical OPF problem, and is it to be the case that the system would have to be solved to OPF for each of $n$ players?
This is a disadvantage of LMP, but not nearly as much as a disadvantage over Shapley Mechanisms.

If we discount the benefit of incentive compatability, then the fundamental question remains: how much \textit{should} (above other considerations) electrical participants be credited/debited according to their immediate influence apon others relative to their hypothetical absense.

In the following sections we will compare the allocations of power/money in randomly generated networks under VCG against other schemes.


\section{Bargaining and Game Theory}

One component of analysis that is completely missing from VCG is any specific consideration of `who can do what', and instead there is only recorded preferences over all-possible outcomes.

One of the most historic classes which involves an allocation which specifies distinct actions for individuals is bargaining solutions.
particularly Nash bargaining and the disagreement outcome.

of solution concepts is `bargaining' - which serves as the perspective from which our investigation extends.\\

A bargaining solution concept applies in a situation where there are a number of parties (or agents) and a space of possible agreements which those agents can reach (and value differently) between themselves.
A bargaining solution concept identifies an outcome which the agents `will' (or `should') agree upon.
Perhaps the most famous bargaining solution concept is called the Nash-bargaining solution concept.

Nash bargaining was introduced \cite{nash1} as an axiomatic approach to predict the result of individuals who are bargaining over potential outcomes.
It is defined over a convex compact set of potential outcomes $F$ (which might coincide with power-flows and fiscal payments on an electricity network) 
where each of the players $P=\{p_1,p_2,\dots\}$ value the outcomes differently with utilities $u_{p\in P}(f)$ for $f\in F$.
Additionally there is a privileged outcome called the `disagreement' outcome $d$ which represents the event of the negotiation between the players breaking down.

Nash identified that in this case there is a unique solution satisfying some very intuitive axioms:
\begin{itemize}
\item \textit{Invariant to affine transformations}: that the solution should not change if the utilities of either players are scaled (by some positive factor) or offset, ie that they are invariant under the set of affine transformations that might also represent their relative preferences.
\item \textit{Pareto optimality}: That the solution will not be inferior to any other point with respect to the preferences of all players.
\item \textit{Independence of irrelevant alternatives}: If any subset of potential outcomes does not feature the solution point then it could be removed without affecting the solution.
\item \textit{Symmetry}: The solution is invariant with regards to the ordering of the players.
\end{itemize}
This solution maximizes the product of utilities above the utility of the disagreement point:\cite{book1}
\begin{equation}\label{nash-product}\text{nash}(F,d) = \argmax_{(f\ge d)\in F}\prod_{p\in P}(u_p(f)-u_p(d))\end{equation}
In this way the Nash-bargaining-solution can be seen as a simple solution concept, whereby the participants report their valuations over potential outcomes, 
and a Pareto-optimal outcome is determined which maximizes the product of those utilities above the disagreement outcome.

In many cases of physical bargaining (which might involve alternating offers etc.) the `disagreement outcome' is often naturally dictated by the context of the bargaining process - 
such as the event of quitting in the context of a wage-negotiation or of walking-away from a potential sale
- It can be seen as the point of `threat' from which the bargaining process occurs.\cite{nash2}

One objection to the Nash bargaining solution is that it may not be considered natural or a reliable description of real-world bargaining.
Although there do exist some evolutionary models suggesting that in some social dynamics Nash bargaining might emerge in a social situation \cite{articlechoakihiko}, there exists discussion and experimental work finding that real behavior between humans exhibits some ambiguity \cite{KROLL2014261}, even in the case when the disagreement point is naturally given by the setting.

However in other cases (such as in an electricity network) a singular disagreement outcome is not very clearly given by the context.
And so if a bargaining-type solution concept is utilized then a disagreement outcome must be chosen `endogenously' from the set of possible outcomes $F$.
How to do this?

In John Nash's paper ``\textit{Two-person cooperative games}"\cite{nash2}, He explicitly addresses the consideration of the agents choosing a disagreement point between themselves in the bargaining process. However it is important to note that there are many possible bargaining solutions\cite{smorodinsky,tempered} which relate to different axioms,
 and also different methods of choosing a disagreement outcome.\cite{bozbay,alter_bargaining1,tale1}

Notwithstanding, Nash's solution concept has a very simple form in the case of two-player games where utility is transferable (TU), equivalent to the sometimes called the `coco-value' \cite{kalai1,Kalai2010}.
Which in-turn also has a well-studied extension to many players (which we are primarily interested in) which has been called `the Value'.\cite{values1,values2,values3}


\section{Cooperative Game Theory}




\section{THE PLAN PLAN}


\subsection{The problem of Energy Allocation}
The purpose of the introductory chapter is to outline the nature of power\&money allocation mechanisms in electricity systems that are orientated toward supplying consumer demand from industrial generators, and that thoes mechanisms are ill equipped to handle the emerging situation where consumers are increasingly supplying their own and their neighbour's electrical energy.

\subsubsection{The NEM}
outlines the function and operation of the National Energy Market and the structure of the current system, and how consumers that generate power ('prosumers') are treated under the current system.

\subsubsection{Decentralisation and Auction Mechanisms}
gives some recent history and back-drop to the NEM structure, providing differences in market structure between Australian Eastern and Western Australian energy systems, and in the context of many other countries process of setting up decentralised market systems, a process that occured throughout the 80's and 90's.

\subsubsection{Consumer Suppliers}
Outlines the complications and proliferation of ideas surrounding the incorporation of prosumers and network-aware consumers into existing structures and energy markets, such as demand-response programs, virtual power-plants, and feed-in tarifs.

\subsubsection{Extended Bidding Schemes}
blah

\subsection{The Question of allocation}
The chapter gives a conceptual outline of the approaches to the fundamental question of how resources (money and/or electrical power) can-or-should be allocated in *any* system.

\subsubsection{The Philosophical 'should'}
A preface addressing the essential philosophical nature of the question, and associated ideals that people may or may not hold in regarding the distribution of resources in any given system or enterprise.

\subsubsection{Bargaining}
Switches the question into one of what ideally 'would' (rather than 'should') occur in the context of resource distribution, particularly between small numbers of participants, outlines idealised and mathematical ideas about Game Theory and bargaining, particularly of Nash and Harsanyi solution concepts.

\subsubsection{Cooperative Schemes}
Outlines ideas of cooperative game theory surrounding the
distribution of resources between many parties, including such concepts as the Shapley Value, Core and Neucleolus; also touches on NTU schemes.

\subsubsection{Considerations of Mechanism Design}
Gives an overview of key ideas and results in the complicated field of Mechanism Design, particularly outlining ideas and considerations of Incentive Compatability (IC) and individual rationality (IR), highlights the principal VC-type mechanisms as a special and potentially unrealistic mechanism, but also highlights how weaker concepts (such as Baysean incentive compatibility) can be applied to bargaining ideas (such as the work of Roger Myerson), implementation theory

\subsubsection{Economic Perspectives}
Gives some history to the ideas surrounding value in an economic context, and particularly outlined is the history of the (Marxist) labor theory of value and neo-liberal marginalism principles. gives some outline of how these ideas square up with Auction Design and how they are proposed to extend to distribution energy markets as Locational Marginal Pricing principles (LMP)

\subsubsection{Alternative approaches}
Gives an overview of other left-over and alternative ideas surrounding bargaining solutions and resource allocation (including egalitarian solutions, and alternative descriptions of bargaining processes)

\subsection{A new solution}
The GNK value - gives the description of the primary solution concept that is developed throughout the PhD, the 'Generalised Neyman \& Kohlberg value'

\subsubsection{Summary Statement \& Derivation}
Gives in a most direct fashion the solution concept as simple and direct as possible, appealing-to and extending-from multiple of the ideas given in the previous chapter.

\subsubsection{Discussion of the New Solution}
Gives some further elaboration to the nature and paradigm of the solution. and how relates to and also differs from other concepts

\subsubsection{Computability}
Addresses the Issues associated with the computability of the GNK value, outlining the extreme difficulty and also proposing an avenue of successful solution, through sampling techniques and a simplification in structure; discusses the nature of the simplification.

\subsubsection{Merits \& Deficits}
Compares the pros\&cons of the GNK value primarily over LMP particularly as computed across a variety of randomly generated synthetic various large scale electricity networks. points of comparison are made assessing the GNK value as appropriate for application for electricity networks.

\subsubsection{Alterations}
Gives an outline of slight alternative formulations of the GNK value that may be more palatable depending on philosophical perspective.

\subsection{Sampling Processes}
discusses the methods that were created and evaluated to sample and solve the GNK value for large-scale networks

\subsubsection{The complexity problem of the Shapley Value}
Gives an overview of the processes that have been used to solve the shapley value approximately

\subsubsection{Probability Bounds}
Gives a background on the development of probability bounds associated with sampling - such as Hoeffding's inequality - and how they relate to processes in probability theory, such as Chernoff, Entropic methods and probability unions.

\subsubsection{A new probability bound for Stratified Sampling (SEBB)}
Gives the full development of a novel probability bound that is applicable for stratified random sampling

\subsubsection{A new method of Stratified Sampling (SEBM)}
Gives the outline of the method of sampling to minimise this new probability bound, and the success it brings in the context of sampling synthetic data.

\subsubsection{The Importance of the method}
Outlines the material relevance of this new method to wider studies, and illustrates this by addresses the potential for its use in calculating the GNK value.

\subsection{Conclusions}
Gives an overview of the arguments surrounding allocation of resources potentially relevant to electricity systems (Money \& Power), summarizes the development of novel techniques throughout the PhD, gives a brief description of the value of the necessity and value of the contribution; and makes concluding statements.








\bibliographystyle{plain}
\bibliography{references}
\end{document}

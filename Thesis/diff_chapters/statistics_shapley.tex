
\section{Shapley Value approximation}\label{sec:shapley}

The Shapley Value is a cornerstone measure in cooperative game theory (introduced in section \ref{subsec:the_shapley_value}). 
It is an axiomatic approach to allocating a divisible reward or cost between participants where there is a clearly defined notion of how much surplus or profit a coalition of participants could achieve by themselves.
It has many applications, 
including analysing the power of voting blocks in weighted voting games \citep{Bachrach2009ApproximatingPI}, 
in cost and surplus division problems  \citep{AzizEtal2016,archie_paper1}, 
and as a measure of network centrality \citep{Michalak:2013}.
But primarily, is useful to us as a method of allocating financial payments in electricity network contexts (see section \ref{the_value_def}).

%Formally, a \textit{cooperative game}, $\langle N,v\rangle\in\mathbb{G}_N$, comprises a set of $n$ players, $N=\{1,2,\dots,n\}$, and a \textit{characteristic function}, $v:S\subset N\rightarrow \mathbb{R}$, which is a function specifying the reward which can be achieved if a subset of the players $S\subset N$ cooperate, where $v(\emptyset)=0$.
%In this context the Shapley value $\varphi$ is a unique mapping from cooperative games to the player rewards $\mathbb{G}_N\rightarrow\mathbb{R}^n$ which satisfies axioms:

%\begin{itemize}
%\item	
%\textbf{Efficiency}: That the total reward is divided up: $\sum_i\varphi_i(\langle N,v\rangle) = v(N)$
%\item	
%\textbf{Symmetry}: If two players $i$ and $j$ are totally equivalent `substitutes' then the receive the same reward: ie. if $v(S\cup i)=v(S\cup j)~~\forall S\subseteq N\setminus\{i,j\}$, then $\varphi_i(\langle N,v\rangle) = \varphi_i(\langle N,v\rangle)$
%\item	
%\textbf{Null Player}: If the addition of a player $i$ to any coalition brings nothing, and is a `null player', then it receives reward of zero: i.e if $v(S\cup i)=v(S)~~\forall S\subseteq N$ then $\varphi_i(\langle N,v\rangle)=0$
%\item	
%\textbf{Additivity}: That for any two games the reward afforded each player is each is the sum of the games considered together: i.e. for any $v_1$ and $v_2$, that: $\varphi(\langle N,v_1+v_2\rangle)=\varphi(\langle N,v_1 \rangle) + \varphi(\langle N,v_2\rangle)$
%\end{itemize}

%Specifically, the Shapley value is expressed as:
%\begin{equation}\label{shap1_X}\varphi_i(\langle N,v\rangle) = \sum_{S\subset N, i\notin S}\frac{(n-|S|-1)!\,|S|!}{n!}(v(S\cup\{i\})-v(S))\end{equation}
%That is, under the Shapley value each player is afforded their average marginal contribution across every possible sequence of player join orderings.  
%Or, if $v_{i,k}$ is the average marginal contribution which player $i$ can make across coalitions of size $k$:
%\begin{equation}
%v_{i,k} = \frac{1}{\binom{n-1}{k}}\sum_{S\subset N\setminus \{ i\} , |S|=k} %\frac{(n-|S|-1)!\,|S|!}{(n-1)!}
%(v(S\cup\{i\})-v(S))
%\end{equation}
%then the Shapley value can be expressed as an average:
%\begin{equation}\label{shap2_X} \varphi_i(\langle N,v\rangle) = \frac{1}{n}\sum_{k=0}^{n-1}v_{i,k} \end{equation}

Though the Shapley Value is conceptually simple, its use is hampered by the fact that its total expression involves exponentially many evaluations of the characteristic function (there are $n\times 2^{n-1}$ possible marginal contributions between $n$ players).
\DIFdelbegin %DIFDELCMD < 

%DIFDELCMD < %%%
\DIFdelend However, since the Shapley Value is expressible as an average over averages by Equation~\eqref{shap2}, 
it is possible to approximate these \DIFaddbegin \DIFadd{inner }\DIFaddend averages via sampling techniques, and \DIFaddbegin \DIFadd{then }\DIFaddend these averages are naturally stratified by coalition size\DIFdelbegin \DIFdel{.
}\DIFdelend \DIFaddbegin \DIFadd{, forming an instance of stratified sampling.
For a given inner average in the Shapley Value expression, we approximate such an average by randomly selecting marginal contributions, calculating the sample average.
Thus the question then becomes how many samples should we compute for each strata to attain the best estimate of the Shapley Value.
}

\DIFaddend In previously published literature, other techniques have been used to allocate samples in this context \DIFaddbegin \DIFadd{of Shapley Value sampling approximation}\DIFaddend , particularly simple sampling \citep{DBLP:journals/cor/CastroGT09}, Neyman allocation \citep{CASTRO2017180,DBLP:journals/tsg/OBrienGR15}, and allocation to minimise Hoeffding's inequality \citep{2013arXiv1306.4265M}.

We assess the benefits of using our bound by comparing its performance to the approaches above in the context of some example cooperative games, with results analysed in Section~\ref{sec:discussion}.
The example games are described below:

\begin{example_game}[Airport Game]
An $n=15$ player game with characteristic function:
$$v(S)=\max_{i\in S}w_i$$
where
$w=\{w_1,\dots,w_{15}\} %\scriptstyle\scriptsize
=\{ 1, 1, 2, 2, 2, 3, 4, 5, 5, 5, 7, 8, 8, 8, 10\}$.
The maximum marginal contribution is $10$, so we assign $D_i=10$ for all $i$.
\end{example_game}

\begin{example_game}[Voting Game]
An $n=15$ player game with characteristic function:
$$v(S)=\begin{cases}
       1, &\quad\text{if}\quad \sum_{i\in S}w_i>\sum_{j\in N}w_j/2\\
       0, &\quad\text{otherwise}\\
     \end{cases}$$
where 
$w=\{w_1,\dots,w_{15}\} %\scriptstyle\scriptsize
=\{ 1, 3, 3, 6, 12, 16, 17, 19, 19, 19, 21, 22, 23, 24, 29\}$.
The maximum marginal contribution is $1$, so we assign $D_i=1$ for all $i$.
\end{example_game}

\begin{example_game}[Simple Reward Division]
An $n=15$ player game with characteristic function:
$$v(S)=\frac{1}{2}\left(\sum_{i\in S}\frac{w_i}{100}\right)^2$$
where
$w=\{w_1,\dots,w_{15}\} = \{ 45, 41, 27, 26, 25, 21, 13, 13, 12, 12, 11, 11, 10, 10, 10 \}$\\
The maximum marginal contribution is $1.19025$, so we assign $D_i=1.19025$ for all $i$.
\end{example_game}

\begin{example_game}[Complex Reward Division]
An $n=15$ player game with characteristic function:
$$v(S)=\left(\sum_{i\in S}\frac{w_i}{50}\right)^2 - \floor[\Bigg]{\left(\sum_{i\in S}\frac{w_i}{50}\right)^2}$$
where
$w=\{w_1,\dots,w_{15}\} = \{ 45, 41, 27, 26, 25, 21, 13, 13, 12, 12, 11, 11, 10, 10, 10 \}$\\
In this game, we assign $D_i=2$ for all $i$.
\end{example_game}

%\input{table-2.tex}





\begin{table*}[]
    \centering 
    \begin{minipage}[]{0.8\textwidth}
        \centering
        \caption{Airport Game Average Errors}\label{tab1}
			\begin{tabular}{llllll}
			\hline
			$\nicefrac{m}{n^2}$ & $10$ & $50$ & $100$ & $500$ & $1000$ \\
			\hline
			$e^{Ma}$   & 298.4 & 133.1 & 99.64 & 41.96 & 29.26 \\
			$e^{sim}$  & 357.8 & 146.1 & 106.2 & 44.55 & 36.33 \\
			$e^{Ca}$   & 325.7 & 115.8 & 75.85 & 31.01 & 22.12 \\
			$e^{SEBM}$ & 259.2 & 73.8 & 54.76 & 7.71 & 1.30  \\
			\hline
			\end{tabular}
    \end{minipage}
	\\\vspace{4mm}
    \begin{minipage}[]{0.8\textwidth}
        \centering
        \caption{Voting Game Average Errors}\label{tab2}
			\begin{tabular}{llllll}
			\hline
			$\nicefrac{m}{n^2}$ & $10$ & $50$ & $100$ & $500$ & $1000$ \\
			\hline
			$e^{Ma}$    & 131.0 & 57.78 & 41.52 & 18.66 & 13.18 \\
			$e^{sim}$   & 145.7 & 59.72 & 40.31 & 17.56 & 12.84 \\
			$e^{Ca}$    & 142.1 & 47.35 & 31.05 & 14.08 & 9.800 \\
			$e^{SEBM}$  & 122.8 & 47.44 & 33.18 & 8.55 & 1.995  \\
			\hline
			\end{tabular}
    \end{minipage}
	\\\vspace{4mm}
    \begin{minipage}[]{0.8\textwidth}
        \centering
        \caption{Simple Reward Division Game average errors}\label{tab3}
			\begin{tabular}{llllll}
			\hline
			$\nicefrac{m}{n^2}$ & $10$ & $50$ & $100$ & $500$ & $1000$ \\
			\hline
			$e^{Ma}$    & 25.68 & 11.62 & 7.792 & 3.481 & 2.290 \\
			$e^{sim}$   & 22.10 & 9.045 & 6.218 & 2.642 & 1.938 \\
			$e^{Ca}$    & 22.37 & 8.925 & 6.692 & 2.727 & 1.940 \\
			$e^{SEBM}$  & 19.25 & 7.044 & 5.158 & 1.183 & 0.2817  \\
			\hline
			\end{tabular}
    \end{minipage}
	\\\vspace{4mm}
    \begin{minipage}[]{0.8\textwidth}
        \centering
        \caption{Complex Reward Division Game average errors}\label{tab4}
			\centering
			\begin{tabular}{llllll}
			\hline
			$\nicefrac{m}{n^2}$ & $10$ & $50$ & $100$ & $500$ & $1000$ \\
			\hline
			$e^{Ma}$   & 276.1 & 118.9 & 87.00 & 40.15 & 27.44 \\
			$e^{sim}$  & 251.4 & 108.0 & 78.63 & 34.64 & 26.82 \\
			$e^{Ca}$   & 290.5 & 116.5 & 81.82 & 35.70 & 26.50 \\
			$e^{SEBM}$ & 214.2 & 78.47 & 54.10 & 12.45 & 2.711  \\
			\hline
			\end{tabular}
    \end{minipage}
    \vspace{3mm}
    \caption{Average absolute errors in the Shapley value calculation across all players in the four cooperative games (units in $10^{-4}$), for the different sampling schemes with different sampling budgets $m$ per number of strata (with $n^2=15^2$ for all).}
    \label{Table2}
\end{table*}



For each game, we compute the exact Shapley Value, and then the average absolute errors in the approximated Shapley Value for a given budget $m$ of marginal-contribution samples across multiple computational runs.
The results are shown in Table \ref{Table2}, 
where the average absolute error in the Shapley Value is computed by sampling with Maleki's method \citep{2013arXiv1306.4265M} is denoted $e^{Ma}$, $e^{sim}$ is Castro's stratified simple sampling method \citep{DBLP:journals/cor/CastroGT09}, $e^{Ca}$ is Castro's Neyman sampling method \citep{CASTRO2017180}, and $e^{SEBM}$ is the error associated with our method, SEBM. 
The results in Table~\ref{Table2} show that our method performs well across the benchmarks. 
A discussion of all of the results is considered in the next section. 







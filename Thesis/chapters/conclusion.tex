
\chapter{Conclusion}\label{sec:the_thesis_conclusion}

%* reiterate the design desiderata that any new considerations be flexible to account for any electricity system detail. X
%* account for specific things addressed in the introduction - such as frequency shift values and the value of inertia -or- effective inertia X
%* value of something by integrated counterfactual (what would happen if such and such were absent) directly linking to reasonable value ideas X
%* integration of the idea of storage (future work of diachronic value of electricity - not just instantaneous) X
%* mention the hetrodgeny of devices and different operating modes - eg. when do you charge your car, and what is the expected value to yourself vs others.
%* mention uncertainty in supply and demand, islanding and future time predictions and valuation over uncertainty
%
%remention ethical critieria:
%* the idea of a level playing field and equality of participation / equalitiy of opportunity
%* reconsider the ideas of utilitarianism embodied in the GNK value, acknowledge equality deficit tradeoff. supply possibility of meta-utility corrections
%
%* mention core failing of freedoms, ensuring individual rationality and why it is important
%* mention not addressing strategising and potential consequences
%* mention the difficulty in extending the GNK value (and similar shapley value) to larger networks and reflect on the accomplishments put forward in the stratified sampling chapter.



Throughout this thesis we have considered the research question: `how \textit{should} electricity be valued and traded?' which we introduced in Chapter \ref{cha:intro}.
We introduced this question by considering the various ways in which the changing nature of electricity supply and demand is provoking a discussion of different possible market structures to address emerging problems.
In that chapter we considered particular problems that are emerging and the different possible market structures proposed to address them, particularly:
\begin{enumerate}
\item	How the integration of intermittent renewable generation is making it more likely that more expensive generators set the marginal price points in the network, and how demand response aggregators could mitigate this issue by making demand more elastic. (section \ref{sec:intro_part1})
\item	We considered how intermittent renewable generators are replacing traditional power generators with frequency-stabilising inertia, and how DERs such as batteries and EVs may provide rapid response storage of electricity to offset this issue, possibly aggregated as VPPs bidding into capacity markets. (section \ref{sec:intro_part12})
\item	We also mentioned how DERs are making consumers less dependant on the grid itself, which may manifest in an untapped capability of consumers (as `prosumers') to support each other and the grid on distribution level energy networks, such as in a possible P2P energy management scheme. (section \ref{sec:intro_part13})
\end{enumerate}

We introduced these examples as to highlight the importance of the research question with its generality.
Particularly, The first point raises the question of how demand-elasticity should be valued, traded and/or aggregated. This dynamic in electricity markets is not often seen in other markets for other goods; where there is recognisable value attached to -not- consuming a respective good in order to avert changes in the marginal price point/s of the market.

This consideration is further highlighted in the second point where stochastic intermittency in supply and demand creates value for capacity (for generation and/or consumption and/or the lack thereof) for frequency correction across small timescales.
One of the proposed mechanisms to address this issue is the addition of DERs such as EVs into the network, however this can also potentially incorporate additional complexity in coordination and value.
Electric vehicles (EVs) have constraints across-time on their consumption or generation capacity, subject to user needs and battery capacity, and this intersects with uncertain needs of the network including other EVs and their charging requirements and availability; how should electricity supply/demand from resources be traded in a stochastic environment with diachronic constraints on resource availability?

In our third point it is seen that electricity supply and consumption is also increasingly going to be (or potentially, effectively be) on distribution level energy networks between prosumers, and such interactions will have to be constrained by the requirements of those networks. Such local network requirements include such factors as line-voltage limits that may depending on network topology and connection point, real and reactive power supply and balancing, and considerations on stochastic and anticipated local demands and supply.

Between these considerations the question `how \textit{should} electricity be valued and traded?' - especially if such a question is considered holistically, becomes quite difficult to answer.
Currently, the Australian system consists of a patchwork of markets, such as: spot, day-ahead, and ancillary markets, and these address the current requirements of electricity supply to consumers.
However, the upcoming changes are anticipated to stretch the current system, and such pressures may be solved by further patching the current system, such as by including additional market structures (eg. introducing a two-sided market structure), or changing/reforming those existing ones (eg. passing regulations treating DER aggregators as if they were generators) - see references in section \ref{sec:intro_part1}.

However there is the potential to approach the problem more holistically, and attempt a more general answer that incorporates all possible factors and electrical system details and constraints, to determine the operating point and transactions between all the parties of the system.

In this thesis, we considered a background of general approaches to allocation, in Chapter \ref{cha:solutions} we considered VCG, LMP, Shapley Value, Bargaining theory and Envy-freeness rules.
What is notable about these approaches is that they have varying levels of flexibility in the systems in which they can be made to apply to.
We considered them as they may be extensible to give different and general answers to account for electricity system considerations that may be relevant in application, and we gave some detail about how they have been proposed as such.

Each of these approaches have features, quirks and problems, and are situated within a wider discussion and academic context. However we saw one of the more common features between them was the broad notion of marginal differences.
Most articulated in LMP, the marginalism principle accords participants in proportion to the marginal difference they make to the operating point of the system; conversely in VCG, the principle accords participants according to the difference they make to the operating point of the system, which is also often marginal.
The Shapley Value principle accords participants in proportion to the difference they make to the operating point of the system in expectation under uncertainty of the contribution of others.
This marginal difference is defined with reference to what would otherwise happen, most directly associated with the `threat' point in bargaining theory.

From these approaches in Chapter \ref{cha:solutions} we can trace an idea that the value of an electrical resource should be -in some way- related to the value difference that its presence could make in the presence of other resources.
We attempted to distil this notion by extending Nash Bargaining theory with Shapley Value axioms to many players, resulting in the development of the GNK value, presented in Chapter \ref{cha:new_solution}.

The GNK value was developed as an extension of Nash Bargaining solution concept to many players, taking into consideration the leverage in zero-sum bargaining between all possible pairs of coalitions and the difference the inclusion of a specific resource would make in that context.
In this way the GNK value aggregates all the possible strategic considerations and counter-considerations of marginal differences an the context of ideal competition between all parties.
The GNK was designed to be extensible to a generalised strategy space, and thus applicable to any context where there is discernible strategies, agents and known valuations over strategic outcomes. 
In Chapter \ref{cha:new_solution} we introduced the GNK value by its setting and axioms, and gave an example application to small electricity networks under DC approximation.
We compared the GNK value against LMP, VCG and Shapley Value in a small-scale electricity context and witnessed the differences in the outcome between them, which we articulated on a point-by-point basis.
We identified that one primary obstacle in the application of the GNK value and Shapley Value to electricity networks was the computational complexity involved in their calculation, which the next Chapter \ref{sec:scaling} was intending to address.
In Chapter \ref{sec:scaling} we considered two different techniques which could scale the GNK to larger sized networks, particularly using sampling techniques and a proxy inplace of the GNK's inner terms.
We compared different sampling techniques, and developed our own novel sampling technique for sample-approximating the GNK and Shapley Values called the Stratified Empirical Bernstein Method (SEBM) in Chapter \ref{chap:stratified_sampling_chapter}.
We also provided a discussion of the merits of the GNK value against ethical criteria identified in our philosophy Chapter \ref{sec:philosophy} against the results witnessed on a larger randomly generated network.

From our discussion in section \ref{sec:GNK_value_discussion} we considered the central failing of the GNK value, that of not preserving the ideal of `individual rationality', meaning that the GNK value can leave individuals at a loss for their participation.
This quality was identified as being axiomatic in the derivation of the VCG mechanism, but was not part of the derivation of the GNK value in favour of other characteristics, particularly having continuum with the Nash bargaining solution and the possession of Shapley Value axioms.
It was hoped that the GNK would be seen to posses individual rationality at scale, but unfortunately this was not witnessed.

In chapter \ref{sec:philosophy} we began by introducing the philosophy surrounding distributive justice in which we acknowledged the vagaries surrounding ethics itself, and the inherent ethical nature of the research question: `how \textit{should} electricity be valued and traded?'.
We considered different ethical notions and ideals that are discussed in literature in relation to electricity systems, including: Equality, Equity, Efficiency, Freedom, Proportionality, the minimisation of Envy, and the context of environmentalism.
Ultimately we feel that the GNK value did not satisfy many of those ideals, and our research serves as a negative result of a particular kind of approach to this hard social problem.
It is possible to contend that if we had started with a better footing defined with more concrete aim and goals we would have arrived at a more satisfying conclusion.
However the proper question of how electricity should be traded has inherent vagueness, and there are many possible ways to miss a vague target:

\begin{displayquote}
``Far better an approximate answer to the \textit{right} question, which is often vague, than an \textit{exact} answer to the wrong question, which can always be made precise.'' \cite{10.2307/2237638}
\end{displayquote}

In considering our approach at a broader level, it is interesting how perfect and idealised competition may or may not coincide with what is equal and ethical.
The question of when and where these coincide, and particularly if they might coincide in the context of electricity networks, was a primary motivation of our research.

The GNK value is not only an extension of Nash bargaining solution concepts, developed and extended in game-theory literature from von Neumann to Neyman \& Kohlberg, but also extends this work to the space of generalised games, such as may exist in the context of applied electricity networks.
In this way we make a contribution to game-theory generally.

Furthermore, the statistics we develop in Chapter \ref{chap:stratified_sampling_chapter} genuinely extends knowledge about (and the use of) empirical concentration inequalities, which are quite recently discussed and applied in various spheres. However it als extends it from the realm of simple random sampling to that of stratified random sampling and this new domain involves more complexity and structure.
In this domain the Stratified Empirical Bernstein Bound (SEBB) is unique, it is an empirically guided concentration inequality tailored to the historic and familiar Stratified Sampling domain.

Through our work we have provided a unique interpretation of the Shapley Value axioms in the electricity domain, and the techniques employed in Chapter \ref{sec:scaling} demonstrate the potential of computing with these axioms to larger networks. As already stated, we view it as an accomplishment that it is possible to reasonably approximate the GNK value for 100-bus nodal networks,
whereas if it were to be calculated exactly would involve a prohibitive $\sim 2^{100}$ optimisation terms.

In concluding this research there are still outstanding questions which could prove to be fruitful future research directions, these include:
\begin{itemize}
\item	Considering game-theoretic measures (VCG, GNK, Shapley Value, Bargaining, etc.) to more directly address real-world situations such as those alluded to in Chapter 1, such as involving the coordination of EV in networks with stochastic demand and network and frequency constraints, and the evaluation of them in such contexts.
\item	Further analysis could be conducted analysing how all the different mechanisms for financial valuation of electricity resources would potentially yield incentives to change the structure of the electricity network itself (and hence the value of the resources therein) - particularly if such measures that only compensate resources that can impact the network operating point (ie. LMP and VCG) would create network fragility, whereas those which reward resources that do not, might further lead to network changes overly reinforce robustness.
\item	The further examination of the relationship between GNK and M-GNK values, and/or the substitution of other measures of `threat' between parties on an electricity network context. Particularly to see if such measures can be established to restore the `individual rationality' property.
\item	The evaluation and extension of measures such as the GNK value to situations involving strategising agents. Particularly modifications which would give `incentive compatability' property, such as surrounding the existing extensions and discussion of \cite{myerson1,Salamanca2019}.
\item	The extension of computational techniques to the GNK/Shapley Value measures to make them computable to even larger electricity networks, Some possible avenues of investigation include employing further approximations such as player clustering (such as implemented by \cite{DBLP:journals/corr/abs-1903-10965}), or transforming the problem into a non-atomic form, similar to non-atomic Shapley Value.
\item	Improvement and experimentation on the concentration inequalities for stratified random sampling methodology. The bounds on the moment generating functions that we developed in Section~\ref{section:SEBB} use various loosening approximations, and hence stronger and/or more representative bounds could be developed at the cost of greater mathematical complexity.
Alternatively, an approach utilising entropic \citep{Boucheron_concentrationinequalities} or Efron-Stein inequalities \citep{efron1981} could yield different and potentially tighter results.
\item	Extension of the Stratified Empirical Bernstein Bound (SEBB) to even more practical situations. Particularly the concentration inequality was derived on the assumption of given and known strata and strata sizes, which does not necessarily correspond to situations encountered by practising statisticians, where the sizes of strata may in some cases only estimated and/or where the strata divisions are flexible and/or constructed from preliminary surveys.
\end{itemize}





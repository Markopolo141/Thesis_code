\chapter{Conclusion}
\label{cha:conc}
Summary your thesis and discuss what you are going to do in the future in Section~\ref{sec:future}.

The derivation of our inequality extends from consideration of Chernoff bounds and probability unions in a similar vein to other EBB derivations \citep{Maurer50empiricalbernstein,bardenet2015}.
%, however we do not argue that our particular concentration inequality is ideal.
However, the bounds on the moment generating functions that we developed in Section~\ref{sec:components} use loosening approximations, and hence stronger and/or more representative bounds could be developed at the cost of greater mathematical complexity.
Alternatively, an approach utilizing entropic \citep{Boucheron_concentrationinequalities} or Efron-Stein inequalities \citep{efron1981} could result in different and potentially tighter results.

% Additionally, although our method works generally, there may be better or more appropriate sampling methods in the event that there is more information known about the underlying distributions.
% It is sometimes possible to derive ideal concentration inequalities in restricted circumstances, and more broadly there exist some computational methods to numerically derive ideal bounds \citep{OUQ1,doi:10.1137/13094712X}.
% Using these techniques it may be possible to derive ideal numerical bounds, particularly for bounds considering very small numbers of samples.



% Full exploration of the potential applications are beyond the scope of this document.
% However, at present, we are pleased to present our analytic concentration inequality (Equation \ref{big_equation}) as an immediately computable expression and practical method for choosing samples from strata.



\section{Future Work}
\label{sec:future}
Good luck.




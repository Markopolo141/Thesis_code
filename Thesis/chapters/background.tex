\chapter{Background}
\label{cha:background}
%At the begging of each chapter, please introduce the motivation and high-level
%picture of the chapter. You also have to introduce sections in the
%chapter. \\

The Australian electricity grid is seen as an evolving system of increasingly interconnected and complex devices, providing a structured platform for the exchange of power and services between people.
However at the heart of this evolving construction there is a morally ambiguous question - by what mechanisms should it be determined how much people \textit{should} pay or be-paid for the power they consume or generate?

Or, to ask the question with more granularity: when the possible power-flows on an electricity network are valued and influenced by participants differently, 
what is a reasonable electrical outcome for the network and what monetary transactions should occur between the participants in that case? -- and this is to ask: how \textit{should} electricity be traded?\\

To begin this discussion we briefly address the state of the electricity system and its possible trajectory into the future in section \ref{sec:electricity_system},
and then we also begin to frame some of the philosophy underlying the question of how the financial transactions on that future network should look like in section \ref{sec:philosophy}.

\section{The state and future of electricity systems}\label{sec:electricity_system}

Electrical power systems have a history of evolution over time, particularly with the introduction of new technologies and new damands.
And this trend continues today as there are divergent ways that new technology is driving a need for additional changes.

One of the primary changes is the continued proliferation of renewable energy technology, and the retirement of older generators - particularly coal powered generators.
For instance, between the fiscal years 2008/2009 - 2018/2019 the volume of renewable electricity generated on Australia's National electricity market (NEM) increased from 18,645 GWh to 44,292 GWh, an increase of approximately 10\% per year, increasing the proportion of renewable energy in the network from 7.5\% to 17\% over the same period (or 1.7\% growth in proportion per year).
(Table O - Department of Energy \& Environment, Australian Energy Statistics) 
This increase was primarily in the deployment of wind and solar generation, with 90\% of solar generation currently being created by small scale solar PV systems.
And this increasing trend is expected to continue, there were ~300,000 installations in the year 2019 - clean energy australia report 2020, with solar PV generation projected to triple by the year 2030 (AEMO observations)).

%in addition to small solar PV system installations  there was an array of medium and large scale solar installations.
%In the same time additional household batteries are increasingly being installed (with 22,661 household battery installations in the 2019 year) taking australias household battery storage to 1GW. with an addtional 15 large scale battery projects scheduled for completion in the 2019 year ranging between 1 and 150MW capacity.
%(clean energy australia report 2020)
One consequence that these solar PV systems are creating is that they are changing the demand profile on the network.
Solar PV systems are often behind-the-meter and have the effect that they change the amount of electricity that consumers require throughout the day from the grid, particularly in the middle of the day the abundance of solar energy offsets household electricity requirements and create the ever increasingly severe decrease in demand - the infamous 'duck-curve' - which is identified to be one source of potential future difficulty.
Not only does solar PV change the average demand profile on the network but it also introduces variability; because solar output is linked to weather there is the ready potential for large swings of solar power output, creating issues related to frequency and voltage control which need to be addressed, and which need accurate forcasting to anticipate.
These demand side changes occur alongside supply side changes such as the retirement of coal generation capacity, and continued uptake of industrial wind generation capacity - further creating problems associated with stability.

A consequence of the changing dynamics of the grid, is that these combinations of changes has resulted in inordinate increase in AEMO direct intervations (AEMO) , and AER rule changes (KPMG) in order to address the emerging situations.
More generally, this increased variability of demand and supply are part of a changing electricity system that need to be managed and addressed, and there are a variety of potential options.

It has been identified that this combination change in mix and concentration of supply and demand (the closure of coal fired power stations, the leveling off of grid demand (partly caused by solar PV systems and other technology changes) and the incorporation of further renewable generation) as been part of the reason that the average price of electricity for consumers between 2008 and 2018 has increased 35\%. (Restoring electricity affordability and Australia’s competitive advantage)

Indeed a wide variety of potential options are presently been considered and developed to ensure future stability and efficiency of the grid at providing customer needs - which are core components of Australia's National Electricity Objective (NEO)\footnote{the NEO is part of australia's National electricity law (NEL) ``to promote efficient investment in, and efficient operation and use of, electricity services for the long term interests of consumers of electricity with respect to: price, quality, safety and reliability and security of supply of electricity; and the reliability, safety and security of the national electricity system.''}

One example of such plans is the current COGATI investigation, exploring changes in the NEM that could be made to further incentivse generation and transmission at grid critical locations to minimise congestion experienced on the grid; ie. incentivising the most appropriate network generation and transmission.
Another broad example of such an investigation isn the so called `NEM2025' plan directly investigates possible changes to market structure to NEM in order to ensure future stable grid, particularly mentioning the potential of utilising DERs.

One way in which this variability is driving change is in the procurement of market systems for the procuring the production of strategic reserve power (RERT), and increasing the participant of demand response, or price resonsive demand, to avoid the risk of non-supply.
particularly the production of a market structure to produce demand-responsive supply/consumption against generation (potentially even extending to normal users of electricity systems) a `two sided market' is something that is being scoped out by COAG energy council australia (ECA) 

In these cases there is an increasing interest in the design of new market sytems that are appropriate for the future electricity grid.
Potential future markets are subject to many requirements, outside of simply delivering a stable and cheap electricity service to consumers, and sometimes they even subject to ethical design considerations such as:

"Level playing field – all competitors, irrespective of their size or financial strength, get equal opportunity to
compete. It is not enough if all players play by the same rules. The rules must accommodate the needs of all,
whether small or large, so the market is free of impediments to smaller players." (AEMO observations: Operational and market challenges to reliability and security in the NEM)

One, instance of a design decision is who should be participating on such future markets, particularly the prospect that consumers of electricity could participate in some kind of electricity market (ECA) is up for discussion.

Another change that is happening, is that increasingly not only are consumers generating their own electricity via solar pannels, but they are also expected to be increasingly storing that energy as well.
the increased proliferation of batteries on the network, and the potential proliferation of Electric Vehicles (EVs) (predicted to account for 17.7\% of vehicles on australian roads by 2036 (Electric Vehicles Insights AEMO)) is another avenue where there may be future role for electricity consumers (or so called `prosumers') to participate in network support and have access to network market structures.

However the incorporation of consumers on distribution energy networks into the national electricity market and infastructure and its accounting poses additional challenges, particularly with regards to voltage and frequency management on distribution network feeders.
One current avenue being explored to do this is the notion and project of creating virtual power plants (VPPs) as large scale aggregators of DER power capabilities. These are experimented with at the moment (Aemo's virtual power plant demonstration project)

However, more than this is the idea that the grid might be able to support consumers selling their excess power and storage capacity to each other - so called peer-to-peer electricity trading (P2P).


These distributed energy resources exist on distribution networks where there is occasional problems with voltage management, where 
solar inverter reactive power stability is not often utilised. (Smart Grid and its future perspectives in Australia)


The implications of these changes are not yet fully realised and there exist differing visions of what structure a future electricity system might have, and what it might look like for the consumers that participate in it \cite{Parag2016}.




, \cite{BELL2018765} name a few:
\begin{itemize}
\item The introduction and continued proliferation of intermittant renewable generation - such as solar and wind - is seen as driving a need for energy storage and grid interconnection to provide stability.
\item The incorporation of Distributed Energy Resources (DERs) such as smart meters, solar pannels, batteries, electric vehicles (EVs), and other electrically flexible components particularly on distribution networks, rather than transmission networks.
\item The increased need for active grid stability mechanisms due to a reduction of synchronous generation, brought about as part of the process towards a zero-carbon emission electricity grid.
\end{itemize}


In the past there have been many changes to the way that electricity has been bought and sold on the grid.
One of the most recent and important changes was the process of electricity system deregulation/liberalisation that took place in many countries throughout the late 80's into the 2000's.
Historically, electricity supply was often operated by vertically integrated companies, who had a hand in the control of the generation, transmission, distribution and/or retail of electricity services to consumers.
Part of the rationale for the regulation of these companies was that electricity supply was a natural monopoly in which economics of scale provide competitive advantage to larger companies, leading naturally to monopoly situations.
The process that occured, was that various governments were able to remove regulations over the generation and retail of electricity, and replaced them with market structures to provide a platform for competition in generation and retail of electricity.
And the introduction of such a market structure, is a challenging process which can lead to unintentioned outcomes if designed poorly.
The history and the challenges of such changes are widely understood, including some failures of the process, such as the 2000-2001 California electricity crisis. \cite{griffin2009electricity}

The process of electricity dergulation/liberalisation process was a systematic change to the electricity system that introduced market platforms for competition between entities on the grid; and many people expect a similar change to take place in the near future.
Particularly as indidivual consumers are now generating and storing their own electricity, and there is investigation about possible market platforms which could enable consumers to sell electricity between themselves.
In these possible futures these consumers are sometimes called 'prosumers', which are individuals armed with electrical devices (such as batteries, solar pannels, and electric vehicles) which can interract with the electricity network in a way that is more than simply by consumption.
And various ideas about how these prosumers might trade directly or indirectly between themself and the wider grid; involving such ideas as peer-to-peer trading (P2P) and/or aggregation into larger virtual power plants (VPPs) \cite{Morstyn2018}.

Today, network participants are payed for the power they consume/generate largely behind their own meter, inducing them to optimise their own energy consumption, but it is anticipated that a future energy market could induce larger and more social benefits than simply offsetting their own consumption.
However there are a many of potential gains and hazards to be managed, such as providing grid stability by securing timely reactive and real power supply to stabilise voltages and frequency, grid robustness such as blackout protection and islanding, improving system efficiency by minimising long-distance transmission costs, and facilitating the advent of a green electricity network, by fairly and equitably managing the interaction between prosumer's devices while preserving their privacy, in a scalable way while minimising the complexity of the management of such devices. \cite{BELL2018765}
%And a future electricity network might include more exotic components such as community level energy storage, and microgrid interconnections.

The way that electricity is being produced and consumed is expected to change in the future, from a system which supports the supply of few big generation companies to many small consumers, to being a system in which prosumers exchange energy.

And for this purpose, there are many potentially important engineering considerations in the operation of powergrids which may bear importance for the exchange of energy between such prosumers - such as voltage rises and line limits, real and reactive power compensation, phase connections, network topologies etc.
And it is hoped that a properly designed system might be extensible to be able to accommodate these considerations arbitrarily.

Additionally, one of political and social questions that the design of such a structure brings is the question of distributive justice.
In the past, it was held that inducing competition among generation companies was desirable, however it is important to ask if inducing competition among electrically generating prosumers is equally a good idea.
A part of the answer comes in the likely result: would a competitive system induce outcomes which are socially desirable and/or serve to promote social equality?

A major component of this question lies in the formulation of the arena in which the competition takes place; which is the primary design question.
And the purpose of this research is an attempt to explore a portion of the space of possible arenas, and evaluate it against desirable ethics.




\section{A Philosophical Prelude}\label{sec:philosophy}

%\begin{displayquote}
%My claim is merely that there is no single fundamental principle that determines or provides guidance on what justice requires in relation to the distribution of access to overall advantage.\cite{mason2006levelling} 
%\end{displayquote}

In considering the question of what distribution energy market structure \textit{should} be implemented, it is essential to atleast acknowledge that there exists a wide range of moral and practical factors that bear on the question.

In this chapter we briefly consider some of the moral considerations that frame the question, however in doing so, we must make clear that the deliberate brevity of this chapter is not to suggest that these moral considerations are not important and worthy of much greater treatment. But that we are not philosophers, and hence with respect, we intend only to acknowledge the moral situtation, with some of its depth.

%However these considerations and concepts are crucial and range much wider than the specific context of electricity markets, though unfortunately we cannot give a comprehensive treatment of them here.

The moral and ethical side of our question is associated with a branch of moral philosophy called \textit{Distributive Justice}, which seeks to ask and make headway on the question of how different kinds of resources (such as money/power/goods/etc) morally should be distributed in society; and this is the much broader question.

We begin by noting there is a long history of philosophical skepticism about the nature of moral knowledge and judgements, and Distributive Justice is not exceptional in this regard.

An example hisorical argument is Hume's `Guillotine' \cite{HumeGutenberg}\footnote{``For as this ought, or ought not, expresses some new relation or affirmation, it is necessary that it should be observed and explained; and at the same time that a reason should be given, for what seems altogether inconceivable, how this new relation can be a deduction from others, which are entirely different from it. But as authors do not commonly use this precaution, I shall presume to recommend it to the readers; and am persuaded, that this small attention would subvert all the vulgar systems of morality, and let us see, that the distinction of vice and virtue is not founded merely on the relations of objects, nor is perceived by reason.'' T3.1.1} which is often read as stating that: no material facts about how the physical world \textit{is}, by-themself, could ever seem to logically imply any claim about how the world (or its material components) \textit{should} be.

Another historic argument is G.E. Moore's open-question argument \cite{MooreGutenberg}\footnote{
``Moreover any one can easily convince himself by inspection that the predicate of this proposition - `good' - is positively different from the notion of `desiring to desire' which enters into its subject: `That we should desire to desire A is good' is not merely equivalent to `That A should be good is good.' ... clearly that we have two different notions before our minds.''Ch1:13\\
``If I am asked `What is good?' my answer is that good is good, and that is the end of the matter. Or if I am asked `How is good to be defined?' my answer is that it cannot be defined, and that is all I have to say about it.'' Ch 1:6}, which argues that for anything which defines what is morally good, then a question-about or statement-of that equivalence would only be tautology.

Such arguments are probably best used as discussion-starters today, however, talking about the ontological nature and the basis of moral knowledge is not our focus.
%Whether or not moral truth ultimately amounts to collective sentiment, or reduces to  statements of prudence, or is metaphysically identical/grounded in some deeper objective fact is beyond the scope of this work.
Instead, our focus tends towards some discussion around the moral views that people are likely to have upon reflection; and we give an extremely briefly survey some of the attitudes expressed in literature where relevant.

What is quite evident, is that different people have different conceptions of how the world should be, and not all of these conceptions are compatible with each other.
That any particular ethical system is likely to be rooted in a specific focus (as encoded by principles, maxims, cultural narrative, language etc) and will yeild outcomes that may be disagreeable to some people and agreeable to others.
%In constructing any particular answer to the question of what counts as a good/better/fair mechanism we will necessarily isolate those who would disagree.
%And for this we must make a modest apology.

%We could just directly constructing an answer to the question \textit{de novo}, but such an approach would miss a lot of humility and leave open the question why those elements should be selected. And so the situation leads us to give at least some consideration the space of peoples moral intuitions before settling down upon a judgment.

We attempt to give a brief survey to address what we believe are often some of the elements that feature in people's moral thinking, and do our best to develop a novel synthesis about electricity systems, which bears some relevance to these moral considerations.

While we must acknowledge the moral ambiguity inherent in the question of electricity allocation, we contend that this does not mean that any answer is simply \textit{as good} as any other. But only that we believe that the suitability of our answer is not something we can totally demonstrate, in principle.

Let us begin.

\section{Some Moral Factors}

The choice of centralized Market structures and process can be seen as a choice between methods of allocating resources between multiple parties in a system based on the parties interaction within it.
An example of a market structure might be a type of auction, and the interractions of the parties might be their choices of bidding or bidding strategy.
In this context the choice of the auction and also the resultant likely distribution of resources can be viewed as being morally/socially desirable or undesirable bassed on a number of factors.
What constitutes a desirable distribution of resources?

%In the process of discussing a choice of system it is important to come to a relatively clear understanding of what concepts are in play. And in these sections we will attempt to briefly survey and break-down some of the surrounding moral concepts.

Throughout time there have been an array of philosophers who have discussed ideas surrounding the moral distribution of resources and capital. And one of the major ideas surrounding the ethics of distribution is \textit{Equality}.

\subsection{Equality}

\begin{displayquote}
``A common characteristic of virtually all the approaches to the ethics of social arrangements that have stood the test of time is to want equality of \textit{something}... They are all `egalitarians' in some essential way ... To see the battle as one between thoes `in favor of' and thoes `against' equality (as the problem is often posed in the literature) is to miss something central to the subject."\cite[Chapter 1]{18084} 
\end{displayquote}

\begin{displayquote}
``for all men have some natural inclination to justice ... what is equal appears just, and is so; but not to all; only among those who are equals: and what is unequal appears just, and is so; but not to all, only amongst those who are unequals;\\
which circumstance some people neglect, and therefore judge ill; the reason for which is, they judge for themselves, and every one almost is the worst judge in his own cause." \cite[Politics, chapter III.9]{AristotleGutenberg}
\end{displayquote}


People tend to believe that they are, should be, or be treated, `equal' in some sense.
And this broad conception has changed throughout time and place in history \cite{themeaningofequalitycapaldi}.
From at least as far back as Aristotle \cite{AristotleGutenberg}\footnote{see section quote}, notions and concepts about equality have come from across culture and peoples, and between Spiritual \footnote{across multiple religions, eg. in Islam ``No Arab is superior to a non-Arab, no colored person to a  white person, or a white person to a colored person except by Taqwa (piety)." [Ahmad and At-Tirmithi], and in Christianity, St Paul's Galatians 3:28 ``There is neither Jew nor Gentile, neither slave nor free, nor is there male and female, for you are all one in Christ Jesus'' (NIV) } and the Materialist \footnote{Such as in Engel's Anti-D\"{u}hring Part 1 Chapter 10 ``The idea that all men, as men, have something in common, and that to that extent they are equal, is of course primeval. But the modern demand for equality is something entirely different from that; this consists rather in deducing from that common quality of being human, from that equality of men as men, a claim to equal political social status for all human beings''} thought.
Throughout the ages the way in which equality in society has been constructed and implemented has varied dramatically - even the abolition of slavery is recent by comparrison.

There is something appealing about the idea of Equality between people.
From the asthetic perspective equality is an ideal with a simple structure. From a humanitarian perspective equality is associated with relief from envy and want. From the social perspective it is associated with community and solidarity.% From the philosophical perspective equality is directly associated with fundamental and core ideas.

%What is morality and justice? and what Rights and Freedoms does it require? and are people actually equal in any morally relevant way?

On a practical level, the divergences between people's ideas of equality can be seen as regarding what things should-be equal (when, where and for whom); and also what should be done about inequalities as may exist.

The question: ``Equality of what?'' can have many answers, some of which are commonly held and seldom controversial today, such as might be gleamed from the United Nation's declaration of Human Rights: Equality before the Law, Democratic Equality to vote, equal freedoms to marry and to live, etc.
However more controversial answers tend to have broader social and political scope, such as: Equality of Opportunity, and Equality of Welfare and/or Economic Equality.% And the controversial nature of these ideas notably come to the fore in discussions such as surrounding affirmative action initiatives and also political socialism.

For some, equality is a contestable notion, or an ideal for the direction of efforts in narrow and specific contexts, but for others equality is an attainable and far-reaching goal with multifaceted implications across social spheres \cite{walzer2008spheres,millerandwalzer,baker1992arguing}.

There are many ways in which equality of specific measures has been argued; particularly, a person can advocate for equality in a given context \textit{directly} and/or \textit{instrumentally}.
%For instance, as a practical measure in some settings, and also as an extension of more fundamental moral principles.
What amounts to an equal allocation can also instrumentally satisfy other values, for instance, David Miller \cite{equalityandjustice:1998} argues for the practicality of giving equal remuneration between hypothetical employees in the context of uncertainty about how much each of them deserved (and/or lack of means about distinguishing them). And also gives some broader examples of reasons for equalities in society: for asthetic and pro-social reasons, because it can be a sufficiently practical and simple social contract, or because it might be politically inevitable. etc.
These (and others) may count as instrumental or prudent reasons to value the implementation of an equality in a specific context, but in other contexts these same reasons could potentially point to other arrangements.
However, there have been arguments directly for equality of measures, particularly from (or in light of) more abstract concepts such the notion that people have \textit{equal moral worth} \cite{doallpersonshaveequalmoralworth} or \textit{moral equality}. Though it is difficult to define\footnote{for instance, Discussion about \textit{who} has equal moral worth (or alternatively \textit{how/why} they do) seems to occasionally to turn into a discussion about the moral rights of animals, \cite{doallpersonshaveequalmoralworth}}, the notion that people have equal moral worth is felt not to logically imply any very specific kind of equality of measure per sei.

\begin{displayquote}
The distinction between ``equal treatment'' and ``treatment as equals'' expresses this difference between offering people the same treatment, and acting in accordance with the fact that they are moral equals. Equal status does not constrain us to a set of identical actions regardless of our differences.\cite{whatisbasicequalitynathan}
\end{displayquote}

The question about what things should be equal (rights, freedoms, duties, responsibilities etc, and for whom and when) can be seen as forming a large component of the various moral systems. And it is sometimes felt that moral equality simply cannot be a logical premiss for these questions.

\begin{displayquote}
The idea of moral equality, while fundamental, is too abstract to serve as a premise from which we deduce a theory of justice. What we have in political argument is not a single premise and then competing deductions, but rather a single concept and then competing conceptions or interpretations of it. Each theory of justice is not \textit{deduced from} the ideal of equality, but rather \textit{aspires to} it, and each theory can be judged by how well it succeeds in that aspiration.\cite{kymlicka2002contemporary}
\end{displayquote}

In anycase, expecting a particular person to give a precisely defined answer to the question ``Equality of what?'' may be asking too much; as even the phrases which people use in everyday life are seldom given exact specifications\footnote{Degrees of vagueness are well witnessed in everyday sentences, ``There are a gathering of people near that tree'', such as argued in the classic Sorties paradox\cite{frances_2018}}, let alone concepts pertaining to the spectra of possible societies.
So, for instance, the space of various contemporary political philosophies which faithfully attempt to construct and interpret some reasonable form of equality between persons has been described as belonging to an `egalitarian plateau'.\footnote{The phrase is originally attributed to Dworkin and subsequently adopted by others.}\cite{Brown2007}
Or conversly, while a specific ethical equality may not be agreed apon, perhaps there may be a more broadly accepted notion of what an `inequality' looks like,
particularly as it is sometimes blurred with the concept of a `social injustice'.\footnote{There is some debate as to when/where/how an inequality also becomes an injustice. It is possible to believe that an inequality constitutes an injustice directly, or perhaps that an inequality is proof (or perhaps only potential evidence) of a injustice in procedure or treatment. see parfit's concept of Telic vs. Deontic Egalitarianism.\cite{equalityandpriorityparfit}}

And in this way, the concept of Equality can be inclusive-of and also contrasted-against other views; such as thoes that emphasize the priority of resources to the poor, or such as emphasise alleviation of insufficiency among the poor; broadly termed ``prioritarianism'' and ``sufficientarianism'' respectively.\cite{sep-egalitarianism}\footnote{for good measure we might also consider Rawl's Theory of Justice \cite{rawls2005theory} as a specific kind of (layered) priority principle.}

Although the concept of equality is the subject of wider analysis, we will focus on two specific interpretations of equality which we feel can be made relevent to mechanisms for electricity allocation.

\subsection{Formal Equality}\label{sec:formal_equality}

One primary interpretation of equality is that people should be subject to systems that treat them in a manner that is \textit{impartial}. The minimal idea is that an impartial system should not afford arbitrary or unjustified special treatment toward any particular individual/s. Hence that systems should operate by rules which are blind to particular identity and sensitive only to morally relevant characteristics.

In literature this idea of moral impartiality to particular identity has been a component of various thought experiments and also stated with moral maxims.
Particularly famous devices include Rawl's ``Original Position'', Kant's categorical imperatives, or various positions defined by hypothetical ideal sympathy and/or perfect detachment \cite{smithGutenberg, nla.cat-vn197822,10.2307/2103988}.

Additionally the idea of impartiality is perhaps mathematically expressible, in that people who are (in all relevant ways) equal should be treated equally; and this is known as \textit{formal equality} \cite{whatisbasicequalitynathan}. Although formal equality is occasionally seen as being an important part of a fair system, it is also sometimes seen to be insufficient to capture broader notions of equality and justice.

\begin{displayquote}
In its majestic equality, the law forbids rich and poor alike to sleep under bridges, beg in the streets and steal loaves of bread.\\
--Anatole France, Le Lys Rouge [The Red Lily] (1894), ch. 7
\end{displayquote}

Indeed, by imagination many kinds treatment or processes could be rationalised as being issued by impartial principles which are universally applied. Additionally, not all kinds of desirable impartiality are mutually compatable, or perfectly achievable in practice.\cite{Hutchinson_2019}

Notwithstanding, formal equality can be seen as a basic doctrine that ascribes value to the incorporating degrees (and/or kinds) of impartiality into the design of social processes from the outset.%; even if they never truly reach perfect impartiality in practice.
% that social systems should be designed with an aim to treat people identically, but-for morally relevent characteristics.
%And thus there is perhaps some value in incorporating some degree (and/or kinds) of impartiality into the design of social processes from the outset; even if they never truly reach perfect impartiality in practice.
%in the fact that some degree (and/or kinds) of impartiality might be incorporated into the design of social processes from the outset; even if they never reach perfect impartiality in practice.
%It might also be profitable to consider the inequalities between the different kinds of rules (and treatement) that people can be effectively subject to \textit{in practice}, or effectively \textit{by consequence}.

In all developments in this thesis, the principle of formal equality - that individuals are treated equally but-for specific factors - is assumed.

\subsection{Equalities of social freedoms}

\begin{displayquote}
I want to emphasise what is, on my view, the most important object of egalitarian distribution, and that is \textit{power}. Of course power is not something which can be parcelled up and shared out like a commodity but we can properly talk of `the distribution of power' and this is, more than anything, the determinant of whether a community is authentically cooperative.\cite{TheSocialBasisofEquality:1998}
\end{displayquote}

There are different and interrelated ways of how to concieve of wider social equality, and one historic way of framing social equality is in terms of power.
For some people, the ideal of equality encodes the hope of a society free of abusive power relations that perpetuate social injustices.

One of more historically notable instances of this framing is featured in Marxist thought, which focuses on abusive economic power relations between social classes. This frame also shows up historically in feminist thought (eg. see \cite{Cudd2006-CUDAO}) where the inequalities of power between men \& women are considered as a form of oppressive dominance \& submission.\footnote{eg. MacKinnon writes ``difference is the velvet glove on the iron fist of domination. The problem is not that differences are not valued; the problem is that they are defined by power''\cite{mackinnon1989toward}.}

But what is notable is that neither Marxist nor feminist writers always viewed power itself, negatively.
For instance, Marx opposed private property (as capitalistic ownership) but seems to have had a more complex attitude toward property relations generally.\footnote{"the theory of the Communists may be summed up in the single sentence: Abolition of private property ... Do you mean the property of the petty artisan and of the small peasant, a form of property that preceded the bourgeois form? There is no need to abolish that"\cite{MarxGutenberg}\\"Property thus originally means no more than a human being's relation to his natural conditions of production as belonging to him, as his, as presupposed along with his own being; relations to them as natural presuppositions of his self, which only form, so to speak, his extended body."\cite[Notebook V]{marx1993grundrisse}}
Additionally some feminists occasionally consider power in the positive (or potentially neutral) language of \textit{empowerment} \cite{doi:10.1111/j.1527-2001.1998.tb01350.x}.

One of the features of power that is associated with abuse, is the exersize of `power over' other people, or `power to' do things which impinge apon other's rights.\cite{doi:10.1111/j.1527-2001.1998.tb01350.x}
But disecting when and where an exersize of power consititues an abusive or morally objectionable act may not be easy.
Particularly the `power to' do something is straightforwardly an example of a freedom, and one well known dichotemy exists between \textit{positive} and \textit{negative} freedoms\footnote{While vague, a negative freedom is associated with an absense of external obstacles to conducting the specific action, and a positive freedom is associated with the possibility (or actuality) of doing the act in accordance with one's will and purposes. The positive/negative dichotemy is also associated with what is or is not effortfull.\cite{Mossel2009-MOSNA}} particularly in the discussion of doing or allowing harm.
But even more broadly, freedoms can be considered as triadic relationships: a freedom \textit{of} a person, \textit{from} particular preventing conditions, \textit{to} do certain things.\cite{Negative_and_Positive_Freedom}

However different freedoms are not equally valued (or compatable), and some are esteemed by individuals and societies more than others.
Some would place an importance on political freedoms (to openly discuss, vote, and run for office) or economic freedoms (to work, to buy, sell and lease property), etc. %\footnote{indeed perhaps too many things, such as, as ownership of resources and money, occupying a social status, opportunity for advantage can be considered as being freedoms}
But particularly, the having and actualising of freedoms associated with the meeting of needs; such as basic needs (of shelter, food, etc - as at the botton of Maslow's heirarchy) as well as higher needs (such as social belonging and self-actualisation); can be considered as defining of human wellbeing, and perhaps even a constituent of the state of having `Freedom' - the moral and political ideal.

Unfortunately most of these (and other) wider conceptions of societal Equality are beyond the scope of what we can earnestly engineer directly. But what we can do is to reflect and evaluate the influence that any proposed system might have on the freedoms of individuals and the wellbeing of society - and this is a task we attempt in later section \ref{sec:GNK_value_discussion}.

\subsection{Efficiency and Utility maximisation}\label{sec:philosophy_efficiency}

%\begin{displayquote}
%Essentially, Utilitarianism sees persons as locations of their respective utilities %- as sites at which such activities as desiring and having pleasure and pain take place.
% ... Persons do not count as individuals in this any more than individual petrol tanks do in the analysis of the national consumption of petrolium.
%\cite{}
%\end{displayquote}

\begin{displayquote}It is the greatest happiness of the greatest number that is the measure of right and wrong. \textit{A Fragment on Government}, Jeremy Bentham\cite{bentham2001fragment}
\end{displayquote} 

It is occasionally thought that what is morally good for society should have some relationship with what is good for the indiviuals of society; and there is a question about how to characterise that relationship.
Historically what is morally good for individuals has been associated with such things as happiness \cite{burns2005happiness} or subjective welfare \cite{10.2307/2264894}, access to resources (such as electrical power) \cite{10.2307/2265047}, and/or opportunity for welfare \cite{10.2307/4320203}.%; particularly as these have been used as the object of various egalitarian and utilitarian formulations.

In more material contexts, what is good for an invidual may be associated with: access to sufficient food and medicine, monetary yeilds, opportunities for educational attainment, probability of survival etc.\footnote{many thought experiments invoke lifeboat/class-room/triage/trolley-problem circumstances, where what is good for specific individuals is unambiguous}

But for whatever measures are considered to be relevent, the question of how these quantities should combine to bear on the broader moral judgement about what is good for society, has a variety of answers.
Although it is potentially contentious, it is usefull to illustrate the question by introducing the concept of utility as a quantification of what is good for individuals.

%What is morally good for individuals has defined the historical concept of `utility' and moral philosophies that focus on the utility of individuals are generally called `utilitarian'.
The concept of utility has changed over time\footnote{historically and notably held by famous utilitarians such as John Stuart Mill\cite{MillGutenberg} and Jeremy Bentham (see section quote), who defined it in terms of happiness or pleasure/pain \cite{bentham1823introduction}}
but minimally it is concieved as a measure of the strength of the preference (or value) that a specific person does (or should, rationally, pragmatically and/or morally) attach to different possible outcomes.% - in a specific context or perhaps more generally.
\footnote{the concept can be seen to extend from the consideration that such preferences should be transitive and comparable between people. If a person prefers A to B, and also B to C, then they ought to also prefer A to C. And if one person can prefer A `more' than another prefers B - then it remains a task of invention to associate numbers to the strength of these preferences over outcomes.}
For this definition, the sum of utility is straightforwardly one of many examples of a \textit{collective utility function}, a function that aggregates the utility of individuals and hence is a possible target for moral decision making.\cite{TheoriesofValueAggregation}

As another example, some egalitarian intuitions might be satisfied by moral decision making that affords equal utility for all individuals - and in some contexts this may be appropriate, even though it might not maximise the sum of utility. This outcome could be constructed as an alternative collective utility function.

What is to be considered is that maximising one collective utility function does not nessisarily maximise the other.
Particularly this is made clear in the famous 'levelling-down objection', which is the objection that, a person implementing a strict Egalitarian distribution of utility would potentially prefer a world in which every single person had less, if it were more equal. \cite{temkin_2003, equalityandpriorityparfit}.

These considerations frame some possible articulartions of the broader contrast between the values of Equality and 'Efficiency'; where the specific concept of efficiency considered in the levelling-down objection is \textit{Pareto optimality}.
Specifically, an outcome is defined as being pareto optimal if there does not exist another outcome which is better than it for every person.
Pareto optimality is one commonly discussed and formalised efficiency condition, and is a satisfiable property in our subsequent developments.
Particularly as it is the case that maximising the sum of utility is pareto optimal \cite{TheoriesofValueAggregation} and forms an axiom in our treatment (in Chapter \ref{cha:new_solution}).

We must note that there is a range of respectable ways to aggregate the moral preferences of people which we cannot addressed here.
Additionally, in various situations the difference between sensible outcomes which are better or best for more people, and what is more equal for them, can be the subject of dispute (such as in the medical field\cite{Reidpath2012,RePEc:chy:respap:120cherp}, the provision of welfare \cite{10.2307/27522452}, and economics \cite{10.1093/oep/gpz040}). Notwithstanding the philosophical distinction between the various articulations of efficiency between people and equality among them, can remain.

%We note however that there is a range of possible collective utility functions, particularly important is the space of prioritarian solutions, where which by weighted functions prefers utility imputation to thoes worst off, which in the extreme case prioritises only the worst off, which is an articulation of Rawl's difference principle and for which some regard as being 'the' egalitarian solution.

%Indeed these approaches are hardly exhaustive, and the result of the considerations of which one is appropriate may be highly context dependant. (people should have food of sufficient nutrition, public housing should be allocated as priority to thoes in most need of it. government subsidies should maximise the total welfare of the nation etc.) 

\subsection{Fairness by proportionality to some reference point}\label{sec:reference_points}

There are various ways in which what is considered fair is determined in related to what is normative, and therefore related to the various counterfactual events that could happen.
%These may be related to the implementation of moral rules, related to what is a natural and expected condition, or moral or culturally constructed expected of people, and may in turn relate to diferences in situations which people could effect.
One of the more basic and famous examples of this relationship, is given in Mill's harm principle:

\begin{displayquote}
That the only purpose for which power can be rightfully exercised over any member of a civilized community, against his will, is to prevent harm to others. - John Stuart Mill, On Liberty, chapter 1 \cite{Mill2Gutenberg}
\end{displayquote}

%Can may be considered by many as a condition of fairness and justice.
Insofar as the harm principle is accepted, there subsequently remains a question about defining where and when harm occurs.
Among its features, harm is often considered to be negative and defined with respect to a more normative `unharmed' state.
However in more complicated cases, it is not always clear what the more normative `unharmed' state should be.%, and sometimes thus who is harmed by whom.

But also reversely, there are also various conceptions of justice which involve compensation for providing benefit to others.
For instance, in Buisiness ethics there is a viewpoint where an employee's just wage should be in proportion to their contribution to the value and productivity of the firm.\cite{sternberg2000just}
In this context, the contribution may by measured by profit relative to their absense and/or by the replacement cost of contracting equivalent work (potentially depending on which ever is more pertinant).

Particularly, the idea that people should be rewarded in proportion to their contribution above their absense individually, is most directly rendered by the VCG mechanism - see section \ref{sec:solutions_VCG}.

%Furthermore it might not be considered fair to compensate employees simply in proportion to their immediate contribution above their absense individually.

Another example of a normative reference point is also found in Buisiness ethics, particularly there is a viewpoint where a buisiness transaction (such as pay for an employee's work) is considered justified if it was attained by a process of truly-free negotiation between the parties, such as to make all parties better off than they would be otherwise.\cite{ExecutiveCompensationUnjustorJustRight} 
This truly-free exchange (sometimes called \textit{euvolentary exchange}) is particularly defined by the fact that all parties could have realistically elected to walk-away from the negotiation.\cite{Guzman2019}
The event that would be triggered if the parties did not successfully negotiate is sometimes called the `disagreement event', which is the normative reference point that determines the morality of the transaction.
Some of these ideas are mathematically rendered by various bargaining solution concepts, such as Nash bargaining - see section \ref{sec:solutions_bargaining}.

In these cases we can see instances where the morality of an event is defined by normative reference points. And this dynamic can extend even to groups of individuals, as we might consider ways that groups of individuals might be exploited even if their individual interactions are truly-free.
\begin{displayquote}
This is the main aim of John Roemer's work on [Marxist] exploitation.\\... %He defines Marxist exploitation, not in terms of surplus transfer, but in terms of unequal access to the means of production. Whether one is exploited or not, on his view, depends on whether one would be better off in a hypothetical situation of distributive equality -- namely, where one withdraw with one's labour and per capita share of external resources. 
If we view the different groups in the economy as players in a game whose rules are defined by existing property-relations then a group is exploited if its members would do better if they stopped playing the game, and withdrew their per capita share of external resources and started playing their own game.\cite{kymlicka2002contemporary}
\end{displayquote}
And this idea of allocation exceeding what any group could achieve if they withdrew to cooperate among themselves, is most directly articulated and formalised by the \textit{Core} solution concept of cooperative game theory - see section \ref{sec:cooperative_game_theory_part}.

Similarly, another solution concept in cooperative game theory is the \textit{Shapley Value} (see section \ref{subsec:the_shapley_value}), which can be summarised as allocating compensation in proportion to each individuals contribution (in expectation) above their absense, under uncertainty about the presence of other group members.

The consideration that the relevent normative reference point occurs under uncertainty (or in expectation), is featured in other formulations.
For instance, if we expect that euvolentary market exchanges would normally occur at a certain market price, then we may consider that the moral trading of goods would occur at this market price.
One famous example of this idea is featured in John Locke's short essay Venditio \cite{locke2003locke} where it is argued that a fair price for something is simply its normal market price at its location.

The idea that moral trading is defined by normalised market conditions then raises the question of how normalised market conditions should be determined,
In economics there are several approaches to integrating the dynamics of market forces, but one of the most historic frameworks is economic marginalism, and most particularly Locational Marginal Pricing (LMP) - see section \ref{sec:solutions_LMP}.\\

%https://reconstructingeconomics.com/2014/06/06/venditio-by-john-locke/

In all these cases, the morality of a situtation is defined with respect to normative reference points, which of these are most relevent (and to what extent) may depends on a range of practical and moral factors.
%, and hence the question is which (or any) should apply in a particular situation?
%In answer to this, it can be seen that there can be a difference between thoes reference points which are practically most pertinent, and thoes which are morally pertinent.
%So, in the case of valuing employee labor, there may exist differences between compensation valued by net profits/revenue actually obtained, and/or measured by the expected cost of replacement labor.

It is also worth noting that these decisions about normative reference points can be defined by social policies %(such as minimum wages and award conditions)
 and embodied in moral codes and standards; and in this thesis, we consider development of a novel synthesis that extends from a specific reference point -- see Chapter \ref{cha:new_solution}. 

%It is also worth noting that the decision about these reference points can be made clear by social policies such as minimum wages and award conditions, and these rules 
%particularly famous is Rule Utilitarianism.

%\subsection{Wider moral considerations}

%There is wide range of perspectives on distributive justice which we can only begin to survey in this chapter.
%And there are many other positions that we could address, including conceptions of fairness as issuing from envy-freeness, ideas of fairness defined by equal share of social surplus, and all the various ways these viewpoints can intersect.%, such as rule utilitarianism.

%However, amongst the plurality of the various moral formulations, there are some contextualising frames which can be considered, such as the role that moral justice actually plays in society, in terms of evolution and psychology.
%The function of justice can be considered in the context of evolution; such as the idea that our drive towards particular moral norms and principles might have extended from repeated play evolutionary equilibria in the past. \cite{Binmore10785}
%For instance, systems of justice might be seen as encoding the evolved social expectations of reciprocity, and hence surving a purpose of mitigating the biological drive towards violence and retribution.\cite{doi:10.1300/J135v02n04}
%Additionally there is a possible psychological function, where, moral attitudes about the world is potentially a part of people's psychology \cite{doi:10.1037/1089-2680.6.1.25, doi:10.1080/00981389.2019.1640337}
%It is also sometimes witnessed that what people morally think should/ought happen is associated with what they \textit{will} to be.\cite{doi:10.1080/13869795.2016.1212395, sep-moral-motivation, LITZ2009695}

\section{Summary}

There is wide range of perspectives on distributive justice which we can only begin to survey in this chapter.
And there are many other positions that we could address, including conceptions of fairness as issuing from envy-freeness, ideas of fairness defined by equal share of social surplus, and all the various ways these viewpoints can intersect.%, such as rule utilitarianism.

Moral considerations are at the heart of the question of how electricity and monetary payments should be distributed.
Unfortunately questions such as these dont have analytically demonstrable answers, but we can consider various flavours of moral ideas which people might assert.
Particularly we sumarise some of the various conceptions of Equality, Formal Equality, and Equality concerned with social freedoms.
Additionally we frame some of the concepts of Efficiency particularly as it is contrast Equality.
And highlight the ways in which morality can be defined in relation to various normative reference points.

We focus particularly on these factors as our further developments relate directly to them.
Our solution obeys formal equality principles, maximises efficiency, and is formalised by reference to normative reference points defined by idealised competition -- see chapter \ref{cha:new_solution}.

%Summary what you discussed in this chapter, and mention the story in next
%chapter. Readers should roughly understand what your thesis takes about by only reading
%words at the beginning and the end (Summary) of each chapter.




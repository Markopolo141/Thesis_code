\section{Conceptual Background}
\label{cha:background}
%At the begging of each chapter, please introduce the motivation and high-level
%picture of the chapter. You also have to introduce sections in the
%chapter. \\

The electricity grid is seen as an evolving system of increasingly interconnected and complex devices, providing a structured platform for the exchange of power and services between people.
However at the heart of this evolving construction there is a morally ambiguous question - how should it be determined how much people \textit{should} pay or be-paid for the power they consume or generate?

Or, to ask the question with more granularity: when the possible power-flows on an electricity network are valued and influenced by participants differently, 
what is a reasonable cooperative outcome for the network and what monetary transactions should occur between the participants in that case? -- and this is to ask: how \textit{should} electricity be traded?\\

%Our development relates to mechanisms which are intentioned to be practical, and a brief elucidation of some economic and game-theory concepts will ultimately be involved. However we begin with the ethical considerations.

\subsection{A Philosophical Prelude}

%\begin{displayquote}
%My claim is merely that there is no single fundamental principle that determines or provides guidance on what justice requires in relation to the distribution of access to overall advantage.\cite{mason2006levelling} 
%\end{displayquote}

In considering the question of what distribution energy market structure \textit{should} be implemented, it is essential to consider a range of moral and practical factors that bear on the question.
%If one particular market mechanism \textit{should} be implemented over another,
%or to say that one market mechanism is \textit{better} than another, begs the question of how the evaluation was determined.
%What factors make one market mechanism \textit{better than} another?

In this chapter we consider some of the moral, social and philosophical considerations that frame the question about what mechanism should be implemented - and relate them directly to potent mechanisms relevant to electricity systems.
However these considerations and concepts are crucial and range much wider than the specific context of electricity markets, though unfortunately we cannot give a comprehensive treatment of them here.
%And we elucidate them here to help articulate the motivations and perspective that a reader may-or-may not have when it comes to a particular mechanism for the allocation of electrical power; and to frame our process towards our particular solution.

The ethical side of our question is broadly associated with a branch of moral philosophy called `Distributive Justice', which seeks to ask and make headway on the question of how resources (such as money/power/goods/etc) should be distributed in society; and particularly, what does `fair' allocation mean? --- this frames our central ethical component.

There is a long history of philosophical skepticism about the nature of moral knowledge and judgements; and Distributive Justice is not exceptional in this regard.
For instance, Hume's Guillotine \cite{HumeGutenberg}\footnote{``For as this ought, or ought not, expresses some new relation or affirmation, it is necessary that it should be observed and explained; and at the same time that a reason should be given, for what seems altogether inconceivable, how this new relation can be a deduction from others, which are entirely different from it. But as authors do not commonly use this precaution, I shall presume to recommend it to the readers; and am persuaded, that this small attention would subvert all the vulgar systems of morality, and let us see, that the distinction of vice and virtue is not founded merely on the relations of objects, nor is perceived by reason.'' T3.1.1} is often read as stating that no material facts about the physical world as-it-is, could ever by-itself seem to justify any claim about how the world \textit{should} be.
Or as another instance, in G.E. Moore's open-question argument \cite{MooreGutenberg}\footnote{
``Moreover any one can easily convince himself by inspection that the predicate of this proposition—`good'—is positively different from the notion of ‘desiring to desire' which enters into its subject: `That we should desire to desire A is good’ is not merely equivalent to `That A should be good is good.' ... clearly that we have two different notions before our minds.''Ch1:13\\
``If I am asked ‘What is good?’ my answer is that good is good, and that is the end of the matter. Or if I am asked ‘How is good to be defined?’ my answer is that it cannot be defined, and that is all I have to say about it.'' Ch 1:6}, that for anything which is equivalent to what is morally good, then a question-about or statement-of that equivalence would only be a tautology - and hence what is morally good cannot be non-tautologically be defined.
Such arguments are probably best used as discussion-starters today, however, talking about the nature and basis of moral knowledge is not our focus.
Whether or not moral truth ultimately amounts to collective sentiment, or reduces to  statements of prudence, or is metaphysically identical/grounded in some deeper objective fact is beyond the scope of this work.
Instead, our source of moral considerations extends from some of the moral intuitions that people are likely to have upon reflection and consideration; and for that we briefly survey some of the attitudes expressed in literature where relevant.

What is evident is that different people have different conceptions of how the world should be, and not all of these conceptions are compatible with each other.
And that any particular ethical system is likely to be rooted in particular focus (as encoded by principles, maxims and axioms etc) and give outcomes that may be disagreeable to some people and agreeable to others.
In constructing any particular answer to the question of what counts as a good/better/fair mechanism we will necessarily isolate those who would disagree.
And for this we must make a modest apology.

We could just directly constructing an answer to the question \textit{de novo}, but such an approach would miss a lot of humility and leave open the question why those elements should be selected. And so the situation leads us to give at least some consideration the space of peoples moral intuitions before settling down upon a judgment.

We attempt to give a brief survey of what we believe are some of the elements in people's moral thinking, and do our best to draw out a particular synthesis which bears direct relevance to electricity systems.

While we must acknowledge the moral ambiguity inherent in the question of electricity allocation, we contend that this does not mean that any answer is simply \textit{as good} as any other. But only that we believe that the suitability of our answer is not something we can totally demonstrate.


Let us begin.

\subsection{Some Moral Factors}

The choice of centralized Market structures and process can be seen as a choice between methods of allocating resources between multiple parties in a system based on their interactions with it.
An example market structure might be a type of auction, and the interractions of the parties might be their choices of bidding or bidding strategy.

In this context the choice of the resultant distribution of resources can be viewed as being morally/socially desirable or undesirable.

In the process of discussing a choice of system it is important to come to a relatively clear understanding of what concepts are in play. And in these sections we will attempt to briefly survey and break-down some of the surrounding moral concepts.

Throughout time there have been an array of philosophers who have discussed the problem of distribution, and the various ideas surrounding the allocation and distribution of resources and capital. And one of the major ideas surrounding the notion of fairness is \textit{Equality}.

\subsubsection{Equality}

\begin{displayquote}
``A common characteristic of virtually all the approaches to the ethics of social arrangements that have stood the test of time is to want equality of \textit{something}... They are all `egalitarians' in some essential way ... To see the battle as one between thoes `in favor of' and thoes `against' equality (as the problem is often posed in the literature) is to miss something central to the subject."\cite{18084} 
\end{displayquote}

\begin{displayquote}
``for all men have some natural inclination to justice ... what is equal appears just, and is so; but not to all; only among those who are equals: and what is unequal appears just, and is so; but not to all, only amongst those who are unequals;\\
which circumstance some people neglect, and therefore judge ill; the reason for which is, they judge for themselves, and every one almost is the worst judge in his own cause." Aristottle, Politics, chapter III.9\cite{AristotleGutenberg}
\end{displayquote}



People tend to believe that they are, should be, or be treated, `equal' in some sense.
And this broad conception has changed throughout time and place in history \cite{themeaningofequalitycapaldi}.
From at least as far back as Aristotle \cite{AristotleGutenberg}\footnote{see section quote}, notions and concepts about equality have come from across culture and peoples, and between Spiritual \footnote{across multiple religions, eg. in Islam ``No Arab is superior to a non-Arab, no colored person to a  white person, or a white person to a colored person except by Taqwa (piety)." [Ahmad and At-Tirmithi], and in Christianity, St Paul's Galatians 3:28 ``There is neither Jew nor Gentile, neither slave nor free, nor is there male and female, for you are all one in Christ Jesus'' (NIV) } and the Materialist \footnote{Such as in Engel's Anti-D\"{u}hring Part 1 Chapter 10 ``The idea that all men, as men, have something in common, and that to that extent they are equal, is of course primeval. But the modern demand for equality is something entirely different from that; this consists rather in deducing from that common quality of being human, from that equality of men as men, a claim to equal political social status for all human beings''} thought.
Throughout the ages the way in which equality in society has been constructed and implemented has varied dramatically (even the abolition of slavery is recent in human history).

There is something appealing about the idea of Equality between people, even if it is difficult to nail down exactly how and why.
From the asthetic perspective equality is an ideal and simple structure. From a humanitarian perspective equality is associated with relief from envy and want. From the social perspective it is associated with community and solidarity. From the philosophical perspective equality is directly associated with fundamental and core ideas.

What is morality and justice? and what Rights and Freedoms does it require? and are people actually equal in any morally relevant way?

On a practical level, the divergences between people's ideas of equality can be seen as regarding what things should-be equal (when, where and for whom); and also what should be done about respective inequalities as may exist.

The question: ``Equality of what?'' can have many answers, some of which are commonly held and seldom controversial today, such as might be gleamed from the United Nation's declaration of Human Rights: Equality before the Law, Democratic Equality to vote, equal freedoms to marry and to live, etc.\cite{udhr}
However more controversial answers tend to have broader social and political scope, such as: Equality of Opportunity, and Equality of Welfare and/or Economic Equality. And the controversial nature of these ideas notably come to the fore in discussions such as surrounding affirmative action initiatives and also political socialism.

For some, equality is a contestable notion, or an ideal for the direction of efforts in narrow and specific contexts, but for others equality (conceived precisely or more broadly) is an attainable and far-reaching goal with multifaceted implications apon processes across and between social spheres \cite{walzer2008spheres,millerandwalzer,baker1992arguing}.

There are several ways in which direct equality of measure has been argued-for specifically.
For instance, as a practical measure in some settings, and also as an extension of more fundamental moral principles.
In some cases, what amounts to an equal allocation can also coincidentally satisfy other values, and thus a person can advocate for equality in a given context \textit{for itself} and/or \textit{instrumentally}.
For instance, David Miller \cite{equalityandjustice:1998} argues for the practicality of giving equal remuneration between hypothetical employees in the context total uncertainty about how much each of them deserved (and/or lack of means distinguishing them). He also gives broader examples of reasons for which we can argue for equality in various contexts: for asthetic and pro-social reasons, because it can be a sufficiently practically simple social contract, or because it might be politically inevitable. etc.
And these may count (among many other things) as instrumental or prudent reasons to value the implementation of an equality in a specific context. However in other contexts, these same reasons could potentially point to other arrangements.
However, reasons to value various equalities \textit{in-and-of-themself} can and have been argued from more abstract moral considerations.

Expecting a person to give a precisely defined answer to the question ``Equality of what?'' may be asking too much; as even the phrases which people use in everyday life are seldom given exact specifications\footnote{Degrees of vagueness are well witnessed in everyday sentences, eg. ``There is a pumpkin by that tree", such as argued in the classic Sorties paradox\cite{frances_2018}}, let alone concepts pertaining to the spectra of possible societies.
So, for instance, the space of various contemporary political philosophies which faithfully attempt to construct and interpret some reasonable form of equality between persons has been described as belonging to an `egalitarian plateau'.\footnote{The phrase is originally attributed to Dworkin and subsequently adopted by others.}\cite{Brown2007}
Or conversly, while a specific ethical equality may not be agreed apon, perhaps there may be a more broadly accepted notion of what an `inequality' looks like,
particularly as it is sometimes blurred with the concept of a `social injustice'.\footnote{There is a debate as to when/where/how an inequality \textit{simpliciter} also becomes an injustice. It is possible to believe that an inequality constitutes an injustice directly, or perhaps that an inequality is proof (or perhaps only potential evidence) of a injustice in procedure or treatment. see parfit's concept of Telic vs. Deontic Egalitarianism.}



Although the concept of equality is the subject of wider analysis, we will focus on two specific interpretations of equality which we feel can be linked directly to mechanisms for electricity allocation particularly.

%And further specific ideas about equality diverge even still, however we briefly discuss some these theses to demonstrate the disagreement between them.

\subsubsection{Formal Equality}

One primary notion surrounding the concept of equality is that people should be subject to systems that treat them in a manner that is \textit{impartial}. Although it is potentially difficult to define, the minimal idea is that an impartial system should not afford special treatment toward any particular individual, or (perhaps) group of individuals. That the system should be designed to be blind to particular identity, and sensitive to morally relevant characteristics.

In literature this idea of moral impartiality has been elucidated by various thought experiments and also stated with moral maxims.
Particularly famous examples such as imaginative devices in Rawl's `veil of ignorance' and Hare's "ideal sympathiser", Kant's Universality categorical imperative, etc.

The particular idea of impartiality is rather mathematically expressible, in that people who are (in all relevant ways) equal should be treated equally. (see Aristotle quote) This maxim that `equals should be treated equally' is sometimes known as formal equality.\cite{whatisbasicequalitynathan}
Although formal equality is (at some level) considered to be an important component of any equal system, it is generally seen to be insufficient to be itself (or to `capture') broader notions of moral equality and social justice.

For instance, an uncontroversial example of such a formal equality doctrine is `equality before the Law'; as embodied in Article 7 of the Universal Declaration of Human Rights: 
\begin{displayquote}
``All are equal before the law and are entitled without any discrimination to equal protection of the law"\cite{udhr}
\end{displayquote}
Many people might consider such a provision to be an necessary constituent of a just and fair society.
What is noteworthy is that systems of Law generally encode conditions from which people will be treated differently and unequally, and even occasionally have different applicable laws for different groups of people - such as depending on their relevant status or societal role.
Examples include lawful specific freedoms and duties for doctors and laywers, rights and responsibilities for parents, provisions for children and directions for public officials, etc.
And these laws, even though they are in a very straightforward sense imposing unequal treatment, do not impede a doctine of equality before the law, since these identities (doctor,lawyer,child,official) are regarded as being morally relevant categories for differential treatment, apart from which people ought be treated equally.

Although it is thought to be necessary, equality before the law (by itself) is often thought to be wildly insufficient to capture and ensure the broader concepts of moral equality and social justice:
\begin{displayquote}
In its majestic equality, the law forbids rich and poor alike to sleep under bridges, beg in the streets and steal loaves of bread.\\
--Anatole France, Le Lys Rouge [The Red Lily] (1894), ch. 7
\end{displayquote}

However there is perhaps some ground to interpret `equality' between people as being (or being `captured by') a kind of formal equality.
One potential way of interpreting the broader concept of moral `equality', is that everybody should have equal status, where status is interpreted to be the conditions among which (at least some) certain moral/social principles depend.
\footnote{This contrasts against other kinds of `status' as we will get to in chapter \ref{blah}}
Under this interpretation, and bluntly put, when people are treated `equally' it means that they are subject to the same rules:
\begin{displayquote}
``The idea that two persons are prescriptively equal presupposes a standard for equal treatment that both satisfy. Prescriptive equality supposes a principle. What, then, do we make of ``equal status'' or ``equal value''? ``Status'' involves no more than the fact that the thing falls under a principle. We need only identify the rules, and not the prior value of individuals, in order to have a full account of what we ought to do and why.
...
it seems that ``equal value'' means nothing more than ``falling under the same rule.'' ...''\cite{whatisbasicequalitynathan}
\end{displayquote}
In this sense, that people's having `equality' might be interpreted to be identical to formal equality, perhaps generally, or in the context of an idealised or sufficiently good rule-set.

However if this route is taken, there is some ambiguity as to what should be considered as a set of rules or principles by which people can have this kind of equality.
As we might imagine that a society with lawful slavery might have a singularly coherent and universally applied rule-set; in that context it might be said that people would effectively be subject to different rule-sets, but that is also true (in-degrees) of societies generally.

Indeed by imagination almost any treatment or process could be rationalised as being by a principle which is universally applied. And in this way it might be said that any system technically satisfies a formal equality.
Although formal equality such as  `equality before the law' might be a widely held ideal, and may not in-practice be perfectly implemented in a real context; and this may not be problematic. If we conversely consider a society in which everyone were technically subject to different rules and treatment (which is inevitably in-practice actually true), we could still potentially thence argue if it was `equal' or just in a social sense.% and therefore what value is formal equality anyway?

Yet perhaps the focus should turn away from considering technicalities and analytic considerations to the much more vague notion of degrees of inequalities between different kinds of rule-sets and treatments that people can be said to be effectively subject to \textit{in-practice}.
perhaps in answer to Anatole France's quote, that a lawful system that provided similar impediments to the rich obtaining shelter, money and sustenance would respectively be more equal.

%In considering equality before the law, a central component to be gleamed is that is something that is valued, it is probably better that is is amenable to be valued by people, and hence publicly disclosed; which is essential component.
%rather than simply that all should be subject to the same rules, but that those rules should be publicly disclosed. , and that these exceptions are not encoded into public Law.

A particular of formal equality is that it should not be directly partial to specific and particular individuals / groups, is a very low bar. But above that, it is perhaps better to characterise a system not by whether or not it is impartial, but rather by what characteristics it is (and is not) partial to.
And hence the question then turns to being primarily about what characteristics should be considered relevant for differential treatment, and henceforth what differential treatment should result.

The particular take-away from considering formal equality and equalty before the law, is that whatever principles of allocation or consequence that are endorsed, that they should be:
\begin{itemize}
    \item not \textit{partial} to specific persons and groups - atleast not directly.
    \item impose differential treatment only in-light-of morally defensible considerations.
    \item considering and/or minimising the difference in treatment and processes that is imposed upon people/s \textit{in practice}.
 %   \item publicly seen to be impartial by these factors.
\end{itemize}

The particular implications we make for Electricity systems by these principles are:
\begin{itemize}
    \item That electricity mechanisms we design should not directly be partial or weighted towards or away from particular participants.
    \item That electricity mechanisms we design should be intensioned to have singular rule-set for the management of small and large consumers and generators.
    \item 
\end{itemize}

particularly some electricity systems dont scale, and exclude individual or small suppliers/generators/consumers for equal consideration.






\subsubsection{Equality of Status}



While for others the term encodes the positive hope of a society, free of abusive power relations that perpetuate social injustices.\footnote{see next section}








Particularly there is a distinction to be made between equal treatment, and a more general `moral equality'.
Consider the quote from Christopher Nathan:

\begin{displayquote}
The distinction between ``equal treatment'' and ``treatment as equals'' expresses this difference between offering people the same treatment, and acting in accordance with the fact that they are moral equals. Equal status does not constrain us to a set of identical actions regardless of our differences.\cite{whatisbasicequalitynathan}
\end{displayquote}

The question then becomes how interpret moral equality.

Perhaps the better thing is to ask what moral inequlity would look like between peoples.
Are people morally unequal, are some \textit{worth more} than others?

\begin{displayquote}
The essential thing, however, in a good and healthy aristocracy is that it should not regard itself as a function either of the kingship or the commonwealth, but as the SIGNIFICANCE and highest justification thereof--that it should therefore accept with a good conscience the sacrifice of a legion of individuals, who, FOR ITS SAKE, must be suppressed and reduced to imperfect men, to slaves and instruments. Its fundamental belief must be precisely that society is NOT allowed to exist for its own sake, but only as a foundation and scaffolding, by means of which a select class of beings may be able to elevate themselves to their higher duties, and in general to a higher EXISTENCE.
\cite{NietzscheGutenberg} Neitzsche, Beyond Good and Evil, Ch 9
\end{displayquote}


\begin{displayquote}
I want to emphasise what is, on my view, the most important object of egalitarian distribution, and that is \textit{power}. Of course power is not something which can be parcelled up and shared out like a commodity but we can properly talk of `the distribution of power' and this is, more than anything, the determinant of whether a community is authentically cooperative.\cite{TheSocialBasisofEquality:1998}
\end{displayquote}


Perhaps as most directly evident in various feminist literature which give analyses of (eg. see \cite{Cudd2006-CUDAO}) the relationship of men\&women in relation to oppressive dominance/submission: eg. MacKinnon writes ``difference is the velvet glove on the iron fist of domination. The problem is not that differences are not valued; the problem is that they are defined by power''\cite{mackinnon1989toward}. Though not all feminists view power so negatively, as some view it as uneven distribution of positive (or potentially neutral) empowerment \cite{doi:10.1111/j.1527-2001.1998.tb01350.x}. Furthermore not all egalitarians about power view it specifically in relation to gender, see Richard Norman's egalitarian position - that disparities of power drive coercion\&exploitation and prohibit a society from being `genuinely cooperative'


\subsubsection{Other Equalities}

Some alleged motivations for the adoption of various political positions on equality are that they extend from humanitarian concerns about the suffering of the poor (in some sense defined), and also that motivation can allegedly extend from anger about the gratuitously rich\footnote{for instance, George Orwell's, The road to wigan pier, chapter 11: ``Though seldom giving much evidence of affection for the exploited, he [a socialist] is perfectly capable of displaying hatred - a sort of queer, theoretical, in vacua hatred - against the exploiters. ... It is strange how easily almost any Socialist writer can lash himself into frenzies of rage against the class to which, by birth or by adoption, he himself invariably belongs.'' }.
And while these allegations may be true or false, neither the provision of resources for the poor (such as per meeting a level of sufficiency, or as a matter of priority) or curtailment of the wealth of the rich (sometimes called `quasi-egalitarianism') entail any direct equality of any measure \textit{per sei}.



\subsubsection{Fitness and Utility}

\begin{displayquote}
\begin{tabular}{ll}
\multicolumn{2}{l}{[reaches for the coffee pot with the regular coffee,} \\
\multicolumn{2}{l}{\-\hspace{5mm}and starts pouring it into the pot with decaffeinated coffee]} \\
Derek Philby:  & Excuse me, what are you doing?\\
Monk:  & Oh - um... just making them even.\\
Derek Philby:  & But you're mixing the regular with the decaf!\\
Monk:  & But they're even.\\
Derek Philby:  & But they're mixed together!\\
Monk:  & But they're - they're even.\\
Derek Philby:  & But they're mixed together.\\
Monk:  & But they're even...\\
\end{tabular}\\
\vspace{-0.5mm}\\
\null\hfill\textit{Monk, Season 2, Mr. Monk Goes Back to School}
\end{displayquote}

A most commonly discussed (and quite general) contention is over the `levelling down' objection, which prompts the consideration of when an hypothetical equality society is better/worse than a hypothetical unequal society which is better off (see parfit).

Conversely there are general arguments that can be made for equality itself as opposed to other modes of distribution, such as arguing for equality in light of `equal moral worth'.
Some other modes of distribution can be contrast-with or overlapping with utilitarianism, prioritarianism, sufficientarianism.

For instance, utilitarianism is commonly articulated as a moral philosophy for the maximising of the sum of social welfare (or utility) in a given circumstance (or across them), and it might coincide (or not) that equal allocation of a particular thing is also maximising of its sum.
Prioritarianism is the idea that the less-well-off should have priority in allocation (perhaps even independent of their utility gains), such as might be encoded in Rawl's difference principle: the idea that any particular inequality is permitted insofar as the worse-off member of society is better-off than he/she would be otherwise.
sufficientarianism is the position such that each person should have some `sufficient' level of the allocated quality - however decided.

In society it has been recognised that some degree of inequality is inevitable, and perhaps evern preferable, one such articulation is Rawles `Difference Principle' - the idea that particular inequality is permitted insofar as the wors-off member of society is better-off than he/she would be otherwise.
perfect equality is not desirable if everyone is therefore equally miserable.

Indeed these approaches are hardly exhaustive, and the result of the considerations of which one is appropriate may be highly context dependant. (people should have food of sufficient nutrition, public housing should be allocated as priority to thoes in most need of it. government subsidies should maximise the total welfare of the nation etc.) 


\subsubsection{Deservedness and Fair Trading}



\subsubsection{Freedom from negative Externalities}

Alternatively that people should not be able to force an individual out, but potentially that they should be able to help each other out.
that the system should be responsive to positive external preferences, but not negative external preferences.
This has been argued in the context of an ethical utilitarianism (see \cite{kymlicka2002contemporary})

\subsubsection{Reward above Extortion}

One primary way which people conceive of inequality is in the context of extortion. Indeed many Marxists viewed the inequality of power between classes in-terms of being maintained by exploitation and extortion; And it is without doubt a measure by which balances of power can be maintained. Additionally people tend to associate equality with a social order in which people are compensated for their contributions.

However what is viewed by some as a matter of exploitation

In the context of directly to trading between parties.

but what exactly is extortion? 

https://plato.stanford.edu/entries/exploitation/


Consider the following very simple game: Two people are brought to a table, player 1 \& player 2. player 1 has a binary choice to make between outcomes A or B, and depending on which choice is made, an amount of money is given to both players.
However before player 1 makes a choice, player 2 has the option to enter into a binding promise to transfer some of the money he will be given to player 1 depending on his choice.

consider:

\begin{table}[h!]
\begin{tabular}{lll}
Player 1's actions: & A   & B   \\
Rewards Given:      & \$0,\$10 & \$1,\$0
\end{tabular}
\caption{a table}
\end{table}

Supposing Player 1 chooses action A, how much money should an ethical player 2 transfer to player 1 for choosing action A?
Obviously player 2 would need to transfer \$1 atleast to make it even worthwhile for player 1 having bothering to have chosen action A, but what above that?

Suppose we adopted a policy of splitting the difference of the remaining money... that might be conceived of as being rather fair and compensatory. ie. If players 1 and 2 ended up with a financial gain of \$5.50 and \$4.50.\\

However the morality of this transfer (of \$5.50 from player 2 to player 1) depends primarily on the normative attitudes which are attached to the actions A and B.\\

If it is normative or expected that player 1 would choose B (say, by rational self interest), then his/her choosing A is exceptional and results in a relative gain of \$10 for player 2, and the resulting monetary transfer can be said to represent a reward.
If conversely it is normative or expected that player 1 would choose A (say, by social mores), then player 1's choosing B is exceptional and results in a relative loss of \$10 for player 2, and the monetary transfer for not executing B can be said to represent an extorting transaction.

In the first case a `split-the difference' principle is viewed as providing a fair compensation, but in the second the principle is viewed as being extortion.
And the difference between them is in-part due to normative and/or expected actions with resultant losses and gains.

How then are particular action or imputation to be expected.

I submit the thesis that this difference between the two is in-part captured by viewing the actions as positive or negative. where in the second case player 2 is being compensated for not engaging in a positive action, but in the first case is being compensated for not engaging in a negative action.

what defines a positive action over and above a negative action?
primarily it is somewhat vague in practice, but positive actions are generally characterised as being: effortfull, deliberate and/or against the normal and default state of affairs. whereas negative actions are not so.
However from a strictly mathematical perspective - where this kind of lucid distinction is absent - both compensation and extortion are actually similar.

In an electricity system, the question of what should be considered (or would likely be viewed) as default/normal state of affairs may be a bit difficult to characterise.
But more generally, the supplying of electricity (as opposed to the non-supply of electricity) may be viewed as an `effortfull' postive action - even if it is done automatically by a machine that is totally indifferent.

A practical example of this is that compensating a company for not positively oversupplying electricity to the grid is an example of extortion, whereas companies which can drop their power generation quickly to limit oversupply on the network are credited with compensation. (and this can be seen as a double standard, and if not- then what precisely should distinguish these cases? is one action above the other to be considered to be positive?)

This general non-exploitation clause, can be viewed as being in-some ways similar to `individual rationality' in mechanism design. but is extremely context sensitive.
It can be directly mathematically specified (as we will get to in bargaining mathematizations) or perhaps more implicit.




\subsubsection{junk junk}



Although these divisions seem a bit vague and fussy, a better division between peoples conceptions of opportunity can be seen, by viewing particular opportunities as being positive or negative freedoms.\footnote{this connection is not made by me, see works by authors A and B.}
Although admittedly vague, the difference between a positive and negative freedom has distinct philosophical treatment (atleast as far back as Kant \cite{}), and has known moral importance in several contexts.
A negative freedom is the absence of specifically identifiable obstacles, barriers, or coercion about executing a particular action; it is more directly about external material circumstances and the absence of particular things.
By contrast, notions of positive freedom tend to be more expansive, and can relate to possible choices and control that might be realized by a person to bring about their purposes in life; it can touch on what is internal and mental, and seems to be more about the presence of agency.

In the context of equality of opportunity, the negative freedom of being externally unhindered about conducting an action might be considered as a prerequisite to the positive freedom of actually having a choice about its execution.
And hence a particular equality of opportunity which is conceived negatively has a  potentially wider base of appeal than its positive counterpart.

So for instance, in the context of opportunity for educational attainment, the difference between opportunity concieved more negatively (equal freedom from specific obstructions to attainment - eg. class structure, racial discrimination, etc) is different from the equality of opportunity conceived more positively (having equal propensity in choice to attain).

The difference between positive and negative freedoms is vague but can be seen to have significant moral relevance, particularly in the consideration of cases about doing vs allowing harm to another person.
The distinction of moral relevance between doing vs allowing harm is one of the cornerstone cases against consequentialist morality; and is encoded in most legal systems.
For instance, various thought experiments exist asking what the moral and legal difference should be between drowning another person, and neglecting to save another person from drowning; or between active and passive euthenasia.
If a person has a right to negative freedom about a particular domain, then that counts as a right to be free of others doing harm to them in that context (as that would count as an introduction of a specifically identifiable obstacle).
%If a person has a right to be free of others' doing harm to them, then that counts as a negative freedom right from such external harm.
Whereas If a person has a right to positive freedom about a particular domain, then that would seem to count as a right to be free of others even allowing harm to them in that context (an example of a positive freedom might be to live free of absolute poverty).

Some of the vagueness between positive and negative freedoms can be resolved by considering them as being about an action/s, above a set of factors.
If the set of factors is very specific and narrow - such as above racial discrimination or physical disability - then it characteristically more negative.
If the set of factors is very expansive - such as above anything that might impinge apon a person's agency at all - then it is characteristically more positive.
This can be seen in Gerald MacCallum'm presentation (1967) (stanford encyclopedia's) of freedoms as triadic relations. Specifcally that a liberty or freedom can be considered as a triadic relation between an agent, an action and preventing conditions.

Particularly the context about people's having negative freedom from other's causing harm brings to the fore the distinction between positive and negative actions.
Where causing harm is concieved as a positive action whereas allowing harm is concieved as negative action.
The vague distinction between positive and negative actions, is that positive actions are considered to be intrinsically `effortfull' whereas negative actions are not nessisarily so.\cite{Mossel2009} And to some extent it also links up with notions of positive and negative rights, where negative rights are to be free of positive adverse actions of others, and positive rights mandate positive beneficial actions of others.
%positive and negative actions have also connection in terms of the commision and/or ommission of actions - although the idea is vague.

In anycase, more people support equalities of opportunity that are more negative in flavour - that people should be able to persue choice of life agasint arbitrary injunctions (especially where they are conceived as positive actions of others), 
much more than are for equality of opportunity concieved more positively - that each should be positively supported to equally be able to persue choice of life -period.
In part simply because the latter tend to be stronger than the former.


The more agreed-apon position of, specific negative freedom of opportunity, is the philosophical take-away.
In sections \ref{} and \ref{} we consider particular negative freedom opportunities, particularly the goals of an electricity system that it should aim to not exclude participants arbitrarily (or without specific warranted justification), and also that it should specifically exclude negative external utilities.
This distinction between postive and negative rights will also be given greater treatment in chapter \ref{} as there it is taken to be a core defining difference between financial transactions of compensation and extortion.







\subsubsection{Compensation for Contribution}

Particularly as best and directly encoded by VCG imputations.

\subsubsection{Freedom from group Exploitation}

Ie that imputations should belong to the Core of the cooperative game.
The shapley value is compromise between this and marginal contribution payments.
best embodied in some of the work of Marxist philosophers such as John Roemer. (citations)

\begin{displayquote}
This is the main aim of John Roemer's work on exploitation. He defines Marxist exploitation, not in terms of surplus transfer, but in terms of unequal access to the means of production. Whether one is exploited or not, on his view, depends on whether one would be better off in a hypothetical situation of distributive equality -- namely, where one withdraw with one's labour and per capita share of external resources. If we view the different groups in the economy as players in a game whose rules are defined by existing property-relations then a group is exploited if its members would do better if they stopped playing the game, and withdrew their per capita share of external resources and started playing their own game.\cite{kymlicka2002contemporary}
\end{displayquote}


The idea specifically of allocating with reference to exceeding what any group could achieve if they withdrew to cooperate among themselves, is most directly articulated and formalised by the Core concept of cooperative game theory.
The Core is a cooperative game is a set of allocations to individuals such that no group of them could gain more by-themselves, and is outlined and defined in our Chapter \ref{}.


\subsubsection{Free from envy}

There is much discussion and topicality about envy free-distributions.
particularly that nobody would prefer what other people are allocated.
While envy freeness, is a difficult criterion to satisfy in general, there is some attempts to systematize envy-freeness into distribution systems.
Particularly the work of Dworkin. (cite,cite,cite)







\subsection{Summary}
Summary what you discussed in this chapter, and mention the story in next
chapter. Readers should roughly understand what your thesis takes about by only reading
words at the beginning and the end (Summary) of each chapter.




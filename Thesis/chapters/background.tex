\section{Conceptual Background}
\label{cha:background}
%At the begging of each chapter, please introduce the motivation and high-level
%picture of the chapter. You also have to introduce sections in the
%chapter. \\

The electricity grid is seen as an evolving system of increasingly interconnected and complex devices, providing a structured platform for the exchange of power and services between people.
However at the heart of this evolving construction there is a morally ambiguous question - how should it be determined how much people \textit{should} pay or be-paid for the power they consume or generate?

Or, to ask the question with more granularity: when the possible power-flows on an electricity network are valued and influenced by participants differently, 
what is a reasonable electrical outcome for the network and what monetary transactions should occur between the participants in that case? -- and this is to ask: how \textit{should} electricity be traded?\\

%Our development relates to mechanisms which are intentioned to be practical, and a brief elucidation of some economic and game-theory concepts will ultimately be involved. However we begin with the ethical considerations.

\subsection{A Philosophical Prelude}

%\begin{displayquote}
%My claim is merely that there is no single fundamental principle that determines or provides guidance on what justice requires in relation to the distribution of access to overall advantage.\cite{mason2006levelling} 
%\end{displayquote}

In considering the question of what distribution energy market structure \textit{should} be implemented, it is essential to atleast acknowledge that there exists a wide range of moral and practical factors that bear on the question.

In this chapter we quickly gloss over some of the moral considerations that frame the question, however in doing so, we must make clear that the deliberate brevity of this underdeveloped chapter it not to suggest that these considerations are not important and worthy of much greater or proper treatment, but only that we are not trained philosophers per sei, and out of respect and modestly we do our best to keep to our own turf, while acknowledging the situtation.

%However these considerations and concepts are crucial and range much wider than the specific context of electricity markets, though unfortunately we cannot give a comprehensive treatment of them here.

The moral and ethical side of our question is broadly associated with a branch of moral philosophy called Distributive Justice, which seeks to ask and make headway on the question of how different kinds of resources (such as money/power/goods/etc) \textit{should} be distributed in society; and this is the much broader and central ethical question.

We begin by noting there is a long history of philosophical skepticism about the nature of moral knowledge and judgements, and Distributive Justice is not exceptional in this regard.
For specific hisorical arguments, see Hume's Guillotine \cite{HumeGutenberg}\footnote{``For as this ought, or ought not, expresses some new relation or affirmation, it is necessary that it should be observed and explained; and at the same time that a reason should be given, for what seems altogether inconceivable, how this new relation can be a deduction from others, which are entirely different from it. But as authors do not commonly use this precaution, I shall presume to recommend it to the readers; and am persuaded, that this small attention would subvert all the vulgar systems of morality, and let us see, that the distinction of vice and virtue is not founded merely on the relations of objects, nor is perceived by reason.'' T3.1.1} which is often read as stating that no material facts about how the physical world is, could ever by-itself seem to logically imply any claim about how the world (or its components) \textit{should} be.
Another example is G.E. Moore's open-question argument \cite{MooreGutenberg}\footnote{
``Moreover any one can easily convince himself by inspection that the predicate of this proposition - `good' - is positively different from the notion of `desiring to desire' which enters into its subject: `That we should desire to desire A is good' is not merely equivalent to `That A should be good is good.' ... clearly that we have two different notions before our minds.''Ch1:13\\
``If I am asked `What is good?' my answer is that good is good, and that is the end of the matter. Or if I am asked `How is good to be defined?' my answer is that it cannot be defined, and that is all I have to say about it.'' Ch 1:6}, which argues that for anything which defines what is morally good, then a question-about or statement-of that equivalence would only be tautology.
Such arguments are probably best used as discussion-starters today, however, talking about the ontological nature and the basis of moral knowledge is not our focus.
%Whether or not moral truth ultimately amounts to collective sentiment, or reduces to  statements of prudence, or is metaphysically identical/grounded in some deeper objective fact is beyond the scope of this work.
Instead, our focus tends towards some discussion around the moral views that people are likely to have upon reflection; and we give an extremely briefly survey some of the attitudes expressed in literature where relevant.

What is quite evident, is that different people have different conceptions of how the world should be, and not all of these conceptions are compatible with each other.
That any particular ethical system is likely to be rooted in a specific focus (as encoded by principles, maxims, cultural narrative, language etc) and will yeild outcomes that may be disagreeable to some people and agreeable to others.
%In constructing any particular answer to the question of what counts as a good/better/fair mechanism we will necessarily isolate those who would disagree.
%And for this we must make a modest apology.

%We could just directly constructing an answer to the question \textit{de novo}, but such an approach would miss a lot of humility and leave open the question why those elements should be selected. And so the situation leads us to give at least some consideration the space of peoples moral intuitions before settling down upon a judgment.

We attempt to give a brief survey to address what we believe are often some of the elements that feature in people's moral thinking, and do our best to develop a novel synthesis about electricity systems, which bears some relevance to these moral considerations.

While we must acknowledge the moral ambiguity inherent in the question of electricity allocation, we contend that this does not mean that any answer is simply \textit{as good} as any other. But only that we believe that the suitability of our answer is not something we can totally demonstrate, in principle.

Let us begin.

\subsection{Some Moral Factors}

The choice of centralized Market structures and process can be seen as a choice between methods of allocating resources between multiple parties in a system based on the parties interaction within it.
An example of a market structure might be a type of auction, and the interractions of the parties might be their choices of bidding or bidding strategy.
In this context the choice of the auction and also the resultant likely distribution of resources can be viewed as being morally/socially desirable or undesirable bassed on a number of factors.
What constitutes a desirable distribution of resources?

%In the process of discussing a choice of system it is important to come to a relatively clear understanding of what concepts are in play. And in these sections we will attempt to briefly survey and break-down some of the surrounding moral concepts.

Throughout time there have been an array of philosophers who have discussed ideas surrounding the moral distribution of resources and capital. And one of the major ideas surrounding the ethics of distribution is \textit{Equality}.

\subsubsection{Equality}

\begin{displayquote}
``A common characteristic of virtually all the approaches to the ethics of social arrangements that have stood the test of time is to want equality of \textit{something}... They are all `egalitarians' in some essential way ... To see the battle as one between thoes `in favor of' and thoes `against' equality (as the problem is often posed in the literature) is to miss something central to the subject."\cite{18084} 
\end{displayquote}

\begin{displayquote}
``for all men have some natural inclination to justice ... what is equal appears just, and is so; but not to all; only among those who are equals: and what is unequal appears just, and is so; but not to all, only amongst those who are unequals;\\
which circumstance some people neglect, and therefore judge ill; the reason for which is, they judge for themselves, and every one almost is the worst judge in his own cause." Aristottle, Politics, chapter III.9\cite{AristotleGutenberg}
\end{displayquote}


People tend to believe that they are, should be, or be treated, `equal' in some sense.
And this broad conception has changed throughout time and place in history \cite{themeaningofequalitycapaldi}.
From at least as far back as Aristotle \cite{AristotleGutenberg}\footnote{see section quote}, notions and concepts about equality have come from across culture and peoples, and between Spiritual \footnote{across multiple religions, eg. in Islam ``No Arab is superior to a non-Arab, no colored person to a  white person, or a white person to a colored person except by Taqwa (piety)." [Ahmad and At-Tirmithi], and in Christianity, St Paul's Galatians 3:28 ``There is neither Jew nor Gentile, neither slave nor free, nor is there male and female, for you are all one in Christ Jesus'' (NIV) } and the Materialist \footnote{Such as in Engel's Anti-D\"{u}hring Part 1 Chapter 10 ``The idea that all men, as men, have something in common, and that to that extent they are equal, is of course primeval. But the modern demand for equality is something entirely different from that; this consists rather in deducing from that common quality of being human, from that equality of men as men, a claim to equal political social status for all human beings''} thought.
Throughout the ages the way in which equality in society has been constructed and implemented has varied dramatically - even the abolition of slavery is recent by comparrison.

There is something appealing about the idea of Equality between people.
From the asthetic perspective equality is an ideal with a simple structure. From a humanitarian perspective equality is associated with relief from envy and want. From the social perspective it is associated with community and solidarity.% From the philosophical perspective equality is directly associated with fundamental and core ideas.

%What is morality and justice? and what Rights and Freedoms does it require? and are people actually equal in any morally relevant way?

On a practical level, the divergences between people's ideas of equality can be seen as regarding what things should-be equal (when, where and for whom); and also what should be done about inequalities as may exist.

The question: ``Equality of what?'' can have many answers, some of which are commonly held and seldom controversial today, such as might be gleamed from the United Nation's declaration of Human Rights: Equality before the Law, Democratic Equality to vote, equal freedoms to marry and to live, etc.\cite{udhr}
However more controversial answers tend to have broader social and political scope, such as: Equality of Opportunity, and Equality of Welfare and/or Economic Equality.% And the controversial nature of these ideas notably come to the fore in discussions such as surrounding affirmative action initiatives and also political socialism.

For some, equality is a contestable notion, or an ideal for the direction of efforts in narrow and specific contexts, but for others equality is an attainable and far-reaching goal with multifaceted implications across social spheres \cite{walzer2008spheres,millerandwalzer,baker1992arguing}.

There are many ways in which equality of specific measures has been argued; particularly, a person can advocate for equality in a given context \textit{directly} and/or \textit{instrumentally}.
%For instance, as a practical measure in some settings, and also as an extension of more fundamental moral principles.
What amounts to an equal allocation can also instrumentally satisfy other values, for instance, David Miller \cite{equalityandjustice:1998} argues for the practicality of giving equal remuneration between hypothetical employees in the context of uncertainty about how much each of them deserved (and/or lack of means about distinguishing them). And also gives some broader examples of reasons for equalities in society: for asthetic and pro-social reasons, because it can be a sufficiently practical and simple social contract, or because it might be politically inevitable. etc.
These (and others) may count as instrumental or prudent reasons to value the implementation of an equality in a specific context, but in other contexts these same reasons could potentially point to other arrangements.
However, there have been arguments directly for equality of measures, particularly from (or in light of) more abstract concepts such the notion that people have \textit{equal moral worth} \cite{doallpersonshaveequalmoralworth} or \textit{moral equality}. Though it is difficult to define\footnote{for instance, Discussion about \textit{who} has equal moral worth (or alternatively \textit{how/why} they do) seems to occasionally to turn into a discussion about the moral rights of animals}, the notion that people have equal moral worth is felt not to logically imply any very specific kind of equality of measure per sei.

\begin{displayquote}
The distinction between ``equal treatment'' and ``treatment as equals'' expresses this difference between offering people the same treatment, and acting in accordance with the fact that they are moral equals. Equal status does not constrain us to a set of identical actions regardless of our differences.\cite{whatisbasicequalitynathan}
\end{displayquote}

The question about what things should be equal (rights, freedoms, duties, responsibilities etc, and for whom and when) can be seen as forming a large component of the various moral systems. And it is sometimes felt that moral equality simply cannot be a logical premiss for these questions.

\begin{displayquote}
The idea of moral equality, while fundamental, is too abstract to serve as a premise from which we deduce a theory of justice. What we have in political argument is not a single premise and then competing deductions, but rather a single concept and then competing conceptions or interpretations of it. Each theory of justice is not \textit{deduced from} the ideal of equality, but rather \textit{aspires to} it, and each theory can be judged by how well it succeeds in that aspiration.\cite{kymlicka2002contemporary}
\end{displayquote}

In anycase, expecting a particular person to give a precisely defined answer to the question ``Equality of what?'' may be asking too much; as even the phrases which people use in everyday life are seldom given exact specifications\footnote{Degrees of vagueness are well witnessed in everyday sentences, ``There are a gathering of people near that tree'', such as argued in the classic Sorties paradox\cite{frances_2018}}, let alone concepts pertaining to the spectra of possible societies.
So, for instance, the space of various contemporary political philosophies which faithfully attempt to construct and interpret some reasonable form of equality between persons has been described as belonging to an `egalitarian plateau'.\footnote{The phrase is originally attributed to Dworkin and subsequently adopted by others.}\cite{Brown2007}
Or conversly, while a specific ethical equality may not be agreed apon, perhaps there may be a more broadly accepted notion of what an `inequality' looks like,
particularly as it is sometimes blurred with the concept of a `social injustice'.\footnote{There is some debate as to when/where/how an inequality also becomes an injustice. It is possible to believe that an inequality constitutes an injustice directly, or perhaps that an inequality is proof (or perhaps only potential evidence) of a injustice in procedure or treatment. see parfit's concept of Telic vs. Deontic Egalitarianism.\cite{equalityandpriorityparfit}}

And in this way, the concept of Equality can be inclusive-of and also contrasted-against other views; such as thoes that emphasize the priority of resources to the poor, or such as emphasise alleviation of insufficiency among the poor; broadly termed ``prioritarianism'' and ``sufficientarianism'' respectively.\cite{sep-egalitarianism}\footnote{for good measure we might also consider Rawl's Theory of Justice \cite{rawls2005theory} as a specific kind of (layered) priority principle.}

Although the concept of equality is the subject of wider analysis, we will focus on two specific interpretations of equality which we feel can be linked directly to mechanisms for electricity allocation particularly.

\subsubsection{Formal Equality}

One primary interpretation of equality is that people should be subject to systems that treat them in a manner that is \textit{impartial}. The minimal idea is that an impartial system should not afford arbitrary or unjustified special treatment toward any particular individual/s. Hence that systems should operate by rules which are blind to particular identity and sensitive only to morally relevant characteristics.

In literature this idea of moral impartiality to particular identity has been elucidated by various thought experiments and also stated with moral maxims.
Particularly famous devices include Rawl's ``Original Position'', Hare's "ideal sympathiser", Kant's categorical imperatives, etc.

Additionally the idea is perhaps mathematically expressible, in that people who are (in all relevant ways) equal should be treated equally.
This is known as formal equality \cite{whatisbasicequalitynathan}. Although formal equality is occasionally seen as being an important part of a fair system, it is also sometimes seen to be insufficient to capture broader notions of equality and justice.

\begin{displayquote}
In its majestic equality, the law forbids rich and poor alike to sleep under bridges, beg in the streets and steal loaves of bread.\\
--Anatole France, Le Lys Rouge [The Red Lily] (1894), ch. 7
\end{displayquote}

Indeed, by imagination almost any treatment or process could be rationalised as being issued by impartial principles which are universally applied. And additionally, in practice not all kinds of impartiality are mutually compatable.\cite{Hutchinson_2019}

Notwithstanding, formal equality is a basic doctrine that ascribes value to the incorporating degrees (and/or kinds) of impartiality into the design of social processes from the outset.%; even if they never truly reach perfect impartiality in practice.
% that social systems should be designed with an aim to treat people identically, but-for morally relevent characteristics.
%And thus there is perhaps some value in incorporating some degree (and/or kinds) of impartiality into the design of social processes from the outset; even if they never truly reach perfect impartiality in practice.
%in the fact that some degree (and/or kinds) of impartiality might be incorporated into the design of social processes from the outset; even if they never reach perfect impartiality in practice.
%It might also be profitable to consider the inequalities between the different kinds of rules (and treatement) that people can be effectively subject to \textit{in practice}, or effectively \textit{by consequence}.

\subsubsection{Wider Social Equality}

\begin{displayquote}
I want to emphasise what is, on my view, the most important object of egalitarian distribution, and that is \textit{power}. Of course power is not something which can be parcelled up and shared out like a commodity but we can properly talk of `the distribution of power' and this is, more than anything, the determinant of whether a community is authentically cooperative.\cite{TheSocialBasisofEquality:1998}
\end{displayquote}

There are different and interrelated ways of how to concieve of wider social equality, and one historic way of framing social equality is in terms of power.
As for some people, the ideal of equality encodes the hope of a society free of abusive power relations that perpetuate social injustices.

One of more historically notable instances of this framing is featured in Marxist thought, in the expounding the abusive economic power relations between social classes. This frame also shows up historically in feminist thought (eg. see \cite{Cudd2006-CUDAO}) in which the inequality of power between men \& women is expounded as a form of oppressive dominance \& submission \footnote{eg. MacKinnon writes ``difference is the velvet glove on the iron fist of domination. The problem is not that differences are not valued; the problem is that they are defined by power''\cite{mackinnon1989toward}.}

What is notable is that neither Marx nor feminist writers always viewed power itself as negative. For instance Marx opposed private property (as capitalistic ownership) but had a more complex attitude toward property relations generally \footnote{"the theory of the Communists may be summed up in the single sentence: Abolition of private property ... Do you mean the property of the petty artisan and of the small peasant, a form of property that preceded the bourgeois form? There is no need to abolish that"\cite{MarxGutenberg}\\"Property thus originally means no more than a human being's relation to his natural conditions of production as belonging to him, as his, as presupposed along with his own being; relations to them as natural presuppositions of his self, which only form, so to speak, his extended body."\cite{marx_capital_I}}.

Additionally not all feminists view power itself negatively, particularly in the positive (or potentially neutral) language of \textit{empowerment} \cite{doi:10.1111/j.1527-2001.1998.tb01350.x}.

One of the features of power that is associated (but not always) with abuse is `power over' other people, or `power to' do things which impinge apon other's rights.\cite{doi:10.1111/j.1527-2001.1998.tb01350.x}
But disecting when and where an exersize of power consititues an abusive or morally objectionable act is not easy.
Particularly the `power to' do something is straightforwardly an example of a freedom, and one well known dichotemy exists between \textit{positive} and \textit{negative} freedoms
\footnote{While vague, a negative freedom is associated with an absense of external obstacles to conducting the specific action, and a positive freedom is associated with the possibility (or actuality) of doing the act in accordance with one's will and purposes. The positive/negative dichotemy is also associated with what is effortfull.\cite{}} particularly in the discussion of doing vs. allowing harm.
But even more broadly, freedoms can be considered as triadic relationships: a freedom \textit{of} a person, \textit{from} particular preventing conditions, \textit{to} do certain things.\cite{Negative_and_Positive_Freedom}

However different freedoms are not equally valued (or compatable), and some are esteemed by individuals and societies more than others.
Some would place a greater importance of political freedoms (to openly discuss, vote, and run for office) or economic freedoms (to work, to buy, sell and lease property), etc.
But particularly, the having and actualising of freedoms associated with the meeting of needs; such as basic needs (of shelter, food, etc - as at the botton of Maslow's heirarchy) as well as higher needs (such as social belonging and self-actualisation); can be considered as defining of human wellbeing, and perhaps even a constituent of the state of having `Freedom' - the moral and political ideal.

Unfortunately most of these (and other) wider conceptions of societal Equality are beyond the scope of what we can earnestly engineer directly, but what we can do is to reflect on the effect that any proposed system might have on the wellbeing of society and evaluate it accordingly - and this is a task we attempt in later section \ref{}.

\subsubsection{Efficiency and Utility maximisation}

\begin{displayquote}
Essentially, Utilitarianism sees persons as locations of their respective utilities %- as sites at which such activities as desiring and having pleasure and pain take place.
 ... Persons do not count as individuals in this any more than individual petrol tanks do in the analysis of the national consumption of petrolium.
\cite{}
\end{displayquote}

Different Equality doctrines are often associated with (or defined by) specific quantities, and a more equal allocation of these quantities may not be possible or even desirable.
While it may-not be possible to comprehensively quantify people's wellbeing, in particular contexts a person's wellbeing might be measured or associated with particular quantites - such as: cost of living, income, number of friends, wealth, educational attainment etc.
And one particularly objection to some of the various Equalities is the famous 'leveling-down objection'; loosely speaking, the objection is that the strict Egalitarian would (if constrained in options) prefer a world in which every person had less, if it were more equal. \cite{temkin_2003, equalityandpriorityparfit}.

This objection frames one possible articulartion of the broader contrast between the values of Equality and 'Efficiency'; and the specific concept of efficiency that is meant here is Pareto optimality.
Particularly an outcome is not pareto optimal if there exists another outome which is prefered by everbody.
Pareto optimality is one commonly discussed efficiency condition, and is a satisfiable property in our subsequent developments (per Chapter \ref{}).

However Pareto optimality is not the only measure of efficiency, and another famous measure is the object of philosophical utilitarianism, particularly the sum of utility.
The historical concept of utility has changed over time\footnote{most historically held as introduced by Mills and Bentham\cite{} \cite{}} but minimally it is a measure of the strength of the preference (or value) that a person does (or should, rationally, pragmatically and/or morally) attach to different possible outcomes.
The concept extends from the consideration that preferences should be transitive and comparable.

Particularly if a person prefers A to B, and also B to C, then they ought to also prefer A to C in the same manner. And additionally if there is a consistent ordering over the strength of these preferences between people - if one person prefers A more than another prefers B - (where `more' has some definition, such as intensity of feeling or moral salience) then it remains a task of invention to associate numbers to the strength of these preferences over outcomes.

The concept of utility remains contentious and the political philosophy of utilitarianism has a long history of being even more so.
Particular objections cast doubt on whether morally relevent preferences are infact comparable (particularly along a single dimension) and how/where is utility supposed to be measured/revealed, and any normative implications these utility measures should have.
Despite this, the object of philosophical utilitarianism is the maximising of utility, and there are different variants, but a most simplistic variant is about maximising the sum of utility.

The sum of utility is an example of a efficiency measure, and in many contexts it is relatively free of ambiguity - and might be associated (or measured by) various things, such as educational performance, happiness or expected lifespan.
In the context of our electricity markets it is associated with monetary yields, and the maximising of the sum of monetary yeilds is an axiom in our treatment (in Chapter \ref{}).





Although the concept of utility has several detractors, in practice it is difficult to make very definite statements simply from qualitatie judgements, particularly in context of resources (such as money) which are inherantly quantitative.

The primary virtue of quantifying welfare, allows us to numerically compare different schemes.


One feature of utilitarian approaches is that it satisfies Pareto optimality.


and there is minimax outcome selection which alocates the outcome that maximises the utility of the least-well-off.
- a particularly well known example of this doctrine is Rawl's difference principle.

Additionally there are weighted schemas that draw the path between these approaches and others.

 


is that the best outcome that is quantitatively equal might not be Pareto optimal.
At a broader level and in practice, the difference between sensible outcomes which are better or best for more people, and what is more equal for them, can be the subject of dispute (in the medical field, access to welfare, and wealth).







Some other modes of distribution can be contrast-with or overlapping with utilitarianism, prioritarianism, sufficientarianism.

For instance, utilitarianism is commonly articulated as a moral philosophy for the maximising of the sum of social welfare (or utility) in a given circumstance (or across them), and it might coincide (or not) that equal allocation of a particular thing is also maximising of its sum.
Prioritarianism is the idea that the less-well-off should have priority in allocation (perhaps even independent of their utility gains), such as might be encoded in Rawl's difference principle: the idea that any particular inequality is permitted insofar as the worse-off member of society is better-off than he/she would be otherwise.
sufficientarianism is the position such that each person should have some `sufficient' level of the allocated quality - however decided.

In society it has been recognised that some degree of inequality is inevitable, and perhaps evern preferable, one such articulation is Rawles `Difference Principle' - the idea that particular inequality is permitted insofar as the wors-off member of society is better-off than he/she would be otherwise.
perfect equality is not desirable if everyone is therefore equally miserable.

Indeed these approaches are hardly exhaustive, and the result of the considerations of which one is appropriate may be highly context dependant. (people should have food of sufficient nutrition, public housing should be allocated as priority to thoes in most need of it. government subsidies should maximise the total welfare of the nation etc.) 



\subsubsection{Other Equalities}

Some alleged motivations for the adoption of various political positions on equality are that they extend from humanitarian concerns about the suffering of the poor (in some sense defined), and also that motivation can allegedly extend from anger about the gratuitously rich\footnote{for instance, George Orwell's, The road to wigan pier, chapter 11: ``Though seldom giving much evidence of affection for the exploited, he [a socialist] is perfectly capable of displaying hatred - a sort of queer, theoretical, in vacua hatred - against the exploiters. ... It is strange how easily almost any Socialist writer can lash himself into frenzies of rage against the class to which, by birth or by adoption, he himself invariably belongs.'' }.
And while these allegations may be true or false, neither the provision of resources for the poor (such as per meeting a level of sufficiency, or as a matter of priority) or curtailment of the wealth of the rich (sometimes called `quasi-egalitarianism') entail any direct equality of any measure \textit{per sei}.


\subsubsection{Deservedness and Fair Trading}



\subsubsection{Freedom from negative Externalities}

Alternatively that people should not be able to force an individual out, but potentially that they should be able to help each other out.
that the system should be responsive to positive external preferences, but not negative external preferences.
This has been argued in the context of an ethical utilitarianism (see \cite{kymlicka2002contemporary})

\subsubsection{Reward above Extortion}

One primary way which people conceive of inequality is in the context of extortion. Indeed many Marxists viewed the inequality of power between classes in-terms of being maintained by exploitation and extortion; And it is without doubt a measure by which balances of power can be maintained. Additionally people tend to associate equality with a social order in which people are compensated for their contributions.

However what is viewed by some as a matter of exploitation

In the context of directly to trading between parties.

but what exactly is extortion? 

https://plato.stanford.edu/entries/exploitation/


Consider the following very simple game: Two people are brought to a table, player 1 \& player 2. player 1 has a binary choice to make between outcomes A or B, and depending on which choice is made, an amount of money is given to both players.
However before player 1 makes a choice, player 2 has the option to enter into a binding promise to transfer some of the money he will be given to player 1 depending on his choice.

consider:

\begin{table}[h!]
\begin{tabular}{lll}
Player 1's actions: & A   & B   \\
Rewards Given:      & \$0,\$10 & \$1,\$0
\end{tabular}
\caption{a table}
\end{table}

Supposing Player 1 chooses action A, how much money should an ethical player 2 transfer to player 1 for choosing action A?
Obviously player 2 would need to transfer \$1 atleast to make it even worthwhile for player 1 having bothering to have chosen action A, but what above that?

Suppose we adopted a policy of splitting the difference of the remaining money... that might be conceived of as being rather fair and compensatory. ie. If players 1 and 2 ended up with a financial gain of \$5.50 and \$4.50.\\

However the morality of this transfer (of \$5.50 from player 2 to player 1) depends primarily on the normative attitudes which are attached to the actions A and B.\\

If it is normative or expected that player 1 would choose B (say, by rational self interest), then his/her choosing A is exceptional and results in a relative gain of \$10 for player 2, and the resulting monetary transfer can be said to represent a reward.
If conversely it is normative or expected that player 1 would choose A (say, by social mores), then player 1's choosing B is exceptional and results in a relative loss of \$10 for player 2, and the monetary transfer for not executing B can be said to represent an extorting transaction.

In the first case a `split-the difference' principle is viewed as providing a fair compensation, but in the second the principle is viewed as being extortion.
And the difference between them is in-part due to normative and/or expected actions with resultant losses and gains.

How then are particular action or imputation to be expected.

I submit the thesis that this difference between the two is in-part captured by viewing the actions as positive or negative. where in the second case player 2 is being compensated for not engaging in a positive action, but in the first case is being compensated for not engaging in a negative action.

what defines a positive action over and above a negative action?
primarily it is somewhat vague in practice, but positive actions are generally characterised as being: effortfull, deliberate and/or against the normal and default state of affairs. whereas negative actions are not so.
However from a strictly mathematical perspective - where this kind of lucid distinction is absent - both compensation and extortion are actually similar.

In an electricity system, the question of what should be considered (or would likely be viewed) as default/normal state of affairs may be a bit difficult to characterise.
But more generally, the supplying of electricity (as opposed to the non-supply of electricity) may be viewed as an `effortfull' postive action - even if it is done automatically by a machine that is totally indifferent.

A practical example of this is that compensating a company for not positively oversupplying electricity to the grid is an example of extortion, whereas companies which can drop their power generation quickly to limit oversupply on the network are credited with compensation. (and this can be seen as a double standard, and if not- then what precisely should distinguish these cases? is one action above the other to be considered to be positive?)

This general non-exploitation clause, can be viewed as being in-some ways similar to `individual rationality' in mechanism design. but is extremely context sensitive.
It can be directly mathematically specified (as we will get to in bargaining mathematizations) or perhaps more implicit.




\subsubsection{junk junk}



Although these divisions seem a bit vague and fussy, a better division between peoples conceptions of opportunity can be seen, by viewing particular opportunities as being positive or negative freedoms.\footnote{this connection is not made by me, see works by authors A and B.}
Although admittedly vague, the difference between a positive and negative freedom has distinct philosophical treatment (atleast as far back as Kant \cite{}), and has known moral importance in several contexts.
A negative freedom is the absence of specifically identifiable obstacles, barriers, or coercion about executing a particular action; it is more directly about external material circumstances and the absence of particular things.
By contrast, notions of positive freedom tend to be more expansive, and can relate to possible choices and control that might be realized by a person to bring about their purposes in life; it can touch on what is internal and mental, and seems to be more about the presence of agency.

In the context of equality of opportunity, the negative freedom of being externally unhindered about conducting an action might be considered as a prerequisite to the positive freedom of actually having a choice about its execution.
And hence a particular equality of opportunity which is conceived negatively has a  potentially wider base of appeal than its positive counterpart.

So for instance, in the context of opportunity for educational attainment, the difference between opportunity concieved more negatively (equal freedom from specific obstructions to attainment - eg. class structure, racial discrimination, etc) is different from the equality of opportunity conceived more positively (having equal propensity in choice to attain).

The difference between positive and negative freedoms is vague but can be seen to have significant moral relevance, particularly in the consideration of cases about doing vs allowing harm to another person.
The distinction of moral relevance between doing vs allowing harm is one of the cornerstone cases against consequentialist morality; and is encoded in most legal systems.
For instance, various thought experiments exist asking what the moral and legal difference should be between drowning another person, and neglecting to save another person from drowning; or between active and passive euthenasia.
If a person has a right to negative freedom about a particular domain, then that counts as a right to be free of others doing harm to them in that context (as that would count as an introduction of a specifically identifiable obstacle).
%If a person has a right to be free of others' doing harm to them, then that counts as a negative freedom right from such external harm.
Whereas If a person has a right to positive freedom about a particular domain, then that would seem to count as a right to be free of others even allowing harm to them in that context (an example of a positive freedom might be to live free of absolute poverty).

Some of the vagueness between positive and negative freedoms can be resolved by considering them as being about an action/s, above a set of factors.
If the set of factors is very specific and narrow - such as above racial discrimination or physical disability - then it characteristically more negative.
If the set of factors is very expansive - such as above anything that might impinge apon a person's agency at all - then it is characteristically more positive.
This can be seen in Gerald MacCallum'm presentation (1967) (stanford encyclopedia's) of freedoms as triadic relations. Specifcally that a liberty or freedom can be considered as a triadic relation between an agent, an action and preventing conditions.

Particularly the context about people's having negative freedom from other's causing harm brings to the fore the distinction between positive and negative actions.
Where causing harm is concieved as a positive action whereas allowing harm is concieved as negative action.
The vague distinction between positive and negative actions, is that positive actions are considered to be intrinsically `effortfull' whereas negative actions are not nessisarily so.\cite{Mossel2009} And to some extent it also links up with notions of positive and negative rights, where negative rights are to be free of positive adverse actions of others, and positive rights mandate positive beneficial actions of others.
%positive and negative actions have also connection in terms of the commision and/or ommission of actions - although the idea is vague.

In anycase, more people support equalities of opportunity that are more negative in flavour - that people should be able to persue choice of life agasint arbitrary injunctions (especially where they are conceived as positive actions of others), 
much more than are for equality of opportunity concieved more positively - that each should be positively supported to equally be able to persue choice of life -period.
In part simply because the latter tend to be stronger than the former.


The more agreed-apon position of, specific negative freedom of opportunity, is the philosophical take-away.
In sections \ref{} and \ref{} we consider particular negative freedom opportunities, particularly the goals of an electricity system that it should aim to not exclude participants arbitrarily (or without specific warranted justification), and also that it should specifically exclude negative external utilities.
This distinction between postive and negative rights will also be given greater treatment in chapter \ref{} as there it is taken to be a core defining difference between financial transactions of compensation and extortion.







\subsubsection{Compensation for Contribution}

Particularly as best and directly encoded by VCG imputations.

\subsubsection{Freedom from group Exploitation}

Ie that imputations should belong to the Core of the cooperative game.
The shapley value is compromise between this and marginal contribution payments.
best embodied in some of the work of Marxist philosophers such as John Roemer. (citations)

\begin{displayquote}
This is the main aim of John Roemer's work on exploitation. He defines Marxist exploitation, not in terms of surplus transfer, but in terms of unequal access to the means of production. Whether one is exploited or not, on his view, depends on whether one would be better off in a hypothetical situation of distributive equality -- namely, where one withdraw with one's labour and per capita share of external resources. If we view the different groups in the economy as players in a game whose rules are defined by existing property-relations then a group is exploited if its members would do better if they stopped playing the game, and withdrew their per capita share of external resources and started playing their own game.\cite{kymlicka2002contemporary}
\end{displayquote}


The idea specifically of allocating with reference to exceeding what any group could achieve if they withdrew to cooperate among themselves, is most directly articulated and formalised by the Core concept of cooperative game theory.
The Core is a cooperative game is a set of allocations to individuals such that no group of them could gain more by-themselves, and is outlined and defined in our Chapter \ref{}.


\subsubsection{Free from envy}

There is much discussion and topicality about envy free-distributions.
particularly that nobody would prefer what other people are allocated.
While envy freeness, is a difficult criterion to satisfy in general, there is some attempts to systematize envy-freeness into distribution systems.
Particularly the work of Dworkin. (cite,cite,cite)







\subsection{Summary}
Summary what you discussed in this chapter, and mention the story in next
chapter. Readers should roughly understand what your thesis takes about by only reading
words at the beginning and the end (Summary) of each chapter.




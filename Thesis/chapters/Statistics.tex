

\section{Stratified sampling by a stratified probability bound}\label{section:SEBB}

In the previous section we considered different possible EBBs as a way of bounding the error in the context of stratified random sampling, and for this purpose developed a new EBB.
This process of using EBBs involved binding EBBs applied to different strata together using union bounds to create a bound on the stratified sample mean error (via Theorem \ref{triangle_theorem1}).

What is worth noting is that this process of binding EBBs together by probability unions is expected to result in a rather weak bound and that this weakness is expected to increase with larger numbers of strata as there are more probability unions needed to bind it together.
It is noted that the triangle equality $|A+B|\le |A|+|B|$, is only an equality in the event that the elements $A$ and $B$ are of the same sign, and in the context of theorems \ref{triangle_theorem1} and \ref{triangle_theorem2} the bound for the error is developed by effectively assuming all the errors of the estimates of the strata are additive - which is the worst case.
Whereas by assumption, the errors in the strata estimates are independent of each other and hence a overestimation in one stratum estimate is likely to be somewhat countered by an underestimation in another.
By using this knowledge stronger bounds can be created, and in this section we do just that. We have created an empirical (ie. depending on sample variances) concentration inequality specifically for stratified random sampling.

The resulting concentration inequality gives an analytic bound on the error of the stratified mean and explicitly considers the sample variances, data widths, sample numbers, and any additional weights on the strata; and includes factors specifically for strata sampled with and/or without replacement.

We proceed with the derivation of this new bound and method in a series of stages:
\begin{enumerate}
\item	in subsection \ref{subsection:new_bounds} we outlay some lemmas which are the building blocks of further derivations in this section, these are new upper bounds are for the moment generating function of samples and sample squares, as well as the sample means depending on whether sampling is done with or without replacement.
\item	in subsection \ref{sec:constructing} we use these elements to begin the derivation of the new bound, called the Stratified Empirical Bernstein Bound (SEBB)
\item	in subsection \ref{subsection:SECB} we additionally derive a variant of the SEBB which uses Chebyshev's inequality
\item	in subsection \ref{sec:SEBMalgorithm} we describe the algorithm of choosing samples in stratified sampling to minimise the new SEBB bound.
\end{enumerate}

After these subsections, in the next section \ref{section:statistics_results} we consider the numerical performance of minimising these new bounds in the context of stratified sampling.
Let us begin with the elements in these new derivations.


%The derivation involves and utilises much of the same components of the previous section with an additional martingale theorem specifically for any any strata sampled without replacement.

\subsection{Some new bounds on the moment generating function}\label{subsection:new_bounds}

To begin the derivation of our new concentration inequality for stratified random sampling, we build upon some of the results of the previous sections.
Specifically we utilise three upper bounds for various moment generating functions.
The first of which has already been given in the previous section, and is Hoeffding's lemma - Lemma \ref{Hoeffdings_lemma_lemma}.
The other two are given here, and are bounds strongly related to Theorems \ref{hoeffdings1} and \ref{sample_squares} respectively.

The first of these other two upper bounds is very much like Hoeffding's Lemma, except it involves additional information about the variance of the random variable.

\begin{lemma}\label{expectation1}
For a random variable $X$ that is bounded on an interval $a\le X\le b$ with $D=b-a$ and variance $\sigma^2$, and any $s>0$:
\[
\E\left[\exp(s(X-\E[x]))\right] 
\le\exp\left(\left(\frac{D^2}{17}+\frac{\sigma^2}{2}\right)s^2\right)
\nonumber
\]
\end{lemma}
\begin{proof}
We assume without loss of generality that $X$ is centred to have a mean of zero.
Then we construct an upper bound for $\E\left[\exp(sX)\right]$ in terms of $D$ by a parabola over $\exp(sX)$ for the permitted values of $X$ in the same way as in the proof of Theorem \ref{hoeffdings1}. By Lemma \ref{thm:parabola} there exists an $\alpha,\beta,\gamma$ such that $\alpha s^2X^2+\beta sX+\gamma\ge \exp(sX)$, and for all $a\le X\le b$, hence:
\[
E\left[\exp(sX)\right] \le \E[\alpha s^2X^2+\beta sX+\gamma] = \alpha s^2\E[X^2]+\gamma = \alpha s^2\sigma^2+\gamma
\nonumber\]
Where it follows that:
\begin{equation}\label{eqn:intermediate_lemma3.5}
E\left[\exp(sX)\right] \le\left(\frac{\sigma^2}{b^2}\exp\left(s\left(b+\frac{\sigma^2}{b}\right)\right) + 1\right)\exp\left(-\frac{s\sigma^2}{b}\right)\left(\frac{\sigma^2}{b^2} + 1\right)^{-1}.
\nonumber 
\end{equation}
This relationship is exactly as in Theorem \ref{hoeffdings1}, now we do something slightly different - the expression in \eqref{eqn:intermediate_lemma3.5} is monotonically increasing with $b$, and $D>b$, therefore substituting $D$ for $b$ gives:
\begin{equation}\label{eq:part1}
\log(E\left[\exp(sX)\right]) \le \log\left(\frac{\sigma^2}{D^2}\exp\left(s\left(D+\frac{\sigma^2}{D}\right)\right) + 1\right)-\frac{s\sigma^2}{D} - \log\left(\frac{\sigma^2}{D^2} + 1\right)
\end{equation}
Because it is true that for any $\kappa,x\ge 0$, that: 
\begin{equation}\label{eq:part2}
\log(\kappa\exp(x)+1)\le\log(\kappa+1)+\frac{x\kappa}{\kappa+1}+x^2\frac{\frac{1}{17}+\frac{\kappa}{2}}{(\kappa+1)^2}
\end{equation}
Thus letting $\kappa=\frac{\sigma^2}{D^2}$ and $x=s(D+\sigma^2/D)$ if follows that:
\begin{equation}\label{eq:part3}
\log(E\left[\exp(sX)\right]) \le \left(\frac{D^2}{17}+\frac{\sigma^2}{2}\right)s^2\qedhere
\end{equation}
\end{proof}

We note that this process of fitting a parabola over the exponential function is exactly the same process as used to derive Bennett's inequality (Theorem \ref{hoeffdings1}), but that we derive a weakened result from that same approach.

The next lemma that we present, is similar to the former, however this time we consider the random variable $X^2$ instead of $X$, and present a weakened bound on the moment generating function of it via a similar process as was used to derive Theorem \ref{sample_squares}: 

\begin{lemma}\label{expectation2}
Let $X$ be a random variable of finite support on an interval $a\le X\le b$, with $D=b-a$ and variance $\sigma^2 = \E[(X-\E[x])^2] = \E[X^2]-\E[X]^2$. Then for any $q>0$:
$$\E[\exp(q(\sigma^2-(X-\E[X])^2))] \le \exp\left(\frac{1}{2}\sigma^2q^2D^2\right)$$
\end{lemma}
\begin{proof}
We assume without loss of generality (and for ease of presentation) that X is centred to have a mean of zero.
We construct an upper bound for $\E\left[\exp(-qX^2)\right]$ in terms of $D$ by a parabola over $\exp(-qX^2)$ for the permitted values of $X$.

For an $\alpha,\gamma$ such that $\alpha X^2+\gamma\ge \exp(-qX^2)$ then:
$$ \E[\exp(-qX^2)] \le \alpha \sigma^2 + \gamma.$$
If $d=\max(b,-a)$ we can choose $\gamma=1$ and $\alpha=(\exp(-q d^2)-1)d^{-2}$ (see figure \ref{fig:graph111}), Thus:
\begin{align*}
\E[\exp(-q X^2)] &\le \frac{\sigma^2}{d^2}\exp(-q d^2)-\frac{\sigma^2}{d^2} + 1\le \frac{\sigma^2}{D^2}\exp(-q D^2)-\frac{\sigma^2}{D^2} + 1 \\
&\le \exp\left(\log\left(\frac{\sigma^2}{D^2}\exp(-q D^2)-\frac{\sigma^2}{D^2} + 1\right)\right)
\end{align*}
Given that for any $0\le \kappa \le 0.5$ and $x\le 0$ that: $$\log\left(\kappa\exp(x)-\kappa + 1\right) \le \kappa x+\frac{1}{2}\kappa(1-\kappa)x^2$$
Letting $\kappa=\frac{\sigma^2}{D^2}$ and $x=-qD^2$, which is valid by Popoviciu's inequality (see \cite{zbMATH05780164}) $\sigma^2\le D^2/4$, then:
$$ \E[\exp(-q X^2)] \le \exp\left(\frac{1}{2}\sigma^2q^2(D^2-\sigma^2)-\sigma^2q\right) \le \exp\left(\frac{1}{2}\sigma^2q^2D^2-\sigma^2q\right)$$
and the result follows by multiplying by $\exp(q\sigma^2)$.
\end{proof}

The inequalities above, Lemmas \ref{expectation1} and \ref{expectation2}, as well as Lemma \ref{Hoeffdings_lemma_lemma}, are used in the derivation of our stratified sampling concentration inequality in Section~\ref{sec:constructing}. One of the primary reasons for utilising these weakened bounds on the moment generating function is that they make the subsequent mathematics far more tractable.
However in order to use these moment generating functions we need to explicitly describe the difference between the moment generating functions of individual random variables (which these are) and the moment generating function of the sample mean of them.

\subsubsection{Some bounds on the moment generating function of sample means}\label{sec:without_replacement}

In order to use the previous bounds on the moment generating function we need a relationship between the moment generating function of a random variable, and the moment generating function of the average of samples of that random variable.
To do this we state two further inequalities, where the first one (Lemma~\ref{martingale1}) is most appropriate for sampling average is taken with replacement, and the second (Lemma~\ref{martingale0}) can optionally be used in the context that the sampling average is without replacement - and may (or may not) give a tighter result.

We first state a lemma that is essentially is a formalisation of the process we are familiar with in the last section (see Lemma \ref{chernoff1}):

\begin{lemma}[Replacement Bound]\label{martingale1}
Let $X$ be a random variable that is bounded $a\le X\le b$ with a mean of zero, with $D=b-a$ and variance $\sigma^2$.
Let $\chi_m = \frac{1}{m}\sum_{i=1}^mX_i$ be the average of $m$ independently drawn (with replacement) samples of this random variable.
If there exists an $\alpha, \beta \ge 0$ such that for any $s>0$ that $\E[\exp(sX)]\le\exp((\alpha D^2 +\beta \sigma^2) s^2)$ then:
$$\textstyle\E[\exp(s\chi_m)]\le\exp(\alpha s^2D^2\frac{1}{m} +\beta s^2\sigma^2 \frac{1}{m}) = \exp((\alpha D^2{\Omega}_m^n +\beta\sigma^2 {\Psi}_m^n)s^2)$$
where ${\Omega}_m^n = {\Psi}_m^n = \frac{1}{m}$
\end{lemma}
\begin{proof} 
By the independence of samples, we have:
\[\E[\exp(s\chi_m)]=\E\left[\exp\left(\frac{s}{m}\sum_{i=1}^mX_i\right)\right]=\prod_{i=1}^m\E\left[\exp\left(\frac{s}{m}X\right)\right]\] 
Thus:
\[\E[\exp(s\chi_m)] \le \exp\left(\frac{s^2}{m^2}\sum_{i=1}^m\left(\alpha D^2 +\beta \sigma^2\right) \right) \qedhere\]
\end{proof}

%The requirement that the mean of the random variable is zero is useful as further derivations are applied to random variables which are shifted such as to have a mean of zero ie. $X-\E[X]$.
For the case of sampling without replacement, there is an alternative result that can be directly substituted, given in Lemma~\ref{martingale0}, below, which can be tighter in certain cases.
Before this, particular note must be made that the inequality above, Lemma~\ref{martingale1} can be used in the context of either sampling with or without replacement.
In contrast, Lemma~\ref{martingale0} can only be used when sampling without replacement. 
This distinction was shown to be true by \cite{hoeffding1}, and is rooted in an already presented Lemma \ref{hoeffdings_reduction}.

We now state Lemma~\ref{martingale0}, an inequality regarding the moment generating function of the average of samples taken specifically \textit{without replacement}.
When the sampling takes place without replacement the inequality of Lemma~\ref{martingale1} can potentially be tightened to take advantage of the finite size of the population.
This inequality extends an important martingale inequality from \cite{bardenet2015}:

\begin{lemma}[Martingale Bound]\label{martingale0}
For finite data $x_1,x_2,\dots x_n$ that is bounded $a\le x_i\le b$, and has a mean of zero and variance $\sigma^2=\frac{1}{n}\sum_{i=1}^nx_i^2$, denote $X_1,X_2,\dots,X_n$ the random variables corresponding to the data sequentially drawn randomly without replacement, and $\chi_m$ the average of the first $m$ of them.
If for any random variable $Z$ with a mean of zero such that $a\le Z\le b$ and $D=b-a$, with variance $\sigma_Z^2$ that there exists an $\alpha, \beta \ge 0$ such that for any $s>0$ that $\E[\exp(sZ)]\le\exp((\alpha D^2 +\beta \sigma_Z^2) s^2)$ then:
\begin{align*}\E[\exp(s\chi_m)]&\le\exp\left(\alpha s^2D^2\sum_{k=m}^{n-1}\frac{1}{k^2} +\beta s^2\sigma^2 \sum_{k=m}^{n-1}\frac{n}{k^2(k+1)}\right)\\ &\le \exp((\alpha D^2\bar{\Omega}_m^n +\beta\sigma^2 \bar{\Psi}_m^n)s^2)\end{align*}
where $\bar{\Omega}_m^n = \sum_{k=m}^{n-1}\frac{1}{k^2}\approx \frac{(m+1)(1-m/n)}{m^2}$ and $\bar{\Psi}_m^n = \sum_{k=m}^{n-1}\frac{n}{k^2(k+1)}\approx \frac{n+1-m}{m^2}$.
\end{lemma}
\begin{proof}
Observe that:
\begin{align*}
\chi_m 
& =\frac{1}{m}\sum_{i=1}^{m}X_i = \chi_{m+1}+\frac{1}{m}(\chi_{m+1}-X_{m+1})\\
& =(\chi_m-\chi_{m+1})+(\chi_{m+1}-\chi_{m+2}) + \dots + (\chi_{n-1}-\chi_n)\\
& =\frac{1}{m}(\chi_{m+1}-X_{m+1})+\frac{1}{m+1}(\chi_{m+2}-X_{m+2}) + \dots + \frac{1}{n-1}(\chi_n-X_n).
\end{align*} 
Then because:
$$\exp(s\chi_m)=\prod_{k=m}^{n-1}\exp\left(\frac{s}{k}(\chi_{k+1}-X_{k+1})\right),$$
we also have that: 
$$\E[\exp(s\chi_m)]=\E\left[\prod_{k=m}^{n-1}\E\left[\exp\left(\frac{s}{k}(\chi_{k+1}-X_{k+1})\right)|\chi_{k+1}\dots \chi_n\right]\right]$$
by repeated application of the Law of total expectation. 
Since:
$$\E[X_{k+1}|\chi_{k+1}\dots \chi_n]=\chi_{k+1},$$ 
then $\chi_{k+1}-X_{k+1}$ is a random variable with a mean of zero bounded within width $D$, and it also has a variance given by: 
\begin{equation}\label{approx1} 
\sigma_{k+1}^2 = \frac{n\sigma^2-\sum_{j=k+1}^nX_j^2}{n-(n-k-1)} - \chi_k^2 \le \frac{n\sigma^2}{k+1}
\end{equation}
by application of Lemma~\ref{variance1}. 
Therefore: 
\[\E[\exp(s\chi_m)]\le\exp\left(\sum_{k=m}^{n-1}\left(\alpha D^2 +\beta \frac{n\sigma^2}{k+1}\right) \frac{s^2}{k^2}\right)\qedhere\]
\end{proof}


This martingale result relates the moment generating function bound of the average of finite variables relative to their mean, to the moment generating function bounds of the  differences of the incremental averages relative to their mean. 
We note that this result could potentially be made much stronger by working around the use of Equation~\eqref{approx1}, but this comes at a cost of increased mathematical complexity.

Since Lemmas~\ref{martingale0} and~\ref{martingale1} share a common form, and because of Hoeffding's reduction (Lemma~\ref{hoeffdings_reduction}), 
all the derivations that follow that invoke Lemma~\ref{martingale1} have direct analogues using Lemma~\ref{martingale0} for the context of sampling without replacement.
Note, however, that the bound without replacement (Lemma~\ref{martingale0}) may or may not be tighter than the bound with replacement (Lemma~\ref{martingale1}). However, the process of substituting one for the other can be done judiciously on a case-by-case basis to create the tightest possible bound.
All the numerical results in Section \ref{section:statistics_results} (that are relevant to sampling without replacement) have been produced with this judicious choice conducted.

\subsection{The Stratified finite Empirical Bernstein Bound (SEBB)}
\label{sec:constructing}

In this section we derive a novel probability bound for the error of the stratified random sampling estimate, 
and use it to define a sequential stratified sampling algorithm. 
Before this, we begin by precisely defining the context of our derivations, to which our bound applies.

\begin{definition}[Problem context]\label{def:ProblemContext}\hspace{1cm}\\
\begin{itemize}
    \item Let a population consist of $n$ number of strata,
    \item where $n_i$ is the total number of data points in the $i$th stratum.
    \item All values in a stratum are bound within a finite support of width $D_i$.
    \item the mean of the $i$th stratum is $\mu_i$, and its variance $\sigma_i^2$.
    \item the random variables corresponding to the samples drawn from the $i$ stratum are:\\ $X_{i,1},X_{i,2},\dots,X_{i,n_i}$
    \item for each stratum $m_i$ samples are taken
    \item forming the sample mean of each stratum: $\chi_{i,m_i}= \frac{1}{m_i}\sum_{j=1}^{m_i}X_{i,j}$
    \item the biased sample variance of each stratum: $\hat{\sigma}_i^2=\frac{1}{m_i}\sum_j^{m_i}(X_{i,j}-\chi_{i,m_i})^2$
    \item the unbiased sample variance of each stratum: $\doublehat{\sigma}_i^2 = m_i\hat{\sigma}_i^2/(m_i-1)$
    \item we consider the average of the means of the strata as weighted by constant positive factors $\{\tau_i \}_{i\in \{1,\dots,n\}}$
    \item And throughout the derivation we also use temporary arbitrary positive variables $\{\theta_i \}_{i\in \{1,\dots,n\}}$
\end{itemize}
\end{definition}


\noindent The bound is now developed in four theorems, which build on each other in sequence:
\begin{enumerate}
    \item in subsubsection~\ref{subsubsection:variance_assisted_sebb} Theorem~\ref{thm:1} develops a concentration inequality for the error in the stratified population mean estimate $\sum_{i=1}^n\tau_i\chi_{i,m_i}$ in the context of the knowledge of stratum variances.
    \item in subsubsection~\ref{subsubsection:variance_square_error_bound} Theorem~\ref{thm:2} is a concentration inequality of the difference between the stratum variances and sample variances in the context the sum of knowledge of the squared stratum mean errors.
    \item in subsubsection~\ref{subsubsection:a_bound_on_sample_squares} Theorem~\ref{thm:3} is an inequality directly that binds the sum of sample squared stratum mean errors.
    \item in subsubsection~\ref{subsubsection:centerpiece} Theorem~\ref{thm:SEBM_bound} combines the three previous theorems together using two union bounds to create a concentration inequality for the error in the stratified population mean estimate given the sample variances.
\end{enumerate}


\subsubsection{A bound on the sample mean assuming variances}\label{subsubsection:variance_assisted_sebb}

In a similar way to what was done in the previous section, to derive an Empirical Bernstein Bound we begin with a derivation of a probability bound on the error of the sample mean of a random variable in terms of the random variable's variance.
In the previous section the bound in question was the venerable Bennett's inequality (Theorem \ref{hoeffdings1}) however in this section we consider a more relaxed and similar version of it which is more easy for us to manipulate, and in this case the random variable in question is the weighted mean $\sum_{i=1}^n\tau_i\chi_{i,m_i}$.

This is probability bound on the absolute error of the weighted stratified sample means about the weighted strata means, which we call a variance-assisted SEBB (stratified empirical Bernstein bound).


\begin{theorem}[Variance-assisted SEBB]\label{thm:1}
Assuming the context given in Definition~\ref{def:ProblemContext}, and let $\Omega_{m_i}^{n_i}$ and $\Psi_{m_i}^{n_i}$ be given as in Lemma~\ref{martingale1}, then:
\begin{equation}\label{eq1} \pr\left(\left|\sum_{i=1}^n\tau_i(\chi_{i,m_i}-\mu_i)\right|\ge \sqrt{4\log(2/t)\sum_{i=1}^n\left(\frac{1}{17}D_i^2\Omega_{m_i}^{n_i}+\frac{1}{2}\sigma_i^2\Psi_{m_i}^{n_i}\right)\tau_i^2}\right)\le t \end{equation}
\end{theorem}
\begin{proof}
In a similar way as Lemma~\ref{chernoff1} we consider a bound for the weighted mean by the moment generating function of the stratum means:
\begin{align*} \pr\left(\sum_{i=1}^n\tau_i\chi_{i,m_i}-\sum_{i=1}^n\tau_i\mu_i\ge t\right)
&\le\E\left[\exp\left(\sum_{i=1}^n\tau_is\left(\chi_{i,m_i}-\mu_i\right)\right)\right]\exp(-st)\\
&= \prod_{i=1}^n\E\left[\exp\left(\tau_is\left(\chi_{i,m_i}-\mu_i\right)\right)\right]\exp(-st) 
\end{align*}
This involves the assumption that the sampling -between- the strata is independent.
This form is sufficient for Lemma~\ref{martingale1} with Lemma~\ref{expectation1} to apply, resulting in a double-sided tail bound:
$$ \pr\left(\left|\sum_{i=1}^n\tau_i(\chi_{i,m_i}-\mu_i)\right|\ge t\right)\le 2\exp\left(\sum_{i=1}^n\left(\frac{1}{17}D_i^2\Omega_{m_i}^{n_i}+\frac{1}{2}\sigma_i^2\Psi_{m_i}^{n_i}\right)\tau_i^2s^2 - st\right) $$
Minimizing with respect to $s$ and rearranging gives result.
\end{proof}

This particular bound assumes knowledge of the variances of the strata, and will be used as a tool for selecting samples primarily for comparison with other sampling methods (such as Neyman sampling) in Section \ref{section:statistics_results}.

It is a concentration inequality for the weighted stratum means, leading to the more general question of what the weights should be. In most cases, the weights $\tau_i$ can be considered as the probability weights $\tau_i=n_i/(\sum_{j=1}^nn_j)$ of standard stratified sampling.
In this context this probability bound can be used as-is for a measure of uncertainty in stratified random sampling if the true variances (or alternatively, upper bounds on the true variances) of the strata are known.
However the weights may be assigned differently - and we will investigate such a case for the Shapley Value in section \ref{section:statistics_results}.

The bound depends on a weighted sum of strata variances $ \sum_{i=1}^n \sigma_i^2 \Psi_{m_i}^{n_i} \tau_i^2 $, and in most situations these values aren't known.
Hence to use this inequality we need to consider the probable error between the weighted sum of strata variances and the weighted sum of strata sample variances.

\subsubsection{A bound on the sample variance in terms of sample error squares}\label{subsubsection:variance_square_error_bound}

To create a bound on the error between the strata variances and the strata sample variances we develop a probability bound for the error estimate of the sum of variances (as weighted by arbitrary $\theta_i$) in terms of the sample square errors (which will also consequently be bounded), as follows:

\begin{theorem}\label{thm:2}
Assuming the context given in Definition~\ref{def:ProblemContext}.
Then with $\Psi_{m_i}^{n_i}$ per Lemma~\ref{martingale1}:
\begin{equation}\label{eq2} \pr\left(\sum_{i=1}^n\theta_i(\sigma_i^2-\hat{\sigma}_i^2 - (\mu_i - \chi_{i,m_i})^2)\ge \sqrt{2\log(1/y)\sum_{i=1}^n\sigma_i^2\theta_i^2D_i^2\Psi_{m_i}^{n_i}}\right) \le y \end{equation}
\end{theorem}


\begin{proof}
To create a probability bound for the sum of variances (weighted by arbitrary positive $\theta_i$), we consider the average square of samples about the strata means. 
Applying Lemma~\ref{chernoff1} gives:
\begin{align*} 
\pr&\left(\sum_{i=1}^n\theta_i(\sigma_i^2-\frac{1}{m_i}\sum_{j=1}^{m_i}(X_{i,j}-\mu_i)^2)\ge y\right) \\
&\le \E\left[\exp\left(\sum_{i=1}^ns\theta_i\left(\sigma^2-\frac{1}{m_i}\sum_{j=1}^{m_i}(X_{i,j}-\mu_i)^2\right)\right)\right]\exp(-sy)\\
& \le \exp(-sy)\prod_{i=1}^n\E\left[\exp\left(\frac{s\theta_i}{m_i}\sum_{j=1}^{m_i}(\sigma^2-(X_{i,j}-\mu_i)^2)\right)\right] 
\end{align*}
And this reason is by the assumption of the independence of the sampling -between- the strata. 
This resulting form is sufficient for Lemma~\ref{martingale1} with Lemma~\ref{expectation2} to apply, giving:
$$ \pr\left(\sum_{i=1}^n\theta_i(\sigma_i^2-\frac{1}{m_i}\sum_{j=1}^{m_i}(X_{i,j}-\mu_i)^2)\ge y\right) \le \exp\left(\frac{1}{2}\sum_{i=1}^n\sigma_i^2\theta_i^2s^2D_i^2\Psi_{m_i}^{n_i}-sy\right)$$
Minimizing with respect to $s$, rearranging, and applying Lemma \ref{variance1} gives result.
\end{proof}

This inequality gives the probability bound between the arbitrarily (by $\theta_i$) weighted variances of the strata and also the same weighted (biased estimator) sample variances.
However it also additionally involves the weighted square error of the sample means as a complicating factor.
Although the weighted square error of the sample means may go to zero quickly as additional samples are taken, we nonetheless need to develop another probability bound to incorporate the specific consideration of it.

\subsubsection{A bound on weighted sample error squares}\label{subsubsection:a_bound_on_sample_squares}

In the previous Theorem \ref{thm:2} the weighted sum of sample squares was a complicating factor which we seek to bound and incorporate. The following probability inequality bounds the weighted square error of the sample means directly:

\begin{theorem}\label{thm:3}
Assuming the context given in Definition~\ref{def:ProblemContext}.
Then with $\Omega_{m_i}^{n_i}$ as in Lemma~\ref{martingale1}:
\begin{equation}\label{eq1.5} \pr\left(\sum_{i=1}^n\theta_i(\mu_i - \chi_{i,m_i})^2 \ge \frac{\log(2n/r)}{2}\sum_{i=1}^n\theta_iD_i^2\Omega_{m_i}^{n_i}\right) \le r \end{equation}
\end{theorem}
\begin{proof}
We consider the weighted square error of the sample means, and by probability complimentarity we know:
$$ \pr\left(\sum_{i=1}^n\theta_i(\mu_i - \chi_{i,m_i})^2 \ge r\right) = 1-\pr\left(\sum_{i=1}^n\theta_i(\mu_i - \chi_{i,m_i})^2 < r\right) $$

As the probability that sum of any random variables is less than $r$ is obviously greater than the probability that those random variables individual are all less than specific values that sum to $r$ hence, for $r_i$ such that $\sum r_i=r$:

$$ \pr\left(\sum_{i=1}^n\theta_i(\mu_i - \chi_{i,m_i})^2 \ge r\right) \le 1-\prod_{i=1}^n\pr\left(\theta_i(\mu_i - \chi_{i,m_i})^2 < r_i\right) $$
hence by probability complimentarities again:
\begin{align*}&\pr\left(\sum_{i=1}^n\theta_i(\mu_i - \chi_{i,m_i})^2 \ge r\right) \le \\&\qquad\qquad 1-\prod_{i=1}^n\left(1-\pr\left(\mu_i - \chi_{i,m_i} \ge \sqrt{\frac{r_i}{\theta_i}}\right) - \pr\left( \chi_{i,m_i}-\mu_i \ge \sqrt{\frac{r_i}{\theta_i}}\right)\right)\end{align*}

And this is exactly the form which we want, in terms of the products of the error in each of the strata sample means.
Thus we can apply Lemma~\ref{chernoff1} together with Lemmas~\ref{martingale1} and~\ref{Hoeffdings_lemma_lemma}, which gives:
$$ 
\pr\left(\sum_{i=1}^n\theta_i(\mu_i - \chi_{i,m_i})^2 \ge r\right) \le 1-\prod_{i=1}^n\left(1-2\exp\left(-\frac{2r_i}{\theta_iD_i^2\Omega_{m_i}^{n_i}}\right)\right)
$$
Next, choosing $r_i$ to minimise the right hand side of this expression gives:
$$
r_i = \frac{r\theta_iD_i^2\Omega_{m_i}^{n_i}}{\sum_j \theta_jD_j^2\Omega_{m_j}^{n_j}}
$$
In this context, we therefore deduce that:
$$ 
\pr\left(\sum_{i=1}^n\theta_i(\mu_i - \chi_{i,m_i})^2 \ge r\right) \le 1-\prod_{i=1}^n\left(1-2\exp\left( \frac{-2r}{\sum_j\theta_j D_j^2\Omega_{m_j}^{n_j}}\right)\right)
$$
Using the fact that $\log(1-(1-\exp(x))^n)\le x+\log(n)$ for any negative $x$, and rearranging, gives the required result.
\end{proof}

This theorem directly bounds the weighted square errors of the sample means independently of any other specifically unknown factors.

\subsubsection{The centerpiece of the SEBB}\label{subsubsection:centerpiece}

In the previous three theorems we have a bound on the stratified mean estimate in terms of the variances, a bound on the variances in terms of the sample variances by the sample square errors, and a bound on the sample square errors.
In the next step we combine all the inequalities of Equations~\eqref{eq1}, \eqref{eq2} and~\eqref{eq1.5} from Theorems \ref{thm:1}, \ref{thm:2} and \ref{thm:3} together, to complete our derivation of a probability bound for the error in stratified sampling in terms of stratum sample variances - our SEBB.


\begin{theorem}[Stratified Empirical Bernstein Bound (SEBB)]\label{thm:SEBM_bound}
Assuming the context given in Definition~\ref{def:ProblemContext}.
Then with $\Omega_{m_i}^{n_i},\Psi_{m_i}^{n_i}$ per Lemma~\ref{martingale1}:
\begin{equation}\label{big_equation}
%\p\left(\frac{\left|\sum_{i=1}^n\tau_i(\chi_{i,m_i}-\mu_i)\right|}{\sqrt{\log(6/p)}}\ge \sqrt{\begin{matrix*}[l]\sum_{i=1}^n\frac{4}{17}\Omega_{m_i}^{n_i}D_i^2\tau_i^2 \\ +\begin{pmatrix*}[l]\sqrt{\log(3/p)\left(\max_i\tau_i^2{\Psi_{m_i}^{n_i}}^2D_i^2\right)} \\ +\sqrt{\begin{matrix*}[l]2\sum_{i=1}^n\tau_i^2\Psi_{m_i}^{n_i}(m_i-1)\doublehat{\sigma}_i^2/m_i \\ + \log(6n/p)\sum_i\tau_i^2D_i^2\Omega_{m_i}^{n_i}\Psi_{m_i}^{n_i} \\ +\log(3/p)\left(\max_i\tau_i^2{\Psi_{m_i}^{n_i}}^2D_i^2\right)\end{matrix*}} \end{pmatrix*}^2\end{matrix*}}\right)
%\le p 
%\end{equation}
%\begin{equation}\label{big_equation_alternate}
\pr\left(\left|\sum_{i=1}^n\tau_i(\chi_{i,m_i}-\mu_i)\right| 
\ge \sqrt{\log(6/p)\left( \alpha
+ \left(\sqrt{\beta} 
+ \sqrt{\gamma}\right)^2\right) } \right)
\le p 
\end{equation}
where:
\begin{align*}
\alpha=&\sum_{i=1}^n\frac{4}{17}\Omega_{m_i}^{n_i}D_i^2\tau_i^2 \\
\beta=&\log(3/p)\left(\max_i\tau_i^2{\Psi_{m_i}^{n_i}}^2D_i^2\right) \\
\gamma=& 2\sum_{i=1}^n\tau_i^2\Psi_{m_i}^{n_i}(m_i-1)\doublehat{\sigma}_i^2/m_i
+ \log(6n/p)\sum_i\tau_i^2D_i^2\Omega_{m_i}^{n_i}\Psi_{m_i}^{n_i} \\
&\quad\quad\quad\quad\quad\quad\quad\quad\quad\quad\quad~+\log(3/p)\left(\max_i\tau_i^2{\Psi_{m_i}^{n_i}}^2D_i^2\right).
\end{align*}


%$$
%\alpha_{m_i}^{n_i}
%=\sum_{i=1}^n\frac{4}{17}\Omega_{m_i}^{n_i}D_i^2\tau_i^2 
%$$
%$$
%\beta_{m_i}^{n_i}
%=\log(3/p)\left(\max_i\tau_i^2{\Psi_{m_i}^{n_i}}^2D_i^2\right) 
%$$
%and
%\begin{align*}
%\gamma_{m_i}^{n_i}
%= 2\sum_{i=1}^n\tau_i^2\Psi_{m_i}^{n_i}(m_i-1)\doublehat{\sigma}_i^2/m_i
%&+ \log(6n/p)\sum_i\tau_i^2D_i^2\Omega_{m_i}^{n_i}\Psi_{m_i}^{n_i}  \\
%&+\log(3/p)\left(\max_i\tau_i^2{\Psi_{m_i}^{n_i}}^2D_i^2\right)
%\end{align*}
\end{theorem}


\begin{proof}
By widening the bound of Equation~\eqref{eq2} we get:
$$
\pr\left(\begin{matrix*}[l]\sum_{i=1}^n\theta_i\sigma_i^2-\sum_{i=1}^n\theta_i(\hat{\sigma}_i^2+(\mu_i - \chi_{i,m_i})^2)\ge \\ \quad\quad\quad\quad\quad \sqrt{2\log(1/y)(\max_i\theta_iD_i^2\Psi_{m_i}^{n_i})\sum_{i=1}^n\theta_i\sigma_i^2}\end{matrix*}\right) \le y.
$$
Completing the square gives for $\sqrt{\sum_{i=1}^n\theta_i\sigma_i^2}$ gives:
$$
\pr\left(\sqrt{\sum_{i=1}^n\theta_i\sigma_i^2} \ge \begin{matrix*}[l]\quad\sqrt{\begin{matrix*}[l]\sum_{i=1}^n\theta_i(\hat{\sigma}_i^2+(\mu_i - \chi_{i,m_i})^2)\\ ~ + \frac{\log(1/y)}{2}\left(\max_i\theta_iD_i^2\Psi_{m_i}^{n_i}\right)\end{matrix*}} \\  +\sqrt{\frac{\log(1/y)}{2}\left(\max_i\theta_iD_i^2\Psi_{m_i}^{n_i}\right)}\end{matrix*}\right) \le y. 
$$
Combining with Equation~\eqref{eq1.5} with a union bound (Lemma~\ref{prob_union}) gives:
\begin{equation}\label{eq:sum_variance_bound_equation}
\pr\left(\sqrt{\sum_{i=1}^n\theta_i\sigma_i^2} \ge \begin{matrix*}[l]\quad\sqrt{\begin{matrix*}[l]\sum_{i=1}^n\theta_i\hat{\sigma}_i^2+ \frac{\log(2n/r)}{2}\sum_i\theta_iD_i^2\Omega_{m_i}^{n_i} \\ ~+ \frac{\log(1/y)}{2}\left(\max_i\theta_iD_i^2\Psi_{m_i}^{n_i}\right)\end{matrix*}} \\ +\sqrt{\frac{\log(1/y)}{2}\left(\max_i\theta_iD_i^2\Psi_{m_i}^{n_i}\right)}\end{matrix*}\right) \le y+r ,
\end{equation}
which is a bound for the weighted sum variances in terms of the sample variances.
Letting $\theta_i = \frac{1}{2}\tau_i^2\Psi_{m_i}^{n_i}$ and combining with \eqref{eq1} with a union bound (Lemma~\ref{prob_union}), 
and then assigning $ r=t=y=p/3 $ and rewriting in terms of unbiased sample variance, gives the result.
\end{proof}

This completes the derivation of the SEBB.
In Equation \eqref{big_equation} of Theorem~\ref{thm:SEBM_bound}, we have a concentration inequality for the sum of weighted strata sample mean errors relative to the sample variances. 
In this context, the weights $\tau_i$ are flexible but would naturally be probability weights proportional to strata size,  $\tau_i=n_i/(\sum_{j=1}^nn_j)$, 
in which case the inequality provides a concentration of measure in stratified random sampling.\\



\subsection{A Stratified Empirical Chebyshev Bound (SECB)}\label{subsection:SECB}

It is also possible to consider another strongly related bound on the error in stratified sampling.
Since the last Theorem \ref{thm:SEBM_bound} ultimately builds upon Theorem \ref{thm:1} which was the embodiment of a simplification of Bennett's inequality per Lemma \ref{expectation1}.
And since we will ultimately compare performance of these bounds against Neyman sampling which is conceptually built upon the minimisation of Chebyshev's inequality (Theorem \ref{thm:chebyshevs}) we will also consider and compare against a empirical Chebyshev's inequality for stratified sampling.


\begin{theorem}[Stratified Empirical Chebyshev Bound (SECB)]\label{thm:SECM_bound}
Assuming the context given in Definition~\ref{def:ProblemContext}.
Then with $\Omega_{m_i}^{n_i},\Psi_{m_i}^{n_i}$ per Lemma~\ref{martingale1}:
\begin{equation}\label{another_big_equation}
\pr\left(\left|\sum_{i=1}^n\tau_i(\chi_{i,m_i}-\mu_i)\right| 
\ge \sqrt{\frac{3}{p}}\left(
\sqrt{\alpha+\beta} 
+ \sqrt{\beta}\right)  \right)
\le p 
\end{equation}
where:
\begin{align*}
\alpha=&\sum_{i=1}^n\tau_i^2\Psi_{m_i}^{n_i}(m_i-1)\doublehat{\sigma}_i^2/m_i+ \frac{\log(6n/p)}{2}\sum_i\tau_i^2\Psi_{m_i}^{n_i}D_i^2\Omega_{m_i}^{n_i} \\
\beta=&\frac{\log(3/p)}{2}\left(\max_i\tau_i^2D_i^2{\Psi_{m_i}^{n_i}}^2\right)
\end{align*}
\end{theorem}
\begin{proof}
We can use Chebyshev's inequality (Theorem \ref{thm:chebyshevs}) for the strata sample estimator giving:
$$ \p\left(\left|\sum_{i=1}^n\tau_i(\chi_{i,m_i}-\mu_i)\right|\ge \frac{1}{\sqrt{k}}\sqrt{\text{Var}\left(\sum_{i=1}^n\tau_i\chi_{i,m_i}\right)}\right) \le k $$
Which is a probability bound for the error of the stratum mean estimator in terms of its variance.
Whereby we can assume the independence of the sampling between the strata and sampling with replacement, giving the decomposition of the variance of the estimator (in a similar process to Equation \ref{eq:variance_decomposition_for_strata_mean}), giving:
$$ \text{Var}\left(\sum_{i=1}^n\tau_i\chi_{i,m_i}\right) = \sum_{i=1}^n\tau_i^2\text{Var}(\chi_{i,m_i}) =  \sum_{i=1}^n\frac{\tau_i^2\sigma_i^2}{m_i} $$
Hence:
$$ \p\left(\left|\sum_{i=1}^n\tau_i(\chi_{i,m_i}-\mu_i)\right|\ge \frac{1}{\sqrt{k}}\sqrt{\sum_{i=1}^n\Psi_{m_i}^{n_i}\tau_i^2\sigma_i^2}\right) \le k $$
Which is a Chebyshev type inequality for the error of the stratum mean estimator in terms of a sum of the stratum variances - and is very analogous to Equation \ref{eq1} of Theorem \ref{thm:1}.
And since equation \ref{eq:sum_variance_bound_equation} (from the previous proof) is a general bound for an arbitrary sum of the stratum variances we can combine with it (in the case of setting $\theta_i = \tau_i^2\Psi_{m_i}^{n_i}$) by a union bound (Lemma~\ref{prob_union}), giving:
\begin{align*}
\pr\left(\left|\sum_{i=1}^n\tau_i(\chi_{i,m_i}-\mu_i)\right| \ge \frac{1}{\sqrt{k}}\left( \begin{matrix*}[l]\quad\sqrt{\begin{matrix*}[l]\sum_{i=1}^n\tau_i^2\Psi_{m_i}^{n_i}\hat{\sigma}_i^2+ \frac{\log(2n/r)}{2}\sum_i\tau_i^2\Psi_{m_i}^{n_i}D_i^2\Omega_{m_i}^{n_i} \\ ~+ \frac{\log(1/y)}{2}\left(\max_i\tau_i^2D_i^2{\Psi_{m_i}^{n_i}}^2\right)\end{matrix*}} \\ +\sqrt{\frac{\log(1/y)}{2}\left(\max_i\tau_i^2D_i^2{\Psi_{m_i}^{n_i}}^2\right)}\end{matrix*}\right)\right)&\\ \le y+&r+k
\end{align*}
And setting $ r=k=y=p/3 $ and rewriting in terms of unbiased sample variance, gives the result.
\end{proof}

We will now turn to how minimising these bounds on the stratified sample mean estimate (theorems \ref{thm:SEBM_bound} and \ref{thm:SECM_bound}) can be used as a method of choosing samples in the context of stratified sampling.
And afterwards, in the next section \ref{section:statistics_results} we will compare how all these methods work in actually sampling stratified data.





\subsection{Sequential sampling using the stratified empirical bernstein method (SEBM)} \label{sec:SEBMalgorithm}

%To evaluate the effectiveness of the process of sequentially minimising the probability bounds on the error of the sample estimated stratified mean, we need a little development of those algorithms before proceeding to give details on the process of testing them in the context of synthetic data.
%It is necessary to consider the way in which the concentration inequality of the previous section (Theorem \ref{thm:SEBM_bound}) should be iteratively minimised; and this process is expounded as Algorithm \ref{alg2}.


In this section, we developed a concentration inequality to bound the error of the sampling mean estimate in stratified random sampling, called the SEBB - as per Theorem \ref{thm:SEBM_bound}.
The process of selecting additional samples to minimise this probability bound on the error is introduced in this section. And we call it the \textit{stratified empirical Bernstein method} (SEBM).

The fundamental principle of the SEBM, is that it is an online method of choosing additional samples which repeatedly scans through possible stratum for optimum sample choice, and the strata which would result in the most reduction of the SEBB bound is then recommended for additional sampling.
The pseudo-code for this process of sampling, is given as Algorithm \ref{alg2}.


Specifically, Algorithm \ref{alg2} is a repetitive process involving a scan through the possible strata and then the selection of one stratum to sample from.
The process of scanning involves calculating the confidence bound width (SEBB) that would result if an additional sample were to be taken from that stratum without changing its sample variance (line numbers 5-17 in Algorithm~\ref{alg2}).
The stratum that yields the smallest confidence bound width in the context of an additional sample is then selected (line 18-21) and sampled (line 24), the sample variance of that stratum is updated (line 26); 
this process repeats until the maximum sample budget is reached (per the outer loop, line 1).
In this way the process attempts to iteratively minimise the SEBB in expectation with each additional sample taken; and hence lead to potentially greater accuracy in stratified sampling as a result.

The primary assumption that exists in this method's selection calculus is that it assumes that the sample variance of the strata would likely remain unchanged for the taking of the additional sample from any strata.
While this technically isn't true, the unbiased sample variance is expected to be almost as likely to increase as it is likely to decrease from the taking of an additional sample.
Plus developing an even more complicated probability bound that explicitly takes account of the likely change in error of the stratified mean estimate due to the expected change in the sample variance is beyond the scope of our investigation.

This SEBM method (Algorithm \ref{alg2}) requires the sample variances of all the strata to be calculated. And accordingly, Algorithm \ref{alg2} must be initialised with at least two samples from each stratum so that sample variance can be calculated for it to be able to function.

We also note that Algorithm \ref{alg2} describes a process specific to the sampling without replacement of all strata, and involves the calculation of the SEBB with the tightest possible use cases of Lemmas \ref{martingale0} and~\ref{martingale1}.
In particular, for any stratum $i$ that is sampled without replacement, any specific bound with an associated $\Omega_{m_i}^{n_i}$ and $\Psi_{m_i}^{n_i}$ may be substituted for $\bar{\Omega}_{m_i}^{n_i}$ and $\bar{\Psi}_{m_i}^{n_i}$ to potentially tighten the bound, and this corresponds to choice of Lemma~\ref{martingale0} or Lemma~\ref{martingale1} in the bound's derivation. 
Since the SEBB is a composition of such bounds with such choices throughout, there is a structure of valid pairs of substitutions $\Omega,\Psi$ for $\bar{\Omega},\bar{\Psi}$ in the optimal calculation of the SEBB, which is shown in the steps 8-15 of Algorithm \ref{alg2}.
The equivalent algorithm for sampling with replacement simply is the same algorithm altered by replacing all use of $\bar{\Omega},\bar{\Psi}$ with $\Omega,\Psi$ respectively.

Similarly it is elementary to modify the terms of Algorithn \ref{alg2} to be amenable to be minimising SECB bound (Theorem \ref{thm:SECM_bound}).

\begin{algorithm}
\caption{Stratified Empirical Bernstein Method (SEBM) algorithm, with replacement}
\label{alg2}
\begin{algorithmic}[1]
    \REQUIRE probability $p$, strata number $N$, stratum sizes $n_i$, initial sample numbers $m_i$, initial stratum sample variances $\doublehat{\sigma}_i^2$, weights $\tau_i$, widths $D_i$, maximum sample budget $B$
    \WHILE{$\sum_i{m_i}<B$}
        \STATE $beststrata \leftarrow -1$
        \STATE $lowestbound \leftarrow \infty$
    	\FOR{$k=0$ to $N$}
    	    \STATE $m_k \leftarrow m_k + 1$
        	\STATE $a \leftarrow [0,0]$, $b \leftarrow [0,0]$, $c \leftarrow [0,0]$, $d \leftarrow [0,0]$
        	\FOR{$i=0$ to $N$}
        		\STATE $a_0 \leftarrow a_0 + \log(6N/p)D_i^2\bar{\Psi}_{m_i}^{n_i}\min(\bar{\Omega}_{m_i}^{n_i},\Omega_{m_i}^{n_i})\tau^2$
        		\STATE $a_1 \leftarrow a_1 + \log(6N/p)D_i^2\Psi_{m_i}^{n_i}\min(\bar{\Omega}_{m_i}^{n_i},\Omega_{m_i}^{n_i})\tau^2$
        		\STATE $b_0 \leftarrow \max(b_0,\log(3/p)D_i^2\bar{\Psi}_{m_i}^{n_i}\min(\bar{\Psi}_{m_i}^{n_i},\Psi_{m_i}^{n_i})\tau^2)$
        		\STATE $b_1 \leftarrow \max(b_1,\log(3/p)D_i^2\Psi_{m_i}^{n_i}\min(\bar{\Psi}_{m_i}^{n_i},\Psi_{m_i}^{n_i})\tau^2)$
        		\STATE $c_0 \leftarrow c_0 + 2\bar{\Psi}_{m_i}^{n_i}((m_i-1)\doublehat{\sigma}_i^2/m_i)\tau^2$
        		\STATE $c_1 \leftarrow c_1 + 2\Psi_{m_i}^{n_i}((m_i-1)\doublehat{\sigma}_i^2/m_i)\tau^2$
        		\STATE $d_0 \leftarrow d_0 + \frac{4}{17}D_i^2\bar{\Omega}_{m_i}^{n_i}\tau^2$
        		\STATE $d_1 \leftarrow d_1 + \frac{4}{17}D_i^2\Omega_{m_i}^{n_i}\tau^2$
        	\ENDFOR
        	\STATE $boundwidth \leftarrow \sqrt{\log(6/p)\min_j(d_j + (\sqrt{c_j + a_j + b_j} + \sqrt{b_j})^2)}$
    	    \IF{$boundwidth < lowestbound$}
    	        \STATE $beststrata \leftarrow k$
    	        \STATE $lowestbound \leftarrow boundwidth$
    	    \ENDIF
    	    \STATE $m_k \leftarrow m_k - 1$
    	\ENDFOR
    	\STATE take an extra sample from strata: $beststrata$
	    \STATE $m_{beststrata} \leftarrow m_{beststrata} + 1$
    	\STATE recalculate $\doublehat{\sigma}_{beststrata}^2$
    \ENDWHILE
\end{algorithmic}
\end{algorithm}



In the next section \ref{section:statistics_results} we will compare how algorithms \ref{alg3} and \ref{alg2} work in actually sampling stratified data.







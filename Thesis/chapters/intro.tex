\chapter{Introduction}
\label{cha:intro}

%\section{Thesis Statement}
%\label{sec:thesisstatement}
%I believe A is better than B.


\section{Introduction}


The way in which electricity has been generated and consumed has changed over time, and presently it is changing yet still.
One of the primary ways that electricity generation is changing is with the introduction of distributed generation.
historically the electricity network was composed of large industrial generation companies - that owned the networks, and a multitude of small private consumers.
several changes have been made over time, with one of the most important being the 




The way that electricity is being produced and consumed is changing in a direction not fully understood, and it has this before.

The means by which electricity has been produced and consumed between individuals and industry players has suffered structural changes over time and these changes are expected to continue into the future.
When Electricity was first introduced in Britain in the 1800s there were different companies offering the service of supplying electricity to the homes of individuals, these companies covered different regions of service in and between cities, and each operated according to their own guidelines on safety and service stability.
These electricity operators owned the poles and lines and supplied energy to houses per subscription at their own voltage and current specifications.
When electricity was first being introduced, its use was in direct competition to other technologies such as gas and coal/keroscene, and even though the electricity companies often had regional monopoly on electricity supply, ther competition between industries kept the pricing in-line.
Through the course of buisiness these companies eventually either went out of buisiness or consolidated into larger coporations, and proceeded to supply larger amounts of electricy at scale to the residents of Britain.
As the people's daily reliance on electricity increased in plurality and popularity of electrical appliances and lighting, greater public pressure came to the fore, for the regulation of the electricity grid.
Primarily for the regulation of Voltages and frequency and for the safe interopration of devices between networks but also to prevent price hikes and extortionate pricing of the electricity companies.



The way that electricity is being produced and consumed is changing in a direction not fully understood, and it has this before.

The means by which electricity has been produced and consumed between individuals and industry players has suffered structural changes over time and these changes are expected to continue into the future.
The challenges today are reminiscient of the challenges historical.
The introduction of new technologies and sources of energy into the strata of society is a process that is prone to problems and features that are historical, political and social.
Before discussing the process of new technologies and their integration into the electricity grid, we will divolve the discussion into some similar features of the historical introduction of electrical grids themselves.
One example of a tension with the introduction of new technology is its regulation.
Today there are institutions such as AEMO and AER which provide industry regulations to incentivise competition and ease monopolising of indistry participants over the concerns of individuals.
It is worth noting that the institution of overly-tight regulations can and potentially has actually hampered the deployment of new technologies.
A particular example from British history is the Electric Lighting act of 1882 which occured in the context of an economic resession; and which was strongly blamed for reducing Brtiain's uptake of electrical energy for a period of 4 years.
The government at the time, was regional-socialist, and



\section{Thesis Outline}
\label{sec:outline}
How many chapters you have? You may have Chapter~\ref{cha:background},
Chapter~\ref{cha:design}, Chapter~\ref{cha:methodology},
Chapter~\ref{cha:result}, and Chapter~\ref{cha:conc}.

\chapter{Introduction}
\label{cha:intro}

The Australian electricity grid is facing structural changes in response to the pressures of new technology.
And the way in which the grid should change to meet these new pressures is cause for reflection on the morally ambiguous question at the heart of electrical transactions on the grid - how \textit{should} electricity be valued and traded?

Or, to ask our research question with more granularity:
when the possible power-flows on an electricity network are valued and influenced by participants differently, 
what is a reasonable electrical outcome for the network and what monetary transactions \textit{should} occur between the participants?

Existing electrical power systems in many countries implicitly embody a historic answer to this essential question, however electrical power systems have a history of evolution over time, particularly with the introduction of new technologies and new demands, and this trend continues today as new technology is driving a need for additional changes.
This technological drive for additional changes is felt by various electricity systems across the world, however here we focus on the Australian case as an instance. 

\section{The changing nature of supply, and the introduction of demand elasticity}\label{sec:intro_part1}

One of most notable technological changes occurring in electricity networks around the world is the development and continued proliferation of renewable energy technologies.
For instance, between the fiscal years 2008/2009 - 2018/2019 the \cite{departmentoftheenvironmentenergy2018} reported that the volume of renewable electricity generated on Australia's National Electricity Market (NEM) increased from 18,645 GWh to 44,292 GWh, an increase of approximately 10\% per year, increasing the proportion of renewable energy in the network from 7.5\% to 17\% over the same period.
This increase was primarily achieved by the deployment of wind and solar generation, with 90\% of solar generation currently being created by small-scale solar photovoltaic (PV) systems, with \cite{australianenergymarketoperatorlimited2018} projecting that solar PV generation will triple by the year 2030.
%And this increasing trend is expected to continue, there were ~300,000 installations in the year 2019 - clean energy australia report 2020, 

In contrast, there is an increasing number number of coal fired power stations being retired from the grid,
with \cite{doi:10.1111/1467-8489.12289} reporting that almost a third of Australia's coal-fired power stations closed between 2012 and 2017,
%Most notably including the Hazelwood Power Station in 2017.
and \cite{australianenergymarketoperatorlimited2018} reporting an extra 12 coal fired power plants are expected to be retired over the next 30 years.

The Australian Energy Market Operator (AEMO) operates multiple markets, but particularly the day-ahead and spot-markets operate on marginal-pricing principles where the most expensive generator that is dispatched to meet demand sets the price which is paid for all dispatched electricity.
In this context the witnessed changes in supply (specifically the retirement of coal fired power and its replacement by variable renewable electricity generation) have created a situation whereby it is increasingly more likely that more expensive generators are setting the marginal price.
This dynamic is identified as being a part of the reason that the average price of electricity for consumers between 2008 and 2018 increased by 35\% \citep{australiancompetitionconsumercommission2018}.

This is an example of a technological change creating an emerging problem for existing electricity networks. The problem is made clear specifically as the stability and efficiency of the grid in providing affordable electricity for customer needs are core components of Australia's National Electricity Objective (NEO)\footnote{the NEO is part of Australia's National electricity law (NEL) ``to promote efficient investment in, and efficient operation and use of, electricity services for the long term interests of consumers of electricity with respect to: price, quality, safety and reliability and security of supply of electricity; and the reliability, safety and security of the national electricity system.''}.

One option to avert this problem is the prospect of a `Demand Response' programs, whereby consumers (large and potentially small) are contracted and paid to reduce their consumption in times of peak demand to avoid dispatch of the more expensive generators which would subsequently set the marginal price, thus making electricity less expensive for everybody.
Demand response programs are one example of participatory scheme designed to interact with consumers to bring greater regularity to electricity grid, and the offering and accepting of such contracts naturally constitutes a new and prospective market structure.
In line with the potential for such a participatory system, there are presently (at time of writing) 12 different Australian Renewable Energy Agency (ARENA) funded Demand response pilot programs across the country (totalling over 100 million dollars of grant money).
Additionally the \cite{australianenergymarketcommission2020} is actively drafting rule changes to formally recognise organisations that provide Demand Response services to the grid, as being electricity market participants directly equivalent to wholesale generators.

The effect of these demand-response programs is to attempt to bring the market price down by creating mechanisms where the demand of electricity effectively becomes more elastic, and one of the more direct way of doing this is to produce a market structure to implement demand-responsive consumption against generation, or a `two sided market'.
The idea of a two-sided market is something that is something that recognised by \cite{australianenergymarketcommission2020} as an enduring solution for the Australian electricity grid, and is currently being investigated by the \cite{energysecurityboard2020}.
This is one example of a potential future market system which may become necessary in response to present technology changes; but it is not the only one.

With the consideration of demand response programs, the question is how the time-limited curtailment of consumption of electricity be traded and valued?

\noindent And this reflects the broader question: ``how should electricity be valued and traded?'' 

\section{Increasingly variable supply, and the potential of battery storage}\label{sec:intro_part12}

%in addition to small solar PV system installations  there was an array of medium and large scale solar installations.
%In the same time additional household batteries are increasingly being installed (with 22,661 household battery installations in the 2019 year) taking Australia's household battery storage to 1GW. with an addtional 15 large scale battery projects scheduled for completion in the 2019 year ranging between 1 and 150MW capacity.
%(clean energy australia report 2020)
The increasing number of solar PV systems connected to the Australian grid is changing the characteristic demand profile on the network and its variability; and this has the potential to create issues surrounding grid frequency and stability.

Solar PV systems are often behind-the-meter and have the effect that they change the amount of electricity that consumers require throughout the day from the grid.
This effect is particularly manifest in the middle of the day where the abundance of solar energy offsets household electricity requirements and create the ever increasingly severe decrease in grid demand (the infamous `duck-curve') which is identified to be one source of potential future difficulty.
Not only does solar PV change the average demand profile on the network but it also recognised as a source of variability; because solar output is linked to weather there is the ready potential for large swings of localised solar power output, which is recognised by \cite{australianenergymarketoperatorlimited2018} as potentially creating stability issues related to frequency and voltage control.
These demand side changes are compounded by supply side changes such as the retirement of traditional generators which provide stabilising inertia, and continued uptake of industrial wind and solar generation capacity.

One of the primary ways in which sudden changes in supply/demand is manifest on the grid is by frequency drift.
In order to keep synchrony, the frequency of the grid needs to be closely monitored and kept on a single frequency and this is an essential stability requirement for the operation of the grid.
When there is an increase in supply or a decrease in demand, there is an excess of power on the network which causes a decrease on the electromagnetic drag on traditional generation rotors, causing them to speed up, leading to a increase in their frequency of rotation and frequency of the power they generate.
Conversely a decrease in supply, or increase in demand, will cause a decrease in frequency. With the rate at which these frequency increase/decrease occurs being related the inertia of generators.

However the decrease in traditional generation technologies which carry generation inertia is expected to cause the system frequency to be more sensitive to supply/demand variability, at the same time that variability is set to increase due to the increase in renewable generation technologies.
There are multiple prospective ways of ameliorating this emerging problem (as discussed by \cite{doi:10.1002/2050-7038.12128}) such as to attempt to build artificial inertia into renewable generators, and another is to requisition generation capacity specifically to maintain system frequency.

Presently there exist a range of electricity markets within the Australian system, including a day-ahead market, a spot-market and several capacity markets specifically for the stabilisation of grid frequency.
Some of these ancillary service markets are the Frequency Control Ancillary Services (FCAS) markets, and there exist six of them, particularly for bidding for the contracts to deposit and/or withdraw power from the electricity system where needed, at an upper, middle and lower timeframe for the response (6 second, 60 seconds, 5 minute respectively).
In light of the technological changes that are being realised it is anticipated that these frequency response markets will become increasingly important to maintain grid stability \citep{RIESZ201586}.

However, a limitation of some traditional generating technologies that operate with large synchronous rotors, is that they are unable to respond quickly to assist in the maintaining of system frequency, and therefore unable to support grid frequency or bid into FCAS markets due to their ramp-rate limitations and costs. \citep{GONZALEZSALAZAR20181497}
And while there does exist some technology for rapid response gas generators (as discussed by \cite{GONZALEZSALAZAR20181497}) the more promising renewable technology for this purpose is battery technologies, which have characteristically fast response times.
While there does presently exist some large institutionally owned grid connected batteries, a large number of batteries currently connected to the grid are residential small scale batteries often connected to solar PV systems.
Additionally it is the expected by \cite{australianenergymarketoperatorlimited2019} that there will be an increase in the adoption of electric vehicles (EVs) with grid connected batteries, with predictions ranging upto 4.5 million EVs on Australian roads by 2040.
In this way there is an expected potential for batteries and electric vehicles to be a source of future grid stability.
%EVs, are predicted to account for 17.7\% of vehicles on australian roads by 2036 (Electric Vehicles Insights AEMO)) 

However there does not currently exist any unified national infrastructure to facilitate these distributed energy resources (DERs) such as EVs and small batteries participating in Australian grid stability.
As presently, network participants are payed for the power they consume/generate largely behind their own meter, inducing them to utilise their DERs to optimise their own energy consumption.
One current avenue being explored to create such infrastructure is the creation of virtual power plants (VPPs) as large scale aggregators of DER power capabilities, and these projects are currently being developed and are the subject of experimentation \citep{australianenergymarketoperatorlimited20188}.
In this way, VPPs are another example of market structure that potentially could emerge in response to current technological trends.
Another avenue currently being ivestigated is the contracting strategic reserve power specifically for grid stability. which are currently being formalised into the Reliability and Emergency Reserve Trader (RERT) rules.

With the consideration of the various ways of trading, valuing and integrating storage in providing diachronic arbitrage of electrical energy to offset frequency deviations, the question is how the storage and time-sensitive rapid injection/consumption of electricity be traded and valued? Because an answer to this should be a consequence of an answer to the broader and more general question: ``how should electricity be valued and traded?''


\section{DERs and the potential for a prosumer era}\label{sec:intro_part13}

Another change that is happening, is that increasingly not only are consumers generating their own electricity via solar panels, but they are also expected to be increasingly storing that energy as well; leading to an increasing level of consumer energy independence and the potential attractiveness of going `off-grid'.
However, there may be future role for electricity consumers not only to participate in network support and have access to market structures; but also ideas that might induce consumers to create more social benefits, such as directly sharing their electricity with each other.
Particularly, there is an idea that the grid might be able to support consumers selling their excess power and storage capacity to each other, as a peer-to-peer electricity trading (P2P) system.

There are many different visions of what the structure the future electricity system might have, and what features it might have to support trading between producer-consumers, or so-called `prosumers', such as considered out by \cite{Parag2016}.
These future visions describe ideas about how prosumers might trade directly or indirectly between themself and the wider grid; such as perhaps between P2P trading and/or aggregation into many larger virtual power plants (VPPs) or via localised community storage. \citep{Morstyn2018}.

However the vision of incorporating prosumers into the national electricity market poses a range of gains and challenges, particularly with regards to the voltage and frequency management on distribution networks.
\cite{BELL2018765} enumerates some of the challenges, such as: providing grid stability by securing timely reactive and real power supply to stabilise voltages and frequency, grid robustness such as blackout protection and islanding, improving system efficiency by minimising long-distance transmission costs, and facilitating the advent of a green electricity network, by fairly and equitably managing the interaction between prosumer's devices while preserving their privacy, in a scalable way while minimising the complexity of the management of such devices.

%The continued proliferation of DERs is a technological change that is expected to create investigation into these future visions of the grid.

With the consideration of the various ways of facilitating the fair trading of energy between heterogeneous devices on distribution networks between consumers, the relevant question is how to integrate these various factors in a flexible and equitable way.

\noindent And this reflects the broader question: ``how should electricity be valued and traded?''


\section{A summary of the future of the Australian grid}\label{sec:intro_summary}

%There are various ways that people are investigating the changes required to the current electricity network.

%Most notably in 2017 Dr Alan Finkel released his review on the future of the Australian electricity "Independent Review into the Future Security of the National Electricity Market" - informally called `the Finkel review', containing a host of 50 recommendations and changes to ensure future grid reliability and operations.

%Among these 50 recommendations, 49 of them were formally accepted by the Turnbull government, including such as: a review of potential frameworks for DER participation, assessment of the need for strategic reserve and/or day-ahead stability market, the review and recommendation of a demand response mechanism, and the creation of an Energy Security Board to implement an Strategic energy plan.

%Some of these recommendations have matterially come to pass, such as the need for contracting strategic reserve power for grid stability being formalised into the Reliability and Emergency Reserve Trader (RERT) rules; and the Energy Security Board has been created and is presently investigating possible price responsive demand programs.
%particularly the production of a market structure to produce demand-responsive supply/consumption against generation (potentially even extending to normal users of electricity systems) a `two sided market' is something that is being scoped out by COAG energy council australia (ECA) 

%A consequence of the changing dynamics of the grid, is that these combinations of changes has resulted in inordinate increase in AEMO direct intervations (AEMO) , and AER rule changes (KPMG) in order to address the emerging situations.
%More generally, this increased variability of demand and supply are part of a changing electricity system that need to be managed and addressed, and there are a variety of potential options.

%One example of such plans is the current COGATI investigation, exploring changes in the NEM that could be made to further incentivse generation and transmission at grid critical locations to minimise congestion experienced on the grid; ie. incentivising the most appropriate network generation and transmission.
%Another broad example of such an investigation isn the so called `NEM2025' plan directly investigates possible changes to market structure to NEM in order to ensure future stable grid, particularly mentioning the potential of utilising DERs.



%One, instance of a design decision is who should be participating on such future markets, particularly the prospect that consumers of electricity could participate in some kind of electricity market (ECA) is up for discussion.


%And evaluating the potential for implementing such a system is potentially part of the NEM2025 investigation.

%peer-to-peer electricity trading and management on small-scale networks - sometimes called microgrids - have been developed in various contexts around the world, including Australia
%These distributed energy resources exist on distribution networks where there is occasional problems with voltage management, where 
%solar inverter reactive power stability is not often utilised. (Smart Grid and its future perspectives in Australia)


%https://arena.gov.au/knowledge-innovation/distributed-energy-integration-program/


The way that electricity is being produced and consumed is expected to change in the future, from a system which supports the supply of energy from a few big generation companies to many small consumers, to being a system in which prosumers potentially exchange energy with each other.
And there is an increasing interest in the design of new market systems that are appropriate for the future electricity grid.

The technological changes witnessed are seen as potentially leading to future problems associated with frequency and grid stability, in which batteries and other distributed energy resources (DERs) are seen as being important; and new market structures are being considered to provide a platform for them to participate in grid operations.

In this context, there are many potentially important engineering considerations in the operation of powergrids which may bear importance for the exchange of energy between such prosumers - such as voltage rises and line limits, real and reactive power compensation, phase connections, network topologies etc, and it is hoped that a properly designed system might be extensible to be able to accommodate these technical considerations where applicable.

Potential future markets are subject to many requirements, outside of simply delivering a reliable and cheap electricity service to consumers, and sometimes they are even directly quoted as subject to ethical design considerations, such as implementing a ``Level playing field'' where:

\begin{displayquote}
``...all competitors, irrespective of their size or financial strength, get equal opportunity to compete. It is not enough if all players play by the same rules. The rules must accommodate the needs of all, whether small or large, so the market is free of impediments to smaller players.''
\\\citep{australianenergymarketoperatorlimited2018}
\end{displayquote}

Ethical criteria such as this, as well as wider political and social implications bear on this discussion. 
Thus our investigation is to explore and evaluate the research question of what kinds of market structures should be implemented in the future.
%In the past, it was held that inducing competition among generation companies was desirable, however it is important to ask if inducing competition among electrically generating prosumers is equally a good idea.
%A part of the answer comes in the likely result: would a competitive system induce outcomes which are socially desirable and/or serve to promote social equality?
%A major component of this question lies in the formulation of the arena in which the competition takes place; which is the primary design question.
%And the purpose of this research is an attempt to explore a portion of the space of possible arenas, and evaluate it against desirable ethics.


\noindent By these considerations we can attempt a new answer to the broader and more general question: ``how should electricity be valued and traded?''


\section{Research and problem approach}

The fundamental research question is `how \textit{should} electricity be valued and traded?', which is general and multifaceted question without an easy or established answer (from previous sections \ref{sec:intro_part1}-\ref{sec:intro_summary})

The research question is easily identified as having a moral and ethical quality, and thus in Chapter \ref{sec:philosophy} we survey some of the relevant philosophy associated with the ethics of \textit{Distributive Justice}.
Distributive Justice is a branch of moral philosophy associated with the distribution of goods and services in society, about which electricity and electrical services are naturally considered as an example.
In this context the philosophy considers different ideas that people have about distribution in relation to more general moral principles, particularly such as various descriptions of `Equality', `Deservedness', `Reward', `Efficiency', etc.
It is unfortunately seen that these broad and blurry ethical ideals conflict between themselves and do not logically imply specific market structures or analytic criteria.


However there do exist various mathematical solutions and structures which have been developed as mathematical formulations of these different ethical perspectives, particularly we review a range of different structures and show how they attempt to describe different broader ethical ideas in Chapter \ref{cha:solutions}.
These existing solutions form a part of the background of our research, and also provide inspiration and contrast for further developments.
For instance, we review the details about the Vickrey-Clarke-Groves (VCG) mechanism, showing that it embodies an idea about compensation for contribution, and identify how it has been proposed as a mechanism for dictating payments between participants on electricity networks.
Other mechanisms which we consider include Locational Marginal Pricing (LMP), cooperative game theory solutions such as the Shapley Value, and descriptions of idealised bargaining such as Nash bargaining solution concepts.
These mechanisms were investigated specifically because of their abstract and general applicability, and hence they could (in theory) account all the practical factors and possible confluences of electrical system details to ascribe value to disparate electrical devices on a future smart-grid.


Each of these mechanisms embody specific ideas about ethics, and they are identified as having features and shortcomings. Because of this we attempted to take the best features of these mechanisms and synthesise a genuinely novel solution for the pricing of electrical resources on electricity networks.
We developed a novel solution concept which we called the \textit{Generalized Neyman and Kohlberg Value} or the \textit{GNK value} in Chapter \ref{cha:new_solution}.

We identified that Nash bargaining was a particularly interesting mechanism which was able to provide a unique and direct answer to the question of how electricity should be traded between two participants that could directly consider all possible ways that the electrical participants could interact and influence each other.
Our new solution, the GNK value, was designed as a generalisation of Nash bargaining to larger numbers of players (more than two), additionally it was made to be even more extensible as it was designed to work in the space of generalised games and so able to account for arbitrary constraints on mutual player interactions.
The GNK value embodies the cooperative game theory axioms (and the marginalism) of the Shapley Value, and contrasts against the more immediate marginalism of VCG and LMP.
We developed and applied the GNK value against VCG and LMP in the context of randomly generated electricity networks, to witness and discuss the differences between them.

The GNK value is rooted in bargaining perspective and rewards participants for the advantage they might have in an idealised competition with others.
It was hoped that an idealised bargaining solution like this would mirror the kinds of arrangements that people with divergent interests would freely come to anyway, and hence would ascribe reasonable economic value to electrical resources in the most natural way; however this process ultimately yielded a disappointing result, as the net result failed specific ethical criteria.
The specific major ethical issue witnessed in the GNK value was that it does not respect the ethical criterion called `individual rationality' - the desirable quality that every participant is ascribed non-negative net utility.
Particularly if zero utility is interpreted as the utility of a non-participant, then individual rationality property implies that every participant is made better-off by participating.

The GNK value extends from the Shapley Value axioms, and so it inherits the NP-hard computational burden associated with the Shapley Value.
However through investigation into sampling techniques and in utilising a particular proxy we were able to extend the GNK value from being intractable for $\sim 14$ bus sized nodal networks, to being readily computable to about $80-100$ sized nodal networks with a standard desktop computer.
This was identified as a computational accomplishment particularly because if the GNK value were to be calculated exactly for a 100 sized nodal network it would involve $2^{100}$ power flow optimisations.
This computational accomplishment was done through a process of considering the different ways that the GNK value value could be sampled, and we developed our own sampling technique called the Stratified Empirical Bernstein Method (SEBM).

The SEBM was derived as a online method of choosing samples in the context of stratified sampling, where the orthodox method of choosing samples (called Neyman sampling) necessarily takes two unique stages to complete.
The development of the SEBM method was conducted in the context of evaluating other methods for sample selection in Chapter \ref{chap:stratified_sampling_chapter}, and the performance of multiple methods were evaluated on sampling synthetic data sets.
All of the stratified sampling methods were identified as being applicable for sampling the Shapley Value and GNK value, and the performance of these sampling methods for such a task was evaluated.
The SEBM method was identified as being computationally expensive but well performing, and the method was extended into a multidimensional form.

\subsection{Contributions}

Within the research program, the primary contributions made are:
\begin{itemize}
\item We developed the GNK value as an extension of Nash bargaining to many players in the context of generalised actions spaces.
\item We developed, applied, and ethically evaluated the GNK value against LMP and VCG in the context of $\sim 100$ node synthetic electricity networks.
\item We developed new concentration inequalities in the context of stratified sampling, leading to new methods of stratified sample selection, and consequently evaluated the effectiveness of these methods for synthetic data sets and in approximation of the Shapley Value.
\end{itemize}

\noindent The contributions in this thesis are also partially given in the following works:\\

\noindent``The Generalized N\&K Value: An Axiomatic Mechanism for Electricity Trading'' by Mark Burgess, Archie Chapman and Paul Scott\\ International Conference on Autonomous Agents and Multiagent Systems\\ (AAMAS) 2018\\
(accessible: \href{https://ifaamas.org/Proceedings/aamas2018/pdfs/p1883.pdf}{https://ifaamas.org/Proceedings/aamas2018/pdfs/p1883.pdf})\\


\noindent``An Engineered Empirical Bernstein Bound'' by Mark Burgess, Archie Chapman and Paul Scott\\ European Conference on Machine Learning (ECML-PKDD) 2019\\
(accessible: \href{https://ecmlpkdd2019.org/downloads/paper/435.pdf}{https://ecmlpkdd2019.org/downloads/paper/435.pdf})\\


\noindent``Approximating the Shapley Value Using Stratified Empirical Bernstein Sampling'' by Mark Burgess and Archie Chapman\\ International Joint Conference on Artificial Intelligence (IJCAI-2021)\\
(accessible: \href{https://www.ijcai.org/proceedings/2021/0011.pdf}{https://www.ijcai.org/proceedings/2021/0011.pdf})\\


Particularly, we would like to make note some specific contributions: that Archie Chapman provided the idea of using Empirical Bernstein Bounds to derive novel method of sampling the Shapley Value.
That Paul Scott encouraged the use of complementary slackness conditions in solving the KKT conditions to allow the solving the bilevel optimisations in GNK value on electricity network examples.
I would like to claim for myself specific contributions of developing the mathematics for Stratified Empirical Bernstein Method \& Method, and the idea behind extending Neyman \& Kohlberg's value formulation into generalised games, and for developing software and techniques for scaling the GNK computation.
Additionally both Sylvie Thi\'{e}baux, Paul Scott and Arhcie Chapman is credited with providing assisting direction and guidance in the production of this research and its publications.


\section{Thesis outline}
The thesis is arranged into the following chapters:
\begin{enumerate}
\item in Chapter \ref{sec:philosophy}, we give a series of brief philosophical points to provide ethical background for the underlying question of `how \textit{should} electrical energy be traded?'. In this section, we refer to the diversity of conceptions about social Equality, the different ways in which systems can be considered better/worse apart from equality (particularly by notions of Efficiency), and by ethical rules and guidance in proportion to various norms and reference points, introducing notions of envy-freeness, in a broader environmental context. 
\item in Chapter \ref{cha:solutions}, we provide a presentation of some of the core ideas and background of already developed and/or applied solutions to electricity networks, particularly each of these ideas mathematically embody different ideas about distributive ethics. The particular ideas we briefly present, are the Vickrey-Clarke-Groves (VCG) mechanism, the Locational Marginal Pricing (LMP) method, cooperative game theory solutions such as the Shapley Value and The Core, and the approach of bargaining solution concepts such as Nash bargaining. 
\item in Chapter \ref{cha:new_solution}, we develop and explore a new solution concept called the GNK value, that is derived from Shapley Value axioms and relates directly to Nash bargaining, and we compare it against LMP and VCG results in the context of a small-scale simulated electricity network, we make observations about its qualities point-by-point and consider the difficulty in computing it for larger networks.
\item in Chapter \ref{sec:scaling} We address the difficulty in computing the GNK value by exploring two approaches to approximate it, particularly the use of sampling schemes, and the use of a proxy for the inner terms of the GNK value. The advantages and disadvantages of the new approach are discussed in relation to its computed result on a larger electricity network.
The GNK value is discussed with regard to the ethical qualities established in Chapter \ref{cha:background}.
\item in Chapter \ref{chap:stratified_sampling_chapter}, we investigate a range of stratified sampling techniques which were developed for computing the new GNK value. this investigation covered different techniques of conducting stratified sampling by minimising concentration inequalities, and a new technique was resolved called the stratified empirical Bernstein method (SEBM).
\item in Chapter \ref{sec:the_thesis_conclusion} we conclude the thesis by providing reflections of the approach against the situations highlighted in this chapter, as well as outline future work and summarising major accomplishments.
\end{enumerate}





\chapter{Introduction}
\label{cha:intro}

%\section{Thesis Statement}
%\label{sec:thesisstatement}
%I believe A is better than B.

Electricity systems are changing in response to the increased proliferation of variable renewable technologies solar PV systems and wind generators and the continued retirement of traditional generation which provide stabilising inertia, which giving rise to an interest in potential future market structures to facilitate the participation of distributed energy resources (DERs) in grid operations (we focus on the Australian electricity system).
However this problem gives rise to general questions surounding the ethics of market structures and how they could fairly apply in future electricity systems. Particularly the most basic question ``how \textit{should} energy be traded'' is fundamentally a moral question without any easy answer. These considerations are presented in our first chapter \ref{cha:background}.

We give a survey of existing philosophical attitudes towards this question of distributive justice (such as notions about equality, efficiency, proportionality, etc), before presenting a series of ways that these intuitions have been cast into mathematics (including the Vickrey-Clarke-Groves mechanism, Locational Marginal Pricing, the Shapley Value, and Nash bargaining solutions); in chapter \ref{cha:solutions}.

We compare these different methods, and attempt a new synthesis that brings together the best features of each of them; called the `Generalized Neymann and Kohlberg Value' or the GNK-value for short.
We demonstrate the features of this GNK-value against the other mathematical solutions in the context of trading the immediate consumption/generation of power on small sized networks under DC approximation before extending the computation to larger networks.
The GNK value proved to be difficult to compute for large networks, but was approximable with a series of sampling techniques and a proxy method, and we show and discus the features of the GNK value for a large network.
Although the GNK value is a novel solution concept for non-cooperative transferrable utility generalized games that was designed to be flexible in its application to powersystems generally, ultimately the GNK value proved to fail a specific ethical critereon that we deemed to be critical.
Specifically it allowed for participants to be left worse-off for participating, violating the ethical notion of euvoluntary exchange.
All of this is presented in chapter \ref{cha:new_solution}

For the computation of the GNK value to larger networks a range of new and different sampling techniques were developed which iteratively minimise newly derived concentration inequalities on the error of the sampling estimate, these are presented in the final chapter \ref{chap:stratified_sampling_chapter}.


%\section*{Thesis Outline}
%Our Thesis consists of four primary chapters:
%\begin{enumerate}
%\item in Chapter \ref{cha:background}, we a series of brief philosophical points to provide background about the nature of the ethical underlying question of how \textit{should} electrical energy be traded. Particularly we refer to the diversity of conceptions about social Equality, the different ways in which systems can be considered better/worse appart from equality - particularly by notions of Efficiency, and by ethical rules and guidance in proportion to various norms and reference points. 
%\item in Chapter \ref{cha:solutions}, we provide a presentation of some of the core ideas and background of already developed and/or applied solutions to electricity networks, particularly eacho of these ideas mathematically embody different ideas about distributive ethics. The particular ideas we briefly present, are the Vickrey-Clarke-Groves (VCG) mechanism, the Locational Marginal Pricing (LMP) method, cooperative game theory solutions such as the Shapley Value and The Core, and the approach of bargaining solution concepts such as Nash bargaining. 
%\item in Chapter \ref{cha:new_solution}, we develop and explore a new solution concept called the GNK value, that is derived from Shapley Value axioms and relates directly to Nash bargaining, and we compare it against LMP and VCG results in the context of simulated electricity networks. Particularly the advantages and disadvantages of the new approach are discussed and related back to the ethical desirata established in Chpater \ref{cha:background}. The particular difficulty of computing this new GNK value is addressed and overcome by utilising sampling techniques in the presence of a proxy for its most difficult part.
%\item in Chapter \ref{chap:stratified_sampling_chapter}, we reveiew the investigation into sampling techniques which was used to compute the new GNK value. partiularly divergent techniques of conducting stratified sampling by minimising concentration inequalities were investigated and a new technique was resolved called the Stratified empirical Bernstein method (SEBM).
%\end{enumerate}


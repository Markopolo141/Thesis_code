\chapter{Introduction: A background on the future of the Australian grid}
\label{cha:intro}

The Australian electricity grid is adapting to a state of technological changes that are occuring.
And the way in which it should adapt is cause for reflection on the morally ambiguous question at the heart of electrical transactions on the grid - by what mechanisms should it be determined how much people \textit{should} pay or be-paid for the power they consume or generate?

Or, to ask the question with more granularity: when the possible power-flows on an electricity network are valued and influenced by participants differently, 
what is a reasonable electrical outcome for the network and what monetary transactions should occur between the participants in that case? -- and this is to ask: how \textit{should} electricity be traded?\\

Existing electrical power systems implicitly have a historic answer to this question, however electrical power systems have a history of evolution over time, particularly with the introduction of new technologies and new demands and this trend continues today as there are divergent ways that new technology is driving a need for additional changes.

\section{The changing nature of supply, and the introduction of demand elasticity}

One of most notable changes occurring in electricity grids is the continued proliferation of renewable energy technology, and the retirement of older generators - particularly coal powered generators.
For instance, between the fiscal years 2008/2009 - 2018/2019 the volume of renewable electricity generated on Australia's National electricity market (NEM) increased from 18,645 GWh to 44,292 GWh, an increase of approximately 10\% per year, increasing the proportion of renewable energy in the network from 7.5\% to 17\% over the same period
\cite{departmentoftheenvironmentenergy2018}.
This increase was primarily by the deployment of wind and solar generation, with 90\% of solar generation currently being created by small scale solar PV systems, with solar PV generation projected to triple by the year 2030.\cite{australianenergymarketoperatorlimited2018}.
%And this increasing trend is expected to continue, there were ~300,000 installations in the year 2019 - clean energy australia report 2020, 

In contrast to this, Australia is seeing an increasing number number of coal fired power stations being retired from the grid,
with almost a third of Australia's coal-fired power stations closed between 2012 and 2017 \cite{doi:10.1111/1467-8489.12289},
%Most notably including the Hazelwood Power Station in 2017.
and an extra 12 coal fired power plants expected to be retired over the next 30 years.\cite{australianenergymarketoperatorlimited2018}

The Australian electricity market operates multiple markets, but the day-ahead and spot-markets operate on marginal pricing principles, where the most expensive generator that is dispatched to meet demand sets the price which is paid for all dispatched electricity.
In this context, the changes in supply, (the retirement of coal fired power, and its replacement by variable renewable electricity generation) have have created a situation whereby it is increasingly more frequent that more expensive generators are setting the marginal price.
And this is identified as been a part of the reason that the average price of electricity for consumers between 2008 and 2018 has increased by 35\%. \cite{australiancompetitionconsumercommission2018}

This is one example of an emerging problem, and there have been proposed a range of options considered to ensure future stability and efficiency of the grid at providing affordable electricity for customer needs - which are core components of Australia's National Electricity Objective (NEO)\footnote{the NEO is part of Australia's National electricity law (NEL) ``to promote efficient investment in, and efficient operation and use of, electricity services for the long term interests of consumers of electricity with respect to: price, quality, safety and reliability and security of supply of electricity; and the reliability, safety and security of the national electricity system.''}

One presented option is the prospect of a `Demand Response' programs, whereby consumers (large and potentially small) are contracted and paid to reduce their consumption in times of peak demand to avoid dispatch of the more expensive generators which would subsequently set the marginal price, making electricity more expensive for everybody.
Demand response programs are one example of participatory scheme designed to interact with consumers to bring greater regularity to electricity grid. And the offering and accepting of such contracts naturally constitutes a market.

In line with the potential for such a participatory system, there are presently (at time of writing) 12 different Australian Renewable Energy Agency (ARENA) funded Demand response pilot programs across the country (totalling over 100 million dollars of grants).
Additionally the Australian Energy Market Commission (AEMC) is actively drafting rule changes to formally recognise organisations that provide Demand Response services to the grid, as being electricity market participants similar to wholesale generators.\cite{australianenergymarketcommission2020}

The effect of demand-response programs is to attempt to bring the market price down by creating mechanisms where the demand of electricity effectively becomes more elastic, and one of the more direct way of doing this is to produce a market structure to implement demand-responsive consumption against generation, or a `two sided market'.
the idea of a two-sided market is something that is something that recognised as an enduring solution for the Australian electricity grid \cite{australianenergymarketcommission2020} and is currently being scoped out by COAG energy council of Australia's Energy Security Board (ECB) \cite{energysecurityboard2020}.

This is one example of a potential future market system which may become necessary in response to present technology changes; but it is not the only one.

\section{Increasingly variable supply, and the potential of battery storage}

%in addition to small solar PV system installations  there was an array of medium and large scale solar installations.
%In the same time additional household batteries are increasingly being installed (with 22,661 household battery installations in the 2019 year) taking australias household battery storage to 1GW. with an addtional 15 large scale battery projects scheduled for completion in the 2019 year ranging between 1 and 150MW capacity.
%(clean energy australia report 2020)
The increasing number of solar PV systems connected to the Australian grid is changing the characteristic demand profile on the network and its variability; and this has the potential to create issues surrounding grid frequency and stability.

Solar PV systems are often behind-the-meter and have the effect that they change the amount of electricity that consumers require throughout the day from the grid.
And this is particularly true in the middle of the day the abundance of solar energy offsets household electricity requirements and create the ever increasingly severe decrease in demand (the infamous `duck-curve') which is identified to be one source of potential future difficulty.
Not only does solar PV change the average demand profile on the network but it also recognised as a source of variability; because solar output is linked to weather there is the ready potential for large swings of localised solar power output, potentially creating issues of future frequency and voltage control which may need to be addressed.\cite{australianenergymarketoperatorlimited2018}
These demand side changes occur alongside supply side changes such as the retirement of coal generation capacity with their stabilising generation inertia, and continued uptake of industrial wind generation capacity, and these compound potential problems associated with stability.

One of the primary ways in which sudden changes in supply/demand is manifest on the grid is by frequency drift.
In order to keep synchrony the frequency of the grid needs to be closely monitored and kept on a single frequency as it is an essential stability requirement for the grid.
When there is an increase in supply or a decrease in demand, there is an excess of power on the network which causes a decrease on the electromagnetic drag on traditional generation rotors, causing them to speed up, leading to a increase in their frequency of rotation and frequency of the power they generate.
Conversely a decrease in supply, or increase in demand, causing a decrease in frequency. With the rate at which these frequency increase/decrease occurs being the inertia of the generator.

However the decrease in traditional generation technologies which carry inertia, is expected to cause the system frequency to be more sensitive to supply/demand variability, just as that variability is set to increase due to the increase in renewable technologies.
There are multiple prospective ways of ameliorating this problem, such as to attempt to build artificial inertia into renewable generators, and another is to requisition generation capacity specifically to maintain system frequency.\cite{doi:10.1002/2050-7038.12128}

Presently there exist a range of electricity markets within the Australian System, including a day-ahead market, a spot-market and several capacity markets specifically for the stabilisation of grid frequency.
Some of these ancillary service markets are the Frequency control ancillary services (FCAS) markets, and there exist six of them, particularly for bidding for the contracts to deposit and/or withdraw power from the electricity system, at an upper, middle and lower timeframe for the response (6 second, 60 seconds, 5 minute respectively). \cite{RIESZ201586}
In light of the technological changes that are being realised it is anticipated that these frequency response markets will become increasingly important to maintain grid stability.

However, one limitation of traditional generating technologies is that they are often unable to respond quickly to assist in the maintaining of system frequency, and therefore unable to support grid frequency or bid into FCAS markets.
And while there does exist some technology for rapid response gas generators \cite{GONZALEZSALAZAR20181497}, the more promising renewable technology is battery technologies, which have characteristically fast response times.
While there does presently exist some large institutional grid connected batteries, a large number of batteries currently connected to the grid are residential small scale batteries often connected to solar PV systems.
Additionally there is the expected increase in the adoption of electric vehicles (EVs) which are predicted to be increasingly adopted over the coming decades with predictions ranging upto 4.5 million EVs on Australian roads by 2040 \cite{australianenergymarketoperatorlimited2019}.
In this way there is a great potential for batteries and electric vehicles to be a source of future grid stability.
%EVs, are predicted to account for 17.7\% of vehicles on australian roads by 2036 (Electric Vehicles Insights AEMO)) 

However there does not currently exist any unified national infrastructure to facilitate these distributed energy resources (DERs) such as EVs and small batteries participating in grid stability.
As presently, network participants are payed for the power they consume/generate largely behind their own meter, inducing them to utilise their DERs to optimise their own energy consumption.
One current avenue being explored to create such infrastructure is the creation of virtual power plants (VPPs) as large scale aggregators of DER power capabilities. These are currently being developed and are the subject of experimentation. \cite{australianenergymarketoperatorlimited20188}

%Another avenue currently being ivestigated is contracting strategic reserve power for grid stability. which are currently being formalised into the Reliability and Emergency Reserve Trader (RERT) rules.

Virtual Power plants are another example of market structure that potentially could emerge in response to current technological trends.

\section{DERs and the potential for a prosumer era}

Another change that is happening, is that increasingly not only are consumers generating their own electricity via solar panels, but they are also expected to be increasingly storing that energy as well; leading to an increasing level of consumer energy independence and the potential attractiveness of going `off-grid'.
However, there may be future role for electricity consumers (or so called `prosumers') not only to participate in network support and have access to market structures; but also ideas that might facilitate consumers to induce larger and more social benefits, such as directly sharing their electricity with each other.
For instance, the idea that the grid might be able to support consumers selling their excess power and storage capacity to each other - instantiating a so called peer-to-peer electricity trading (P2P) system.

However there exist differing visions of what structure a future electricity system might have to support trading between prosumers.\cite{Parag2016}.
Such as ideas about how these prosumers might trade directly or indirectly between themself and the wider grid; such perhaps between P2P trading and/or aggregation into many larger virtual power plants (VPPs) \cite{Morstyn2018}.

However the future vision of incorporating prosumers on distribution energy networks into the national electricity market, poses a range of gains and challenges, particularly with regards to voltage and frequency management on distribution networks.
Such challenges include, providing grid stability by securing timely reactive and real power supply to stabilise voltages and frequency, grid robustness such as blackout protection and islanding, improving system efficiency by minimising long-distance transmission costs, and facilitating the advent of a green electricity network, by fairly and equitably managing the interaction between prosumer's devices while preserving their privacy, in a scalable way while minimising the complexity of the management of such devices. \cite{BELL2018765}

The continued proliferation of DERs is a technological change that is expected to create investigation into these future visions of the grid.

\section{A summary of the future of the Australian grid}

%There are various ways that people are investigating the changes required to the current electricity network.

%Most notably in 2017 Dr Alan Finkel released his review on the future of the Australian electricity "Independent Review into the Future Security of the National Electricity Market" - informally called `the Finkel review', containing a host of 50 recommendations and changes to ensure future grid reliability and operations.

%Among these 50 recommendations, 49 of them were formally accepted by the Turnbull government, including such as: a review of potential frameworks for DER participation, assessment of the need for strategic reserve and/or day-ahead stability market, the review and recommendation of a demand response mechanism, and the creation of an Energy Security Board to implement an Strategic energy plan.

%Some of these recommendations have matterially come to pass, such as the need for contracting strategic reserve power for grid stability being formalised into the Reliability and Emergency Reserve Trader (RERT) rules; and the Energy Security Board has been created and is presently investigating possible price resonsive demand programs.
%particularly the production of a market structure to produce demand-responsive supply/consumption against generation (potentially even extending to normal users of electricity systems) a `two sided market' is something that is being scoped out by COAG energy council australia (ECA) 

%A consequence of the changing dynamics of the grid, is that these combinations of changes has resulted in inordinate increase in AEMO direct intervations (AEMO) , and AER rule changes (KPMG) in order to address the emerging situations.
%More generally, this increased variability of demand and supply are part of a changing electricity system that need to be managed and addressed, and there are a variety of potential options.

%One example of such plans is the current COGATI investigation, exploring changes in the NEM that could be made to further incentivse generation and transmission at grid critical locations to minimise congestion experienced on the grid; ie. incentivising the most appropriate network generation and transmission.
%Another broad example of such an investigation isn the so called `NEM2025' plan directly investigates possible changes to market structure to NEM in order to ensure future stable grid, particularly mentioning the potential of utilising DERs.



%One, instance of a design decision is who should be participating on such future markets, particularly the prospect that consumers of electricity could participate in some kind of electricity market (ECA) is up for discussion.


%And evaluating the potential for implementing such a system is potentially part of the NEM2025 investigation.

%peer-to-peer electricity trading and managment on small-scale networks - sometimes called microgrids - have been developed in various contexts around the world, including Australia
%These distributed energy resources exist on distribution networks where there is occasional problems with voltage management, where 
%solar inverter reactive power stability is not often utilised. (Smart Grid and its future perspectives in Australia)


%https://arena.gov.au/knowledge-innovation/distributed-energy-integration-program/


The way that electricity is being produced and consumed is expected to change in the future, from a system which supports the supply of few big generation companies to many small consumers, to being a system in which prosumers potentially exchange energy with each other.
And there is an increasing interest in the design of new market systems that are appropriate for the future electricity grid.

The technological changes witnessed (the continued proliferation of renewable and the closing of traditional generators) is seen as potentially leading to future problems associated with frequency and grid stability, in which batteries and other distributed energy resources (DERs) are seen as being important; and new market structures are being considered to provide a platform for them to participate in grid stability.

In this context, there are many potentially important engineering considerations in the operation of powergrids which may bear importance for the exchange of energy between such prosumers - such as voltage rises and line limits, real and reactive power compensation, phase connections, network topologies etc.
And it is hoped that a properly designed system might be extensible to be able to accommodate these technical considerations where applicable.

Potential future markets are subject to many requirements, outside of simply delivering a reliable and cheap electricity service to consumers, and sometimes they are even directly quoted as subject to ethical design considerations, such as implementing a ``Level playing field'' where:

"all competitors, irrespective of their size or financial strength, get equal opportunity to compete. It is not enough if all players play by the same rules. The rules must accommodate the needs of all, whether small or large, so the market is free of impediments to smaller players." \cite{australianenergymarketoperatorlimited2018}

Ethical criteria such as this, as well as political and social implications bring into question the nature of ethics, and particularly of distributive justice. 
Thus our investigation is to explore and evaluate the general question of what market structures \textit{should} be implemented.
%In the past, it was held that inducing competition among generation companies was desirable, however it is important to ask if inducing competition among electrically generating prosumers is equally a good idea.
%A part of the answer comes in the likely result: would a competitive system induce outcomes which are socially desirable and/or serve to promote social equality?
%A major component of this question lies in the formulation of the arena in which the competition takes place; which is the primary design question.
%And the purpose of this research is an attempt to explore a portion of the space of possible arenas, and evaluate it against desirable ethics.


\section{Thesis Outline}
The document is arranged into the following chapters:
\begin{enumerate}
\item in Chapter \ref{sec:philosophy}, we a series of brief philosophical points to provide background about the nature of the ethical underlying question of how \textit{should} electrical energy be traded. Particularly we refer to the diversity of conceptions about social Equality, the different ways in which systems can be considered better/worse appart from equality - particularly by notions of Efficiency, and by ethical rules and guidance in proportion to various norms and reference points. 
\item in Chapter \ref{cha:solutions}, we provide a presentation of some of the core ideas and background of already developed and/or applied solutions to electricity networks, particularly eacho of these ideas mathematically embody different ideas about distributive ethics. The particular ideas we briefly present, are the Vickrey-Clarke-Groves (VCG) mechanism, the Locational Marginal Pricing (LMP) method, cooperative game theory solutions such as the Shapley Value and The Core, and the approach of bargaining solution concepts such as Nash bargaining. 
\item in Chapter \ref{cha:new_solution}, we develop and explore a new solution concept called the GNK value, that is derived from Shapley Value axioms and relates directly to Nash bargaining, and we compare it against LMP and VCG results in the context of simulated electricity networks. Particularly the advantages and disadvantages of the new approach are discussed and related back to the ethical desirata established in Chapter \ref{cha:background}. The particular difficulty of computing this new GNK value is addressed and overcome by utilising sampling techniques in the presence of a proxy.
\item in Chapter \ref{chap:stratified_sampling_chapter}, we investigate a range of stratified sampling techniques which was used to compute the new GNK value. partiularly divergent techniques of conducting stratified sampling by minimising concentration inequalities were investigated and a new technique was resolved called the Stratified empirical Bernstein method (SEBM).
\end{enumerate}





\chapter{Introduction}
\label{cha:intro}

%\section{Thesis Statement}
%\label{sec:thesisstatement}
%I believe A is better than B.


\section*{Thesis Outline}
Our Thesis consists of four primary chapters:
\begin{enumerate}
\item in Chapter \ref{cha:background}, we a series of brief philosophical points to provide background about the nature of the ethical underlying question of how \textit{should} electrical energy be traded. Particularly we refer to the diversity of conceptions about social Equality, the different ways in which systems can be considered better/worse appart from equality - particularly by notions of Efficiency, and by ethical rules and guidance in proportion to various norms and reference points. 
\item in Chapter \ref{cha:solutions}, we provide a presentation of some of the core ideas and background of already developed and/or applied solutions to electricity networks, particularly eacho of these ideas mathematically embody different ideas about distributive ethics. The particular ideas we briefly present, are the Vickrey-Clarke-Groves (VCG) mechanism, the Locational Marginal Pricing (LMP) method, cooperative game theory solutions such as the Shapley Value and The Core, and the approach of bargaining solution concepts such as Nash bargaining. 
\item in Chapter \ref{cha:new_solution}, we develop and explore a new solution concept called the GNK value, that is derived from Shapley Value axioms and relates directly to Nash bargaining, and we compare it against LMP and VCG results in the context of simulated electricity networks. Particularly the advantages and disadvantages of the new approach are discussed and related back to the ethical desirata established in Chpater \ref{cha:background}. The particular difficulty of computing this new GNK value is addressed and overcome by utilising sampling techniques in the presence of a proxy for its most difficult part.
\item in Chapter \ref{chap:stratified_sampling_chapter}, we reveiew the investigation into sampling techniques which was used to compute the new GNK value. partiularly divergent techniques of conducting stratified sampling by minimising concentration inequalities were investigated and a new technique was resolved called the Stratified empirical Bernstein method (SEBM).
\end{enumerate}


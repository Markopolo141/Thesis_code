\definecolor{wwwwww}{rgb}{0.4,0.4,0.4}
\definecolor{ccqqqq}{rgb}{0.8,0,0}
\definecolor{qqqqff}{rgb}{0,0,1}
\definecolor{wqwqwq}{rgb}{0.3764705882352941,0.3764705882352941,0.3764705882352941}
\definecolor{cqcqcq}{rgb}{0.7529411764705882,0.7529411764705882,0.7529411764705882}
\definecolor{strategy}{rgb}{0.7,0.7,0.9}

\begin{figure}[tb]
\centering
\begin{tikzpicture}[line cap=round,line join=round,>=triangle 45,x=1cm,y=1cm]
%strategy polygon
\clip(-3,-3) rectangle (4,4);
\draw[fill, opacity=1, color=strategy] (2,1) -- (1,2) -- (-2,-1) -- (-1,-2);
%grid and axes
\draw [color=cqcqcq,dotted, xstep=1cm,ystep=1cm] (-3,-3) grid (4,4);
\draw[->,color=wqwqwq] (-3,0) -- (4,0);
\draw[->,color=wqwqwq] (0,-3) -- (0,4);
%grid numbers
\foreach \x in {-3,-2,-1,1,2,3,4}
\draw[shift={(\x,0)},color=wqwqwq] (0pt,2pt) -- (0pt,-2pt) node[below] {\scalebox{0.35}{$\x$}};
\foreach \y in {-3,-2,-1,1,2,3,4}
\draw[shift={(0,\y)},color=wqwqwq] (2pt,0pt) -- (-2pt,0pt) node[left] {\scalebox{0.35}{$\y$}};
\draw[color=wqwqwq] (-2pt,-3pt) node[right] {\scalebox{0.35}{$0$}};
%red lines
\draw [line width=1.0pt,color=ccqqqq] (-1,4) -- node[above,sloped,pos=0.8]{\tiny Pareto Frontier} (4,-1);
%\draw [line width=0.6pt,color=ccqqqq] (-3,0) -- (0,-3);
%grey lines
\draw [dash pattern=on 1pt off 1pt,color=wwwwww] (-2,-1) -- (3,4);
%\draw [dash pattern=on 1pt off 1pt,color=wwwwww] (1,-5) -- (6,0);
%hyperbola
\draw[samples=50, domain=-0.5:4,smooth,color=wwwwww,line width=0.5pt,dash pattern=on1pt off 1pt] 
plot ({\x},{(9/(\x+2))-1});

\begin{tiny}
%labels
%\draw [color=black] (1.5,-5) node {maxmin};
%\draw [color=black] (3.572727273,-1.272727273) node {coco};
%\draw [color=black] (4.4,-2) node {a-coco};
%\draw [color=qqqqff] (1.5,1) node {maxmax};
%\draw [color=black] (0.5,-4.4958677686) node {minimax};
\draw [color=wqwqwq] (0.4,1.7) node [label={[align=left]Row Player\\utility}]{};
\draw [color=wqwqwq] (1.9,0.15) node {Column Player utility};
%points
\draw [fill=qqqqff] (2,1) circle (1.7pt);
\draw [fill=qqqqff] (1,2) circle (1.7pt);
\draw [fill=black] (-2,-1) circle (1.7pt);
\draw [fill=qqqqff] (-1,-2) circle (1.7pt);
%\draw [fill=wwwwww] (1,-5) ++(-2.5pt,0 pt) -- ++(2.5pt,2.5pt)--++(2.5pt,-2.5pt)--++(-2.5pt,-2.5pt)--++(-2.5pt,2.5pt);
%\draw [fill=wwwwww] (0.0495867769,-4.4958677686) ++(-2.5pt,0 pt) -- ++(2.5pt,2.5pt)--++(2.5pt,-2.5pt)--++(-2.5pt,-2.5pt)--++(-2.5pt,2.5pt);
%\draw [color=black,line width=0.8pt] (3.272727273,-1.272727273)-- ++(-3pt,0 pt) -- ++(6pt,0 pt) ++(-3pt,-3pt) -- ++(0 pt,6pt);
%\draw [color=black,line width=0.8pt] (4,-2)-- ++(-3pt,0 pt) -- ++(6pt,0 pt) ++(-3pt,-3pt) -- ++(0 pt,6pt);
\end{tiny}

\end{tikzpicture}
\caption[Utility diagram for example game.]{The potential outcomes for the TU example game (Equation \ref{eq:example_game1}). Red lines show all potential outcomes of pure strategies with TU. Blue points show non-TU pure strategy outcomes. The blue polygon shows space of outcomes for non-TU mixed strategies, The outcome that corresponds to the minimax strategy of the zero-sum component is shown in black. The Nash Bargaining value is shown on the pareto frontier as the feasible outcome that maximises the product of utilities above the minimax disagreement-point, the hyperbola of this maximum product is shown.}
\label{fig:graph1_utilities}
\end{figure}

\chapter*{Abstract}
\addcontentsline{toc}{chapter}{Abstract}
\vspace{-1em}
The electricity grid is seen as an evolving system of increasingly interconnected and complex devices, providing a structured platform for the exchange of power and services between people.
However at the heart of this evolving construction there is a morally ambiguous question - how much \textit{should} people pay or be-paid between themself for the power they consume or generate for each other?

To ask the same question with more granularity: when the possible power-flows on an electricity network are valued and influenced by participants differently, what is a reasonable cooperative outcome for the network and what monetary transactions should occur between the participants in that case?

We review the ethical considerations that frame the question, before we give consideration of a range of background solutions to the problem.
Then we proceed to give a novel solution concept for the question called the `GNK value', which we compare against others.
We investigate the implementation and consequences of different solutions for large scale networks under DC-approximation powerflows, where it is found that our novel solution unfortunately fails some ethical requirements.

We then conclude by presenting a sophisticated and novel method of stratified sampling that was developed for approximating the GNK value for larger networks.

%%% Local Variables: 
%%% mode: latex
%%% TeX-master: "paper"
%%% End: 

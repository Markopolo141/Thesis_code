\chapter*{Abstract}
\addcontentsline{toc}{chapter}{Abstract}
\vspace{-1em}


Electricity systems are changing in response to the increased proliferation of variable renewable technologies such as solar PV systems and wind generation, leading to the continuing retirement of traditional generation which provide stabilising inertia.
These changes create an imperative to consider potential future market structures to facilitate the participation of distributed energy resources (DERs) in grid operations.
However this investigation gives rise to general questions surrounding the ethics of market structures and how they could be fairly applied in future electricity systems. Particularly the most basic question ``how \textit{should} energy be traded'' is fundamentally a moral question without any easy answer.

We give a survey of existing philosophical attitudes towards this question of distributive justice (such as notions about equality, efficiency, proportionality, etc), before presenting a series of ways that these intuitions have been cast into mathematics: including the Vickrey-Clarke-Groves mechanism, Locational Marginal Pricing, the Shapley Value, and Nash bargaining solutions.

We compare these different methods, and attempt a new synthesis that brings together the best features of each of them; called the `Generalised Neyman and Kohlberg Value' or the GNK-value for short.
The GNK value was developed as a novel solution concept for non-cooperative transferable utility generalised games, and as such it that was designed to be flexible in its application to various aspects of powersystems.
We demonstrate the features of this GNK-value against the other mathematical solutions in the context of trading the immediate consumption/generation of power on small sized networks under DC approximation before extending the computation to larger networks.
The GNK value proved to be difficult to compute for large networks, but was approximable with a series of sampling techniques and a proxy method.
Because of this we were able to show and discus the features of the GNK value for a large network.

Ultimately the GNK value proved to fail a specific ethical criterion that we deemed to be critical, specifically it allowed for participants to be left worse-off for participating, violating the ethical notion of `euvoluntary exchange' and `individual rationality'.

Additionally, a range of new and different sampling techniques for stratified random sampling were developed which iteratively minimise newly derived concentration inequalities on the error of the sampling estimate.
These techniques were developed to assist in the computation of the GNK value to larger networks, and they were evaluated at sampling synthetic data, and in computation of the Shapley Value of cooperative game theory.




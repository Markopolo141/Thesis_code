\chapter*{Abstract}
\addcontentsline{toc}{chapter}{Abstract}
\vspace{-1em}


Electricity systems are facing the pressure to change in response to the incorporation and effects of new technology.
Particularly the increased proliferation of variable renewable technologies such as solar PV systems and wind generation, leading to the retirement of traditional generation technologies which provide stabilising inertia.
These changes create an imperative to consider potential future market structures to facilitate the participation of distributed energy resources (DERs) in grid operation.
However, this gives rise to general questions surrounding the ethics of market structures and how they could be fairly applied in future electricity systems. Particularly the most basic question ``how \textit{should} energy be traded'' is fundamentally a moral question without any easy answer.
We give a survey of existing philosophical attitudes towards this question of distributive justice before presenting a series of ways that these intuitions have been cast into mathematics: including the Vickrey-Clarke-Groves mechanism, Locational Marginal Pricing, the Shapley Value, and Nash bargaining solutions.

We compared these different methods, and attempted a new synthesis that brought together the best features of each of them; called the `Generalised Neyman and Kohlberg Value' or the GNK-value for short.
The GNK value was developed as a novel bargaining solution concept for many player non-cooperative transferable utility generalised games, and thus was designed to be flexible in its application to various aspects of powersystems.
We demonstrated the features of the GNK-value against the other mathematical solutions in the context of trading the immediate consumption/generation of power on small sized networks under linear-DC approximation before extending the computation to larger networks.
The GNK value proved to be difficult to compute for large networks, but was shown to be approximable with a series of sampling techniques and a proxy method, in this way the GNK value was thus able to be solved for a $\sim 100$ bus network.
The GNK value was ethically compared to other mechanisms with the unfortunate discovery that it allowed for participants to be left worse-off for participating, violating the ethical notion of `euvoluntary exchange' and `individual rationality'; but offered as an interesting discovery in the space of transferable utility generalised games.

For sampling the GNK value, there was a range of new and different sampling techniques developed for stratified random sampling which iteratively minimise newly derived concentration inequalities on the error of the sampling.
These techniques were developed to assist in the computation of the GNK value to larger networks, and they were evaluated in the context of sampling synthetic data, and in computation of the Shapley Value of cooperative game theory.
These new sampling techniques were demonstrated to be comparable to the more orthodox Neyman sampling method despite not having access to stratum variances.



